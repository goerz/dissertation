% vim: ft=tex spelllang=de_20

Wir befinden uns an der Schwelle einer Revolution zu einer Quantentechnologie,
die nicht nur auf der passiven Nutzung von Quanteneffekten, sondern auf ihrer
aktiven Kontrolle beruht. An vorderster Front beinhaltet dies die Realisierung
eines Quantencomputers. Das Kodieren von Informationen in Quantenzuständen als
``Qubits'' erlaubt es, Verschränkung und Quantensuperposition zu nutzen, um
Rechnungen durchzuführen, die auf einem klassischen Computer unpraktikabel sind.
Eine zentrale Schwierigkeit ist es dabei, Dekohärenz zu vermeiden -- der Verlust
von Quanteneigenschaften aufgrund ungewollter Wechselwirkung mit der Umgebung.
Diese Arbeit thematisiert die Realisierung verschränkender Zwei-Qubit-Gatter,
die sowohl gegenüber Dekohärenz als auch klassischen Störeinflüssen robust sind.
Sie behandelt dabei drei Aspekte: die Nutzung effizienter numerischer Methoden
zur Simulation und optimaler Kontrolle offener und geschlossener
Quantensysteme, die Rolle fortgeschrittener Optimierungsfunktionale zur
Begünstigung von Robustheit, sowie die Anwendung dieser Techniken auf zwei
führende Umsetzungen von Quantencomputern, gefangene Atome und
supraleitende Schaltkreise.

Nach einem Überblick über die theoretischen und numerischen
Grundlagen beginnt der zentrale Teil dieser Arbeit mit der Idee einer
Ensembleoptimierung, um Robustheit sowohl gegenüber klassischen Fluktuationen als
auch Dekohärenz zu erreichen. Für das Beispiel eines kontrollierten
Phasengatters auf gefangenen Rydberg-Atomen wird gezeigt, dass Gatter erreichbar
sind, die um mindestens eine Größenordnung robuster sind als der beste bekannte
analytische Ansatz. Darüber hinaus bleibt diese Ro\/bust\/heit selbst dann
erhalten, wenn die Gatterdauer signifikant gegenüber der kurzmöglichsten Dauer
des analytischen Gatters verkürzt wird.

Supraleitenden Schaltkreise sind eine besonders vielversprechende Architektur
zur Implementierung eines Quantencomputers. Ihre Flexibilität wird durch
Optimierungen sowohl für diagonale als auch nicht-diagonale Gatter gezeigt. Um
Robustheit gegenüber Dekohärenz zu gewährleisten, ist es essentiell, das Gatter
in so kurzer Zeit wie möglich zu realisieren. Das Erreichen dieses Ziels wird
durch die Optimierung hin zu einem beliebigen perfekten Verschränker
erleichtert, basierend auf einer geometrischen Theorie der Zwei-Qubit-Gatter.
Für das Beispiel supraleitender Qubits wird gezeigt, dass dieser Ansatz zu
kürzeren Gatterzeiten, höheren Fidelitäten, sowie schnellerer Konvergenz führt,
im Vergleich zur Optimierung hin zu vorausbestimmten, festen Zwei-Qubit-Gattern.

Eine Optimierung im Liouville-Raum zur sauberen Berücksichtigung von
Dekohärenzeffekten ist mit erheblichen numerischen Herausforderungen verbunden,
da die Dimension im Vergleich zum Hilbert-Raum quadratisch wächst. Es kann
allerdings gezeigt werden, dass es für ein unitäres Optimierungsziel
ausreichend ist, höchstens drei Zustände anstelle der vollen Basis des
Liouville-Raums zu propagieren. Sowohl für das Beispiel gefangener Rydberg-Atome
also auch für supraleitende Qubits wird die erfolgreiche Optimierung von
Quantengattern gezeigt, mit einem numerischen Aufwand, der weit unterhalb der
bisher angenommenen Untergrenze liegt. Insgesamt zeigen die Ergebnisse dieser
Arbeit zu einem umfassenden Gerüsts zur Optimierung robuster Quantengatter, und
bereiten den Weg für die mögliche Realisierung eines Quantencomputers.
