% the acknowledgments section

First and foremost, I would like to thank my advisor, Christiane Koch, for her
constant support and guidance. She has been truly exceptional in her dedication
and patience, giving me the opportunity to work on a wide range of fascinating
problems in quantum computing and optimal control, and to follow my passion for
computing. Also, she has been providing me the chance, through
several extended research visits and workshops, e.g.\ the KITP program on
Control of Complex Quantum Systems, to exchange ideas and collaborate
with a great number of wonderful people.

I would like to thank Birgitta Whaley for letting me spend a combined
few months in her group at Berkeley. Her guidance has also been
invaluable. Many of the results presented in
this thesis have originated from work done as part of this collaboration.
Felix Motzoi has taught me much about superconducting qubits.
Jon Aytac and especially Eli Halperin spent much effort in analyzing the
analytical schemes for Rydberg gates.

My thanks also goes to all the past and present members of the group in Kassel,
Daniel Reich, Giulia Gualdi, Wojciech Skomorowski, Michał Tomza, Martin
Berglund, and Esteban Goetz. Daniel and Giulia in particular have contributed
much to my work. Daniel's deep understanding of the mathematics of optimal
control has been the basis of the results in chapter~\ref{chap:3states}.
Giulia has taught me much about open quantum systems.
As part of their Bachelor theses, Daniel Basilewitsch and Lutz Marder helped in
some of the calculations for the transmon system, and in the implementation of
the Newton propgator.

On a personal level, I would like to thank Cathy Kudlick for being a source of
inspiration and comfort. Lastly, and especially, I thank my parents, Christoph
Goerz and Lieve Gevaert-Goerz, to whom this thesis is dedicated, for their
endless support.
