% vim: ft=tex spelllang=en

We are currently at the cusp of a revolution in quantum technology that relies
not just on the passive use of quantum effects, but on their active control.
At the forefront of this revolution is the implementation of a
quantum computer. Encoding information in quantum states as ``qubits'' allows to
use entanglement and quantum superposition to perform calculations that are
infeasible on classical computers.  The fundamental challenge in the realization
of quantum computers is to avoid decoherence -- the loss of quantum properties --
due to unwanted interaction with the environment. This thesis
addresses the problem of implementing entangling two-qubit quantum gates that
are robust with respect to both decoherence and classical noise. It covers three
aspects: the use of efficient numerical tools for the simulation and optimal
control of open and closed quantum systems, the role of advanced optimization
functionals in facilitating robustness, and the application of these techniques
to two of the leading implementations of quantum computation, trapped atoms and
superconducting circuits.

After a review of the theoretical and numerical foundations, the central part of
the thesis starts with the idea of using ensemble optimization to achieve
robustness with respect to both classical fluctuations in the system parameters,
and decoherence. For the example of a controlled phasegate implemented with
trapped Rydberg atoms, this approach is demonstrated to yield a gate that is at
least one order of magnitude more robust than the best known analytic scheme.
Moreover this robustness is maintained even for gate durations significantly
shorter than those obtained in the analytic scheme.

Superconducting circuits are a particularly promising architecture for the
implementation of a quantum computer. Their flexibility is demonstrated by
performing optimizations for both diagonal and non-diagonal quantum gates. In
order to achieve robustness with respect to decoherence, it is essential to
implement quantum gates in the shortest possible amount of time.  This may be
facilitated by using an optimization functional that targets an arbitrary
perfect entangler, based on a geometric theory of two-qubit gates. For the
example of superconducting qubits, it is shown that this approach leads to
significantly shorter gate durations, higher fidelities, and faster convergence
than the optimization towards specific two-qubit gates.

Performing optimization in Liouville space in order to properly take into
account decoherence provides significant numerical challenges, as the dimension
increases quadratically compared to Hilbert space. However, it can be shown that
for a unitary target, the optimization only requires propagation of at most
three states, instead of a full basis of Liouville space. Both for the example
of trapped Rydberg atoms, and for superconducting qubits, the
successful optimization of quantum gates is demonstrated, at a significantly
reduced numerical cost than was previously thought possible. Together, the
results of this thesis point towards a comprehensive framework for the
optimization of robust quantum gates, paving the way for the future
realization of quantum computers.
