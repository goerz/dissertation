\chapter{Three states are sufficient to assess
  whether a desired target unitary is implemented}
\label{app:3states_proof}

In the following we discuss the functional $J_{dist}$,
\begin{equation}
  J_{dist} = \sum_{i=1}^{3}
   \Tr\left[\left(\Op O\Op\rho_{i}(0)\Op
       O^{\dagger}-\Op\rho_{i}\left(T\right)\right)^2\right]\,, 
  \label{eq:3st_app_f_dist}
\end{equation}
which is built on the distance between the ideal and actual
states at time $T$. It attains its global minimum, $J_{dist}=0$,
if and only if the initial states defined in Section~\ref{sec:3st_oct},
$\Op\rho_i(0)$ for $i=1,2,3$, are mapped to their correct target
states, i.e., fulfill condition~\eqref{eq:3st_cond}.
This functional motivates the use of the optimization functional
$J_T$, Eq.~\eqref{eq:3st_functional},
which is also built on only three states, as discussed in
Sec.~\ref{sec:3_states_proof_funct}. $J_T$ and $J_{dist}$ differ in that
$J_T$ evaluates the Hilbert-Schmidt
products, i.e., the projections of the actual onto the ideal states
instead of the trace distance.
The construction of $J_{dist}$, and subsequently $J_T$,
is rationalized by a theorem for unital, i.e., identity preserving, dynamical
maps. Specifically, the theorem states that a \emph{complete and totally
rotating} set of density matrices is sufficient to determine whether 
a given time evolution is unitary. The
functional~\eqref{eq:3st_app_f_dist} exploits
the further property of a complete and totally rotating set
of density matrices to differentiate any two
unitaries~\cite{ReichKochPRA13}. The theorem for unital dynamical maps
is proven in Sec.~\ref{sec:3_states_proofs}. 

It should be stressed that we use $J_T$, Eq.~\eqref{eq:3st_functional}, 
instead of $J_{dist}$, Eq.~\eqref{eq:3st_app_f_dist},
as optimization functional. This is motivated by the convexity of
$J_T$ which implies a much more favorable convergence behavior than
would be obtained with a non-convex functional~\footnote{Optimization
  using non-convex functionals is possible but requires additional
  terms in the update equation for the field to preserve monotonicity
  of the convergence~\cite{ReichKochJCP12}.
}. Mathematically, however, the two functionals are not equivalent. 
This is illustrated by rewriting a single summand of $J_{dist}$,
Eq.~\eqref{eq:3st_app_f_dist}, and comparing it to 
the corresponding term in $J_T$, Eq.~\eqref{eq:3st_functional},
\begin{equation}
  \Tr\left[\left(\Op O\Op\rho_{i}(0)\Op
      O^{\dagger}-\Op\rho_{i}\left(T\right)\right)^2\right] 
  = \Tr\left[\left(\Op O\Op\rho_{i}(0)\Op
      O^{\dagger}\right)^2\right]-2\Tr\left[\Op O\Op\rho_{i}(0)\Op
    O^{\dagger}\Op\rho_{i}\left(T\right)\right] 
   + \Tr\left[\left(\Op\rho_{i}\left(T\right)\right)^2\right]\,.
  \label{eq:3st_functional_expansion}
\end{equation}
The first term on the rhs of Eq.~\eqref{eq:3st_functional_expansion} is
constant and thus irrelevant. The second term corresponds 
to the Hilbert-Schmidt overlap as used in $J_T$,
Eq.~\eqref{eq:3st_functional}, up to a prefactor. The main difference between 
$J_T$ and $J_{dist}$ is due to the third term, 
the purity of the propagated density matrix. 
$J_T$ neglects this term. This could potentially disturb convergence,
because  the functional value of $J_T$ can be decreased 
by (artificial) purification of the totally mixed states $\Op\rho_1$ and
$\Op\rho_3$, cf. Eq.~\eqref{eq:3st_rhos}, instead of being decreased due
to the desired approach to the target. 
Note that this problem can only arise for mixed states, i.e., when
using the minimal set of states. For the reduced sets consisting of $d+1$,
respectively $2d$, states, propagation starts from pure states, and the global
minimum of $J_T$ is identical to the global minimum of $J_{dist}$.
Note that the problem of artificial purification is purely
hypothetical and was never encountered 
in our optimizations --  'artificial purification traps' in
the optimization landscape of the functional  
$J_T$ with mixed states are apparantly avoided.

\section{Construction of the functional}
\label{sec:3_states_proof_funct}

We first define the concept of \emph{complete and total
rotation}, which we then use to formulate the required theorem.
Let $\mathcal{H}$ be a Hilbert space with dimension $N$. Let
$\mathcal{A}$ be a set of
$N$ one-dimensional orthogonal projectors. A one-dimensional projector
is a projector with rank one, which means that its spectrum consists of a
single eigenvalue equal to one with all remaining eigenvalues being zero.

\textbf{Definition}:
A one-dimensional projector $\Op P_{TR}$ is called \emph{totally rotated} with
respect to the set $\mathcal{A}$ if $\forall \Op P\in\mathcal{A}:\ 
\Op P_{TR}\Op P\neq0$.

\textbf{Definition}:
A set of density operators, $\left\{ \Op\rho_{i}\right\}$ with
$\Op\rho_{i}\in\mathcal{H}\otimes\mathcal{H}$, is called \emph{complete} if
the set $\mathcal{P}$ of projectors onto the eigenspaces of
$\left\{ \Op\rho_{i}\right\}$ contains exactly
$N$ one-dimensional orthogonal projectors.

\textbf{Definition}:
A set of density operators, $\left\{ \Op\rho_{i}\right\}$ with
$\Op\rho_{i}\in\mathcal{H}\otimes\mathcal{H}$, is called \emph{complete and
totally rotating} if it is complete and there exists a
one-dimensional projector in $\mathcal{P}$ that is totally rotated
with respect to the orthogonal set of one-dimensional orthogonal
projectors necessary for completeness.


\textbf{Theorem 1}:
Let $\text{DM\ensuremath{\left(N\right)}}$ be the space of $N\times N$
density matrices and $\mathcal{D}:\text{DM\ensuremath{\left(N\right)}}
\mapsto\text{DM\ensuremath{\left(N\right)}}$ a
dynamical map. The following three statements are equivalent:
\begin{enumerate}
\item $\mathcal{D}$ is unitary, i.e.,
  $\mathcal{D}\left(\rho\right)=U\rho U^{\dagger}$
  $\forall\rho\in\text{DM\ensuremath{\left(N\right)}}$
  and $U$ some element of the projective unitary group,
  $U\in\text{PU\ensuremath{\left(N\right)}}$.
\item $\mathcal{D}$ maps a set $\mathcal{A}$ of $N$ one-dimensional
  orthogonal projectors onto a set of $N$ one-dimensional orthogonal
  projectors as well as a totally rotated projector $\Op P_{TR}$
  (with respect to $\mathcal{A}$) onto a one-dimensional projector.
\item $\mathcal{D}$ is unital and leaves the spectrum of a complete
  and totally rotating set of density matrices invariant.
\end{enumerate}

We now explain how Theorem~1 can be used to prove the claim that
$J_{dist}$, Eq.~\eqref{eq:3st_app_f_dist},
attains its global minimum if and only if condition~\eqref{eq:3st_cond}
is fulfilled for the three states defined in Sec.~\ref{sec:3st_oct}. We
first discuss the role of $\Op\rho_3 = \frac{1}{N} \openone$. It is used
to check whether the evolution corresponds to a dynamical map in the
optimization subspace and whether it is unital. This
dynamical map is obtained by
projecting the action of the dynamical map, defined on the total
Hilbert space,
onto the optimization subspace.
The term in the functional~\eqref{eq:3st_app_f_dist} involving $\Op\rho_3 =
\frac{1}{N} \openone$ becomes minimal, and so does the total
functional, only if the identity in the optimization subspace is
mapped onto itself. Minimization of $J_{dist}$ thus ensures a unital
dynamical map on the subsystem such that Theorem~1 is applicable.

We now discuss the role of $\Op\rho_1$ and $\Op\rho_2$ which by construction
form a complete and totally rotating set of density matrices. The
functional~\eqref{eq:3st_app_f_dist}
becomes zero only if $\mathcal{D}(\Op\rho_1) = \Op O \Op\rho_1 \Op
O^\dagger$ and $\mathcal{D}(\Op\rho_2) = \Op O \Op\rho_2 \Op
O^\dagger$. This requires the actual evolution to be unitary. Unitary
evolution leaves the spectrum of a density matrix invariant. Due to
the equivalence relation $(1)\Longleftrightarrow (3)$ in Theorem~1,
preservation of
the spectrum of a complete and totally rotating set of density matrices, i.e.,
the two states $\Op\rho_1$ and $\Op\rho_2$, is sufficient to ensure
unitarity.
Furthermore, it was proven in Ref.~\cite{ReichKochPRA13} that the density
matrices $\Op\rho_{1}$ and $\Op\rho_{2}$ are
unitary differentiating, i.e., it is possible to distinguish any two
unitary evolutions by inspection of $\Op\rho_{1}(T)$ and $\Op\rho_{2}(T)$
only. In particular there is only one unitary dynamical map,
$\mathcal{D}(\Op\rho) = \Op U \Op\rho \Op U^\dagger$, which leads to
$\mathcal{D}(\Op\rho_{i}(0)) = \Op O \Op\rho_i \Op O^\dagger$ for both $
i=1,2$, namely the one induced by the target unitary $\Op O$. Therefore
the functional~\eqref{eq:3st_app_f_dist} becomes minimal if and only if
the target gate $\Op O$ is implemented.

To summarize, $J_{dist}$ is additively composed of three terms,
each corresponding to a distance measure between the desired result,
$\Op O \Op\rho_i \Op O$, and the actually implemented evolution,
$\mathcal{D}(\Op\rho_i)$. For the total functional to be minimal, the
evolutions of all three states have to match. This
is the case only if a unital dynamical map on the optimization
subspace is implemented \emph{and} if this is the unitary evolution
according to $\Op O$. More explicitly, the distance measure formed by the
density matrices $i=1,2$ is only meaningful provided the evolution
within the optimization subspace corresponds to a unital dynamical map.
However, this is ensured by the third density matrix. Consequently, the 
global minimum of the functional~\eqref{eq:3st_app_f_dist} will 
only be attained if this condition is fulfilled, too.

Note that the functional~\eqref{eq:3st_app_f_dist} weights
all three states equally. This is not a unique choice. In fact, all
crucial properties of the functional remain unchanged when scaling the
three terms with different positive factors, which has been done in the
main text for example when discussing the optimisation using three states
with weighting which significantly improved the performance of the
optimization.

\section{Proof}
\label{sec:3_states_proofs}

We utilize in the following the representation of operators by
$N\times N$ matrices and therefore omit the operator notation.
In order to prove Theorem 1, it is useful
to first show the validity of the following lemma.

\textbf{Lemma 1}:
Let $\mathcal{D}$ be a unital dynamical map, i.e.,
$\mathcal D$ is completely positive and maps identity onto itself,
acting on $N \times N$
density matrices. If and only if there exists a set of $N$
one-dimensional, orthogonal projectors that is mapped by $\mathcal{D}$
onto another set of $N$
one-dimensional orthogonal projectors,
there exists a complete set of density matrices whose spectrum is invariant
under $\mathcal{D}$.

\textbf{Proof of Lemma 1}: ($\Longrightarrow$ direction) We denote the set of
$N$ one-dimensional projectors $P_{i}$ by $\mathcal{P}$.
By assumption, we know that
\begin{equation*}
\forall i:\ \mathcal{D}\left(P_{i}\right)=\tilde{P}_{i} \,,
\end{equation*}
where the $\tilde{P}_{i}$ also form a set of $N$
one-dimensional, orthogonal projectors.
Clearly, $\text{spec}\left(P_{i}\right)=\text{spec}(\tilde{P}_{i})$,
hence $\forall P_i\in\mathcal{P}$
\begin{equation*}
\text{spec}\left(\mathcal{D}\left(P_i\right)\right)=
\text{spec}\left(P_i\right) = (1,0,\dots,0)\,.
\end{equation*}
Obviously, $\mathcal{P}$ itself corresponds to a specific complete set of
density matrices, $\rho_i=P_i$.

($\Longleftarrow$ direction)
This part of the proof proceeds as follows: First we show that the
assumption, a dynamical map leaving the spectrum of a given density
matrix invariant, implies that $\mathcal D$ maps projectors onto
the eigenspaces of the initial density matrices
into projectors onto the eigenspaces of the resulting density matrix
with the same eigenvalue. As a consequence, a one-dimensional projector
onto a corresponding one-dimensional eigenspace is mapped into a
one-dimensional projector.
We then repeat this argument for all density matrices in the complete
set. In this set, by definition, there exist density matrices with $N$
one-dimensional, orthogonal projectors onto one-dimensional
eigenspaces which, according to the first step of the $\Longleftarrow$
proof, is mapped onto another set of one-dimensional projectors. We
show in a second step
that the set of the mapped one-dimensional projectors is also orthogonal.

We start by assuming that $\mathcal{D}$ leaves the
spectrum of a given density matrix, $\rho$, invariant,
\begin{equation*}
\text{spec}\left(\mathcal{D}\left(\rho\right)\right)=
\text{spec}\sum_{k}\left(E_{k}\rho E_{k}^{\dagger}\right)=
\text{spec}\left(\rho\right) \,,
\end{equation*}
where we have expressed $\mathcal{D}$ in terms of Kraus operators
$E_k$. We can write $\rho=\sum_{i}\lambda_{i}P'_{i}$ where
$\mathcal{P}'=\left\{P'_{i}\right\}$ is a set of $M$ orthogonal
projectors onto the eigenspaces of $\rho$ with $M$ the number of
distinct eigenvalues of $\rho$.
We assume the $\lambda_{i}$ to be ordered by magnitude
with $\lambda_{1}$ corresponding to the largest eigenvalue.
Since we know that the spectrum of $\mathcal{D}\left(\rho\right)$
to be identical to that of $\rho$, we can
decompose $\mathcal{D}\left(\rho\right)$,
\begin{eqnarray*}
\mathcal{D}\left(\rho\right) & = & \sum_{i}\lambda_{i}\tilde{P}'_{i}
\end{eqnarray*}
with $\{ \tilde{P}'_{j}\} $ another set of $M$ orthogonal
projectors.
Note that neither the $P'_{i}$ nor the $\tilde{P}'_{i}$
have to be one-dimensional but
for a given $i$, $\tilde{P}'_{i}$ has the same dimensionality
as the corresponding $P'_{i}$. Specifically,
\begin{equation*}
\mathcal{D}\left(\rho\right)=
\mathcal{D}\left(\sum_{i}\lambda_{i}P'_{i}\right)=
\sum_{i}\lambda_{i}\mathcal{D}\left(P'_{i}\right)=
\sum_{j}\lambda_{j}\tilde{P}'_{j} \,.
\end{equation*}
Multiplying by another projector $\tilde{P}'_{p}$ from the set,
where $p$ can take integer values between $1$ and $M$,
%for arbitrary $p=1,\dots,k$
we obtain
\begin{equation}
\sum_{i}\lambda_{i}\mathcal{D}\left(P'_{i}\right)\tilde{P}'_{p}
=\sum_{j}\lambda_{j}\tilde{P}'_{j}\tilde{P}'_{p}
=\lambda_{p}\tilde{P}'_{p} \,, \label{eq:3st_app_startingequation}
\end{equation}
since $\tilde{P}'_{j}$, $\tilde{P}'_{p}$ are orthogonal.
Using proof by (transfinite) induction we now show that
\begin{equation*}
\mathcal{D}\left(P'_{k}\right)=\tilde{P}'_{k} \quad \forall i=k,\dots, M\,.
\end{equation*}
The idea of the induction is the following: To show that indeed the
projectors onto the eigenspaces of $\rho$, $P'_i$, are mapped into
projectors onto the eigenspaces of $\mathcal{D}(\rho)$ with the same
eigenvalue, we start with the projector onto the eigenspace
with the largest eigenvalue and then inductively proceed to
increasingly smaller eigenvalues. Furthermore, to prevent having to
deal with a possible smallest eigenvalue of $0$, we treat the
lowest eigenvalue case separately.
Calling the induction variable $k$,
we have to show that $\mathcal{D}\left(P'_{k}\right)=\tilde{P}'_{k}$
follows from the assumption
$\mathcal{D}\left(P'_{i}\right)=\tilde{P}'_{i}~\forall
i<k$.
Note that if $k=M$, i.e., for the smallest eigenvalue,
\begin{equation}
\label{eq:3st_app_unital}
\sum_{i}\mathcal{D}\left(P'_{i}\right)
=\mathcal{D}\left(\sum_i P'_{i}\right)
=\mathcal{D}\left(\openone\right)=\openone\,,
\end{equation}
since, by definition, a unital dynamical map maps identity
onto itself. So assume $k\neq M$. Then $\lambda_{k}>0$ since it
is not yet the smallest eigenvalue because each $\lambda{k}$ corresponds
by construction to a different eigenspace, hence they are different, and
we assumed them to be ordered.
For $k=p$, we can rewrite Eq.~\eqref{eq:3st_app_startingequation},
multiplying by an arbitrary normalized eigenvector
$\vec{x}_{k}\in\mathbb{C}^N$ of $\tilde{P}'_{k}$ from the left and right,
\begin{equation}
\label{eq:3st_app_lambdas}
\sum_{i}\lambda_{i}\vec{x}_{k}\cdot\mathcal{D}\left(P'_{i}\right)\cdot\vec{x}_{k}
=\lambda_{k} \,.
\end{equation}
By assumption of the induction,
$\mathcal{D}\left(P'_{i}\right)=\tilde{P}'_{i}~\forall i<k$, therefore
\begin{equation*}
  \vec{x}_{k}\cdot\mathcal{D}\left(P'_{i}\right)\cdot\vec{x}_{k}=0
  \quad \forall i<k\,.
\end{equation*}
Introducing $d_{kk}^{\left(i\right)}
\equiv \vec{x}_{k}\cdot\mathcal{D}\left(P'_{i}\right) \cdot\vec{x}_{k}$,
Eq.~\eqref{eq:3st_app_lambdas} can be written as
\begin{equation}
\label{eq:3st_app_secondequation}
\sum_{i\geq k}\lambda_{i}d_{kk}^{\left(i\right)}=\lambda_{k} \,.
\end{equation}
Due to Eq.~\eqref{eq:3st_app_unital} and the assumption of the induction,
\begin{eqnarray*}
\sum_{i}d_{kk}^{\left(i\right)}
=\sum_{i\geq k}d_{kk}^{\left(i\right)}
=\sum_{i}\vec{x}_{k}\cdot\mathcal{D}\left(P'_{i}\right)\cdot\vec{x}_{k}
= 1\,,
\end{eqnarray*}
and, since $\mathcal{D}\left(P'_{i}\right)$ is the image of a positive
semidefinite matrix which has to be positive semidefinite
itself,
\begin{eqnarray*}
d_{kk}^{\left(i\right)}=\vec{x}_{k}\cdot\mathcal{D}\left(P'_{i}\right)
\cdot\vec{x}_{k} \geq  0\ ~\forall i \,.
\end{eqnarray*}
Now remember that $\lambda_{k}\neq0$ is strictly larger than
all the other $\lambda_{i}$ with $i>k$  since the eigenvalues are assumed
to be ordered. In addition, $d_{kk}^{\left(i\right)}\geq0\ ~\forall i$
and at least one $d_{kk}^{\left(i\right)}$ with $i\ge k$ must be nonzero,
otherwise the $d_{kk}^{\left(i\right)}$ would not sum up to $1$. Then
\begin{equation*}
\sum_{i\geq k}\lambda_{i}d_{kk}^{\left(i\right)}
\leq \lambda_{k}\sum_{i\geq k}d_{kk}^{\left(i\right)}
=\lambda_{k} \,,
\end{equation*}
with equality if and only if $d_{kk}^{\left(i\right)}=0$ for $i\neq
k$. In fact, equality has to hold since otherwise we would contradict
Eq.~\eqref{eq:3st_app_secondequation}. We conclude that
\begin{equation*}
d_{kk}^{\left(i\right)}
=\vec{x}_{k}\cdot\mathcal{D}\left(P'_{i}\right)\cdot\vec{x}_{k}=\delta_{ik} \,.
\end{equation*}
Since $\vec{x}_{k}$ is normalized and arbitrary as long
as it lies in the eigenspace $\tilde{\mathcal E}_{k}$
of $\tilde{P}'_{k}$,
$\vec{x}_{k}$ must be an eigenvector of $\mathcal{D}\left(P'_{k}\right)$
with eigenvalue $1$. Consequently, the operator
$\mathcal{D}\left(P'_{k}\right)$ maps the eigenspace of
$\tilde{P}'_{k}$ onto itself. Now we are almost done
with showing that $\mathcal{D}\left(P'_{k}\right)$ and $\tilde{P}'_{k}$ are indeed
identical. Since $\tilde{\mathcal E}_{k}$ is mapped by
$\mathcal D$ into itself,
$\mathcal{D}\left(P'_{k}\right)$ has at least $\dim(\tilde{\mathcal E}_{k})$
eigenvalues equal to $1$. The fact that
$\mathcal{D}\left(P'_{k}\right)$ has exactly $\dim(\tilde{\mathcal E}_{k})$
eigenvalues equal to $1$ follows from $\mathcal{D}$ being
trace-preserving:
$\text{Tr}\left[\mathcal{D}\left(P'_{k}\right)\right]
=\text{Tr}\left[P'_{k}\right]=\dim (\mathcal E_{k})$
and $\dim (\mathcal E_{k})=\dim(\tilde{\mathcal E}_{k})$,
where $\text{Tr}\left[\mathcal{D}\left(P'_{k}\right)\right]$
is the sum over the eigenvalues of $\mathcal{D}\left(P'_{k}\right)$.
Since all eigenvalues of $\mathcal{D}\left(P'_{k}\right)$
are non-negative, all other eigenvalues must vanish.
Hence $\mathcal{D}\left(P'_{k}\right)=\tilde{P}'_{k}$.
This completes the induction and concludes the first step of the
$\Longleftarrow$ proof, i.e., we have shown that a unital dynamical
map that leaves the spectrum of a given arbitrary density matrix
invariant, maps projectors onto the eigenspaces of this density
matrix onto projectors of the same rank. This is specifically true
for one-dimensional projectors.
Iterating the argument for all density matrices in the complete set
and selecting a set $\mathcal{P}$ of $N$ orthogonal, one-dimensional
projectors,  it follows that these projectors will be mapped by
$\mathcal{D}$ onto another set of one-dimensional projectors.

In the second step of the $\Longleftarrow$ proof we still need to show
that the mapped set is also orthogonal. We denote the
complete set of projectors by $\{P_{i}\}$. From the first step of the
$\Longleftarrow$ proof we know that the $\tilde{P}_{i}$,
\begin{equation*}
\mathcal{D}\left(P_{i}\right)=\tilde{P}_{i} \,,
\end{equation*}
need to be one-dimensional projectors.
Using the unitality of $\mathcal{D}$, we see that
\begin{equation*}
\openone=\mathcal{D}\left(\openone\right)=\mathcal{D}\left(\sum_{i}P_{i}\right)=
\sum_{i}\mathcal{D}\left(P_{i}\right)=\sum_{i}\tilde{P}_{i} \,.
\end{equation*}
The unit matrix can only be summed by $N$ one-dimensional projectors
if these are orthogonal. Hence we have accomplished the second step, and
the lemma follows.

\textbf{Proof of Theorem 1}:
The equivalence relation of statement $(1) \Longleftrightarrow (2)$ has already
been proven in Ref.~\cite{ReichKochPRA13}. To complete
the proof of the more general Theorem~1, we are left with proving
$(2) \Longleftrightarrow (3)$.

$\left(2\right)\Longrightarrow\left(3\right)$:
If $\mathcal D$ maps a set of $N$ one-dimensional orthogonal
projectors onto another set of $N$ one-dimensional orthogonal
projectors, it leaves the spectrum of the projectors invariant.
This can be seen as follows. Projectors are idempotent and positive
semi-definite, hence their spectrum can only consist of zeros and
ones. Since the projector is one-dimensional,
its image under $\mathcal{D}$ has to be one-dimensional, too, and
there can only be one eigenvalue equal to one. Thus any
one-dimensional projector has the spectrum $\{1,0,0,\dots\}$ which
must be invariant under a mapping between one-dimensional orthogonal
projectors.
We now use the linearity of dynamical maps to show
that $\mathcal D$ must be unital.
Specifically, let $\{P_i\}$ be the initial set of orthogonal projectors
that is mapped to another set of orthogonal projectors, $\{\bar{P}_i\}$.
We find for the image of the totally mixed state,
$\rho_M = \frac{1}{N} \openone$,
\begin{eqnarray*}
  \mathcal{D}(\rho_M) & = &
  \mathcal{D}\left(\frac{1}{N} \sum_{i=1}^N P_i\right) =
  \frac{1}{N} \sum_{i=1}^N \mathcal{D}(P_i) \\
  & = & \frac{1}{N} \sum_{i=1}^N \bar{P}_i = \rho_M \,,
\end{eqnarray*}
i.e., $\mathcal D$ maps identity onto itself, making it unital. We can
thus use Lemma~1 to obtain that the
spectrum of a complete set of density matrices is invariant under
$\mathcal D$.
We now just have to add $\rho_{TR}=P_{TR}$ to realize a complete and
totally rotating set. The spectrum of $\rho_{TR}=P_{TR}$ is also
invariant under $\mathcal D$ since it is a one-dimensional projector
that is mapped onto another one-dimensional projector.

$\left(3\right)\Longrightarrow\left(2\right)$:
From Lemma~1, we obtain  that $\mathcal{D}$ maps a set of $N$
one-dimensional, orthogonal projectors onto another set of $N$
one-dimensional orthogonal projectors. We are thus only left with
showing that $\mathcal{D}$ maps a totally rotated projector onto a
one-dimensional projector: There always exists a density
matrix with a one-dimensional eigenspace corresponding to a totally
rotated projector $P_{TR}$ whose spectrum is invariant under the action
of $\mathcal{D}$. In the proof of Lemma~1, we have shown that
a dynamical map that leaves the spectrum of projectors invariant
maps these projectors onto projectors of the same rank.
Repeating the steps of the proof of Lemma~1, we see that the image of
$P_{TR}$ has to be a one-dimensional projector. This completes the
proof of Theorem~1.
