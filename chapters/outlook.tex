% vim: ft=tex iskeyword=@,48-57,_,-,192-255,\: dictionary=bibkeys.lst,labels.lst:
\chapter{Summary and Outlook}
\label{chap:outlook}

\section{Summary and Conclusions}

The material presented in this thesis provides a comprehensive framework for the
realization of robust two-qubit quantum gates using numerical optimal control.
It covers three aspects:
\begin{enumerate}[noitemsep]
  \item the use of efficient numerical tools for the modeling, simulation, and
        control of both closed and open quantum systems,
  \item the facilitation of robustness with respect to decoherence and classical
        fluctuations of system parameters through advanced control techniques,
        and
  \item the application of these techniques to two of the leading candidates for
        the implementation of universal quantum computers, trapped atoms and
        superconducting circuits.
\end{enumerate}

The numerical approach to the design of quantum gates allows to go beyond
parameter regimes where the system can accurately be described by simple,
analytically solvable models. The flip side of this promise is that the design
of \emph{efficient} representations, algorithms, and implementations becomes
crucial. Some of the fundamental techniques have been presented in
chapter~\ref{chap:numerics}.
An important part of the work in this thesis has consisted of contributing to
their implementation in the QDYN Fortran 90 library.
The core aspect of the efficient storage and
application of Hamiltonians, respectively Liouvillians is to exploit their
sparsity. Especially for systems with spatial degrees of freedom, obtaining
sparse operators may require the use of spectral and pseudo-spectral
representations. For the propagation of the time-dependent Schrödinger equation,
respectively the Liouville von-Neumann equation for open quantum systems, the
most efficient method is the expansion of the time evolution operator,
respectively the dynamical map in a fast-converging polynomial series. For an
Hermitian operator, the fastest converging expansion is in Chebychev
polynomials. For a non-Hermitian operator (especially a Liouvillian), an
expansion in Newton polynomials must be used instead. A memory efficient
variation of this method involves the use of restarted Arnoldi iterations. Both
the Chebychev propagator and the Newton propagator with restarted Arnoldi are
presented in pseudocode in appendix~\ref{AppendixAlgos}. For optimal control,
algorithms can be divided in gradient-free optimization methods that are
applicable to low-dimension control problems, and gradient-based methods that
provide significantly faster convergence, but require additional numerical
effort to evaluate the gradient, and place restrictions on the types of
functionals that can be optimized. The GRAPE/LBFGS algorithm is suitable for
discrete parametrizations of the control, whereas Krotov's method
guarantees monotonic convergence for quasi-time-continuous controls.
These methods, and Krotov's method in particular, form the basis of the
results presented in this thesis.

The realization of robust quantum gates can be achieved by using advanced
control techniques. A first example of such a technique is the optimization over
an ensemble of Hamiltonians to address the issue of classical fluctuations and
uncertainties in the system and/or the control. In chapter~\ref{chap:robust},
this has been used to obtain quantum gates for trapped Rydberg atoms that are
robust with respect to  variations of the amplitude of the control field, and
fluctuations in the Rydberg levels caused by stray electromagnetic field. With
respect to the best available analytical schemes, robustness has been increased
by more than one order of magnitude. Moreover, this level of robustness can be
maintained as pulses are shortened to approach the quantum speed limit, beyond
what is achievable using analytic pulses. Lastly, the optimized pulses
successfully avoided the effects of spontaneous decay from an intermediary level
for the transition to the Rydberg state.

Superconducting qubits provide one of the most versatile platforms for quantum
computing. They may be engineered in an almost arbitrary range of parameters,
using widely available production techniques. Recent implementations using
the transmon design are approaching decoherence times close to \SI{1}{ms}.
This makes transmon qubits one of the most promising contenders for quantum
computing. Yet, the implementation of high fidelity entangling gates reaching
the quantum error correction limit has not been achieved. In
chapter~\ref{chap:transmon}, the fundamental concepts and recent advances in the
realization of two-qubit gates for transmon qubits are reviewed. We show that
a holonomic phasegate can be implemented by driving the system with an
off-resonant drive near the cavity transition which induces a Stark shift on the
levels of the logical subspace. Using Krotov's method, we obtain optimized
pulses that successfully implement a CPHASE and a CNOT gate, as well as the
holonomic phasegate. However, fidelities remain limited by the loss of
population from the logical subspace. Addressing this loss of population in
order to achieve high fidelity gates below the quantum error correction limit,
and taking into account the dominant sources of decoherence
will be the focus of future work.

Using an effective model described in chapter~\ref{chap:transmon} for two
transmon qubits, a further advanced control technique that aids the realization
of robust gates has been illustrated in chapter~\ref{chap:pe}:
Based on the fundamental insight that any perfect entangler together with single
qubit operations is sufficient for universal quantum computing, the application
of a functional that optimizes for an arbitrary perfect entangler has been
demonstrated.  Avoiding the use of an overly specific optimization functional,
such as the optimization for a ``standard'' quantum gate such as CNOT or iSWAP,
allows the optimal control method to find the entangling gate that is easiest to
realize.  This becomes especially relevant as constraints are added to the
optimization.  The optimization functional is based on the geometric theory of
two-qubit gates, reviewed in section~\ref{sec:C_LI} of
chapter~\ref{chap:quantum}. Any two-qubit gate can be mapped to a point in the
Weyl chamber. Gates that map to the same point are identical up to single-qubit
operations. The set of perfect entanglers form a polyhedron inside the Weyl
chamber.  In earlier work~\cite{ReichDipl10, MullerPRA11}, the optimization
towards a specific point in the Weyl chamber has been demonstrated, allowing the
optimization towards a gate up to arbitrary single-qubit operations. The results
of chapter~\ref{chap:pe} have demonstrated that an optimization towards the
surface of the polyhedron of perfect entanglers is significantly faster and
reaches better fidelities than the optimization towards a specific point in the
Weyl chamber, and thus a specific two-qubit gate.  There are two aspects under
which the optimization towards an arbitrary perfect entangler benefits the
implementation of gates that are robust with respect to dissipation. First, the
optimization can find the perfect entangler that can be realized in the shortest
amount of time, ideally shorter than the relevant dissipation processes. Second,
the presence of decoherence places and effective constraint on the system. For
example, dissipation rates increase with higher excitation numbers. Thus, in
order to minimize the effects of decoherence, it is beneficial to implement
a quantum gate with the least excitation. Optimizing for a general perfect
entangler has the potential to automatically identify the quantum gate that is
least affected by decoherence.  This remains to be demonstrated in future work.

In order to allow the optimization to avoid the effects of decoherence, the
dissipation must be explicitly included in the equations of motion. Modeling the
dynamics in Liouville space greatly exacerbates the numerical challenges.
Instead of Hilbert space vectors of dimension $d$, every state is now a density
matrix of dimension $d^2$. This both significantly increases the required
storage for storing all propagated states, and the CPU time required in
propagation, as all matrix-vector operations in Hilbert space become
matrix-matrix operations in Liouville space. The optimization of a two-qubit
gate in Hilbert space requires the propagation of the four logical basis states
$\{\Ket{00}, \Ket{01}, \Ket{10}, \Ket{11}\}$. A naive extension of the standard
functionals to Liouville space would require the propagation of the 16 matrices
that form the basis of the two-qubit Liouville logical subspace.

Chapter~\ref{chap:3states} has demonstrated that the propagation of the full
Liouville space basis is not necessary for the optimization of a unitary gate,
but that a reduced set of states can be employed. In general, the minimum number
of states that need to be propagated is 3. In situations where the Hamiltonian
only allows for a subset of two-qubit gates to be implemented, this number may
reduce further. For example, the Hamiltonian for two trapped Rydberg atoms,
chapter~\ref{chap:robust}, only allows for diagonal gates. We have demonstrated
that in this case, propagation of only two density matrices is sufficient to
optimize for a CPHASE gate. For the general case, we have considered again the
effective model for two coupled transmon qubits. There, optimization has been
shown to be successful using three states. However, in order to achieve an
efficiency comparable to the propagation of the full basis, the three states
must be properly weighted. We have shown that convergence can be further improved
using a set of 5, respectively 8 states. In all cases, considerable savings both
in memory and CPU time have been demonstrated, addressing the issue of numerical
efficiency of the optimization of quantum gates in Liouville space.


\section{Future Work}

In order to achieve the ultimate goal of implementing a universal set of quantum
gates for superconducting qubits at high fidelity with robustness against
decoherence, several issues must yet be addressed.
In a full model, Eq.~\eqref{eq:tm_fullham}, fidelities have been limited by the
loss of population from the logical subspace caused by high cavity excitation in
particular (chapter~\ref{chap:transmon}). These excitations necessitate a large
Hilbert space dimension $d>3000$, making optimization in the corresponding
Liouville space (dimension $d^2$) numerically infeasible. Thus, no pulses
optimized with respect to dissipation have been obtained.

There are several approaches to improving the fidelities in this case.
First, the gradient employed in Krotov's method has been demonstrated to
become small when close to the optimum~\cite{MachnesPRA2011}, yielding slow
asymptotic convergence. An optimization algorithm using second order
information, such as LBFGS, can be more effective in this region, allowing to
return population fully to the logical subspace. However, an even better
approach is to avoid large excitations in the first place.
A state-dependent constraint~\cite{PalaoPRA08} can be added to the optimization
functional in order to suppress high cavity population. Alternatively,
frequencies driving unwanted transitions can be avoided in the optimized pulses
using spectral constraints~\cite{JosePRA13, ReichJMO2014}. To avoid high
excitation of the qubit levels, the DRAG technique~\cite{MotzoiPRL2009,
MotzoiPRL2009} could be used, adding a second pulse that suppresses excitation
out of the logical subspace.

Eliminating large excitations allows to truncate the Hilbert space at
significantly smaller dimension, making it feasible to describe the
system in Liouville space and to take dissipative effects into account in the
optimization explicitly. Decay and dephasing rates for higher levels of the
transmon qubit have recently been obtained~\cite{PetererPRL2015}, allowing to
model the dynamics as a master equation in Lindblad form.
We can apply full range of techniques presented in this thesis to obtain
maximally robust gates. Optimization for a general perfect entangler
(chapter~\ref{chap:pe}) allows to identify the gate that can be implemented in
the shortest amount of time in order to beat decoherence. Numerical effort is
kept minimal by propagating only a reduced set of states
(chapter~\ref{chap:3states}).  Also, one could consider using ensemble
optimization (chapter~\ref{chap:robust}) to obtain further robustness against
fluctuations of the system parameters, e.g.\ the frequency of the transmon
qubits.

An alternative possibility to reduce the dimension of the Hilbert space is to
employ an effective model that eliminates the cavity from the description of the
system. For a simplified effective model, Eq.~\eqref{eq:tm_jc_eff}, gates of
arbitrary fidelity in the non-dissipative case, and gate fidelities limited only
by the decoherence in the general case have been attained. However, we have
found the pulses obtained from this model not to be transferable to the full
model of Eq.~\eqref{eq:tm_fullham}. Even for the properly first-order effective
Hamiltonian of Eq.~\eqref{eq:H_red_dispersive}, agreement between the full and
effective model is limited to small pulse amplitudes~\cite{GambettaPRA2006}.
In order to to obtain optimized pulses that are exchangeable between the
effective and the full model, at high fidelity, a rigorous derivation to higher
order perturbation theory, using a computer algebra system, is required. In this
way, full correspondence between the parameters of the effective and the full
model can be obtained, and the validity of any approximations made in the
derivation ensured. The transformation yielding the effective Hamiltonian can
also be applied to the full master equation~\cite{GambettaPRA2008,
BoissonneaultPRA2012}. Decay of the cavity leads to an additional effective
dephasing of the qubit in this case.

Specifically for the holonomic gate considered in chapter~\ref{chap:transmon},
a drive that is only slightly detuned from the cavity induces a significant
Stark shift in the cavity levels. In the non-dressed basis, this is reflected in
a large cavity excitation, see Fig~\ref{fig:adiabatic_popdyn}. Therefore, it is
not possible to simply suppress cavity population with a state-dependent
constraint. However, in principle, and effective model can be derived through
a series of transformations~\cite{FelixHolonomic}. Most importantly, these
include a Polaron transformation
\index{Polaron transformation}%
\begin{equation}
  D(\alpha) = \exp\left( \alpha \Op{a}^{\dagger} - \alpha^{*} \Op{a} \right)
\end{equation}
that describes the displacement of the cavity by $\alpha$ induced by the
field. With a properly chosen value for $\alpha$ depending on field $\Omega(t)$
the driving term for the cavity is eliminated from the Hamiltonian in the
shifted frame~\cite{BlaisPRA2007}, allowing to truncate the Hilbert space to
a smaller dimension.

One of the fundamental promises of numerical optimal control is the ability to
go beyond the regime of simple analytically solvable models. Thus, it would be
possible to explore parameter regimes beyond those of the standard
Jaynes-Cummings model for the qubit-cavity interaction,
Eq.~\eqref{eq:tm_fullham}. For example, in the ultra-strong coupling regime,
Eq.~\eqref{eq:tm_jc_condition} would be violated, and the interaction
Hamiltonian would take the form of Eq.~\eqref{eq:qubit_cavity_interaction}.
A master equation for superconducting qubits in the ultrastrong coupling regime
has been derived in Ref.~\cite{BeaudoinPRA2011}.

While with the approaches discussed above we can fully expect to obtain high
fidelity quantum gates for transmon qubits, including qubit decay and dephasing
as well as cavity decay will place limits on the achievable fidelity,
cf.~chapter~\ref{chap:3states}. In this case, we are limited to using optimal
control to implement quantum gates on a shorter time scale than the decoherence.
Furthermore, we can minimize -- but not eliminate -- the effects of decoherence,
e.g.\ by avoiding cavity population. Microscopically, decoherence in
superconducting qubits has been linked to dielectric defects as a primary
source~\cite{MartinisPRL2005,ShaliboPRL2010} that may be modeled as two-level
systems. While weak coupling leads phenomenologically to decay and dephasing and
a master equation of Lindblad form, for strong coupling the dynamics can be
non-Markovian~\cite{KroviPRA2007,FerraroPRB2008,LorenzoPRA2013}. That is, the
conditions listed in section~\ref{subsec:markov_microscopic} of
chapter~\ref{chap:quantum} are violated and the dynamics can no longer be
treated with a master equation in Lindblad form. Remarkably, while in the
Markovian regime, information that passes from the system to the environment
through dissipation is irretrievably lost, non-Markovian dynamics allow for
a backflow of information from the environment to the
system~\cite{BreuerPRL2009,BreuerJPB2012}. This opens up new possibilities for
control, and may in fact aid in the implementation of quantum gates. For
example, it has recently been shown that if the qubit
strongly couples to just a few of the bath spins, which in turn couple weakly to
the rest of the environment, the non-Markovianity can be exploited for the
realization of quantum gates~\cite{ReichNonMarkov}.

Numerical optimal control has previously been applied to non-Markovian quantum
systems~\cite{RebentrostPRL09,AsplundPRL11,SchmidtPRL11,FloetherNJP12},
provided the dynamics can be calculated with sufficient efficiency.
One possibility to tackle the simulation of the system's time evolution is by
a renormalization approach~\cite{GualdiPRA2013}.
It is important to note that none of the techniques presented in this thesis are
limited to a specific, e.g.\ Markovian, dissipation model or a specific equation
of motion. They may thus be used to explore a wider range of physics that may
lead to entirely new possibilities of quantum control.

