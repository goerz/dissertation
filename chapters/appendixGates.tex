\chapter{Overview of Two-Qubit Gates}
\label{AppendixGates}

  The CNOT gate is the archetypal two-qubit gate in sets of operations for
  universal quantum computing; it flips the \emph{target} qubit if
  the \emph{control} qubit is in state \ket{1}.
  \begin{equation}
  \text{CNOT} =
    \begin{pmatrix}
    1 & 0 & 0 & 0 \\
    0 & 1 & 0 & 0 \\
    0 & 0 & 0 & 1 \\
    0 & 0 & 1 & 0
    \end{pmatrix}
  \end{equation}
  \index{CNOT gate}

  The CPHASE gate induces a phase shift of $\gamma$ on the target qubit if the
  control qubit is in state \ket{1}.
  \begin{equation}
  \text{CPHASE}_{\gamma} =
    \begin{pmatrix}
    1 & 0 & 0 & 0 \\
    0 & 1 & 0 & 0 \\
    0 & 0 & 1 & 0 \\
    0 & 0 & 0 & \ee^{\ii \gamma}
    \end{pmatrix}
  \end{equation}
  \index{CPHASE gate}
  The gate is a perfect entangler for $\gamma=\pi$, where it is locally
  equivalent to CNOT. Indeed, \emph{all} controlled operators are locally
  equivalent to a $\text{CPHASE}_{\gamma}$ \cite{ZhangPRA03}. We refer to
  $\text{CPHASE}_{\pi}$ simply as CPHASE. In the Weyl chamber, the
  $\text{CPHASE}_{\gamma}$ gates are on the line $O$--$A_1$.

  The SWAP gate exchanges the two qubits. The gates at the $A_3$ point in the
  Weyl chamber are the only true two-qubit gates that yield zero entanglement.
  \begin{equation}
  \text{SWAP} =
    \begin{pmatrix}
    1 & 0 & 0 & 0 \\
    0 & 0 & 1 & 0 \\
    0 & 1 & 0 & 0 \\
    0 & 0 & 0 & 1 \\
    \end{pmatrix}
  \end{equation}
  \index{SWAP gate}

  The $\sqrt{\text{SWAP}}$, located at the point $P$ in the Weyl chamber,
  however, is a perfect entangler, indicating that
  a SWAP gate is implemented by first entangling and then disentangling the two
  qubits.
  \begin{equation}
  \sqrt{\text{SWAP}}=
    \begin{pmatrix}
    1 & 0 & 0 & 0 \\
    0 & \frac{1}{2}-\frac{\ii}{2} & \frac{1}{2}+\frac{\ii}{2} & 0 \\
    0 & \frac{1}{2}+\frac{\ii}{2} & \frac{1}{2}-\frac{\ii}{2} & 0 \\
    0 & 0 & 0 & 1
    \end{pmatrix}
    \label{eq:sqrt_SWAP}
  \end{equation}
  A secondary square root of SWAP is located at the $N$ point; it is simply
  the complex conjugate of the principal square root, Eq.~\eqref{eq:sqrt_SWAP},
  and we thus label is as $\sqrt{\text{SWAP}}^*$.
  %\begin{equation}
  %\sqrt{\text{SWAP}}^* =
    %\begin{pmatrix}
    %1 & 0 & 0 & 0 \\
    %0 & \frac{1}{2}+\frac{\ii}{2} & \frac{1}{2}-\frac{\ii}{2} & 0 \\
    %0 & \frac{1}{2}-\frac{\ii}{2} & \frac{1}{2}+\frac{\ii}{2} & 0 \\
    %0 & 0 & 0 & 1
    %\end{pmatrix}
  %\end{equation}

  The iSWAP gate performs a SWAP, with and additional relative phase shift
  of $\pi$. The gate is also known  as DCNOT (Double-CNOT), since is
  implemented by two consecutive CNOT gates, where the control qubit for the
  second CNOT is the target qubit of the first CNOT.
  \begin{equation}
  \text{iSWAP} =  \text{DCNOT} =
    \begin{pmatrix}
    1 & 0 & 0 & 0 \\
    0 & 0 & \ii & 0 \\
    0 & \ii & 0 & 0 \\
    0 & 0 & 0 & 1 \\
    \end{pmatrix}
  \end{equation}
  \index{iSWAP gate}
  \index{DCNOT gate}

  The principal square root of iSWAP is located at the $Q$ point in the Weyl
  chamber.
  \begin{equation}
  \sqrt{\text{iSWAP}} =
    \begin{pmatrix}
    1 & 0 & 0 & 0 \\
    0 & \frac{1}{\sqrt{2}} & \frac{\ii}{\sqrt{2}} & 0 \\
    0 & \frac{\ii}{\sqrt{2}} & \frac{1}{\sqrt{2}} & 0 \\
    0 & 0 & 0 & 1 \\
    \end{pmatrix}
  \end{equation}
  Note that none of the gates at the $M$ point are square roots of the exact
  iSWAP, even though their square is still locally equivalent to iSWAP.


  The B-GATE is at the center of the perfect entanglers. It has been shown to be
  extremely efficient for the construction of arbitrary two-qubit gates
  \cite{ZhangPRL2004}.
  \begin{equation}
  \text{B-GATE} =
    \begin{pmatrix}
    \cos\frac{\pi}{8} & 0 & 0  & \ii \sin\frac{\pi}{8} \\
    0 & \cos\frac{3\pi}{8} & \ii \sin\frac{3\pi}{8} & 0 \\
    0 & \ii \sin\frac{3\pi}{8} & \cos\frac{3\pi}{8} & 0 \\
    \ii \sin\frac{\pi}{8} & 0 & 0 & \cos\frac{\pi}{8}
    \end{pmatrix}
    \index{B-GATE}
  \end{equation}

\begin{table}
\centering

%\vspace{-20pt}
\centering\includegraphics{weylchamber}
\vspace{10pt}

{\small
\begin{tabularx}{\textwidth}{m{18mm}|X|lll|l|lll}
\toprule
Gate & Hamiltonian & $c_1$ & $c_2$ & $c_3$ & $W$  & $g_1$ & $g_2$ & $g_3$ \\
\midrule
& & & & & & & & \\
%
%
$\unity$ &
(single qubit gates)&
$0$ & $0$ & $0$ &
$O$&
$1$ & $0$ & $3$
\\
%
%
&
&
$\pi$ & $0$ & $0$ &
$A_1$&
$1$ & $0$ & $3$
\\[5mm]
%
%
CNOT &
$\SigmaZ^{(1)} + \SigmaX^{(2)} - \SigmaZ\SigmaX$ &
$\frac{\pi}{2}$ & $0$ & $0$ &
$L$&
$0$ & $0$ & $1$
\\[5mm]
%
%
CPHASE$_\gamma$ &
$\SigmaZ^{(1)} + \SigmaZ^{(2)} - \SigmaZ\SigmaZ$ &
$\frac{\gamma}{2}$ & $0$ & $0$ &
%$O$--$A_1$ &
&
$0 $ & $0 $ & \hspace*{-5mm}$-\!\cos\gamma $
\\[5mm]
%
%
iSWAP \newline DCNOT &
\vspace*{-17pt}
$\SigmaX\SigmaX + \SigmaY\SigmaY$ \newline
$= \frac{1}{2} \left(\SigmaPlus\SigmaMinus
  + \SigmaMinus\SigmaPlus \right)$ &
$\frac{\pi}{2}$ & $\frac{\pi}{2}$ & $0$ &
$A_2$ &
$0 $ & $0 $ & $-1 $
\\[5mm]
%
%
$\sqrt{\text{iSWAP}}$ &
$\SigmaX\SigmaX + \SigmaY\SigmaY $ &
$\frac{\pi}{4} $ & $\frac{\pi}{4} $ & $0 $ &
$Q$ &
$\frac{1}{4}$ & $0$ & $1$
\\[5mm]
%
%
SWAP&
$\SigmaX\SigmaX +\SigmaY\SigmaY +\SigmaZ\SigmaZ$ &
$\frac{\pi}{2}$ & $\frac{\pi}{2}$ & $\frac{\pi}{2}$ &
$A_3$ &
$-1 $ & $0 $ & $-3 $
\\[5mm]
%
%
$\sqrt{\text{SWAP}}$&
$\SigmaX\SigmaX +\SigmaY\SigmaY +\SigmaZ\SigmaZ$ &
$\frac{\pi}{4}$ & $\frac{\pi}{4}$ & $\frac{\pi}{4}$ &
$P$ &
$0 $ & $\frac{1}{4} $ & $0 $
\\[5mm]
%
%
$\sqrt{\text{SWAP}}^*$&
$-\SigmaX\SigmaX \!-\SigmaY\SigmaY \!-\SigmaZ\SigmaZ$ &
$\frac{3\pi}{4}$ & $\frac{\pi}{4} $ & $\frac{\pi}{4}$ &
$N$ &
$0$ & $-\frac{1}{4}$ & $0$
\\[5mm]
%
%
B-GATE&
$2\SigmaX \SigmaX + \SigmaY \SigmaY$ &
$\frac{\pi}{2}$ & $\frac{\pi}{4}$ & $0$ &
$B$ &
$0 $ & $0 $ & $0 $
\\[5mm]
%
%
M-GATE &
$3\SigmaX \SigmaX + \SigmaY \SigmaY$ &
$\frac{3\pi}{4} $ & $\frac{\pi}{4}$ & $0$ &
$M$ &
$\frac{1}{4}$ & $0 $ & $1$
\\[5mm]
%%
%%
\bottomrule
\end{tabularx}
}
\caption{Summary of two-qubit gates at special points in the Weyl chamber (shown
at the top, with the polyhedron of perfect entanglers indicated by the shaded
area).  For each gate, the Hamiltonian generating that gate up to a global
phase is given in terms of the Pauli matrices, where $\Op{\sigma}_{i}^{(1,2)}$
indicates an operator acting only on the first and second qubit, respectively,
and $\Op{\sigma}_i\Op{\sigma}_j$ is a shorthand for $\Op{\sigma}_i^{(1)} \otimes
\Op{\sigma}_{j}^{(2)}$.
Also, the Weyl coordinates $c_1$, $c_2$, $c_3$, the name of the respective point
in the Weyl chamber, and the local invariants $g_1$, $g_2$, $g_3$ are listed.}
\label{tab:appendixGates}
\end{table}

