% vim: ft=tex iskeyword=@,48-57,_,-,192-255,\: dictionary=bibkeys.lst,labels.lst:
\chapter[Rabi-Cycling in the Two-Level System]{Rabi-Cycling in the \\Two-Level System}
\label{AppendixRabi}


We consider a two-level system
\begin{equation}
  \Ket{\Psi(t)} = a(t) \Ket{0} + b(t) \Ket{1}
\end{equation}
with time-dependent complex coefficients $a(t)$ and $b(t)$. Under the rotating
wave approximation presented in appendix~\ref{AppendixRWA}, the Hamiltonian
takes the form
\begin{equation}
  \Op{H} = \begin{pmatrix}
    0                      & \frac{1}{2}\Omega_0(t) \\
    \frac{1}{2}\Omega_0(t) & \Delta
  \end{pmatrix}\,.
\end{equation}
where $\Omega_0(t)$ is a slowly varying pulse shape and $\Delta$ is the
detuning of the central pulse frequency from the $\Ket{0} \rightarrow \Ket{1}$
transition. We first consider a pulse that is a simple continuous oscillation
i.e.\ a constant $\Omega_0(t) \equiv \Omega_0$ in the RWA.

For the initial conditions $a(0) = 1$ and $b(0) = 0$, the Schrödinger equation
then has the solution~\cite{TannorBook}
\begin{subequations}
\label{eq:rabi_amplitudes}
\begin{align}
 a(t) &= \ee^{-\frac{\ii}{2} \Delta t} \left(
              \cos\left( \frac{\Omega t}{2} \right)
            - \ii \frac{\Delta}{\Omega} \sin\left( \frac{\Omega t}{2} \right)
         \right)\,,
 \\
 b(t) &= -\ii \frac{\Omega_0}{\Omega} \ee^{-\frac{\ii}{2} \Delta t}
            \sin\left( \frac{\Omega t}{2} \right) \,,
\end{align}
\end{subequations}
with
\begin{equation}
  \Omega = \sqrt{\Delta^2 + \Omega_0^2}\,.
\end{equation}
The population undergoes \emph{Rabi oscillations} between $\Ket{0}$ and
\index{Rabi oscillation}
$\Ket{1}$ with a period of $\frac{2\pi}{\Omega}$,
\begin{subequations}
\label{eq:rabi_populations}
\begin{align}
  \Abs{a(t)}^2 &= \left( \frac{\Delta}{\Omega} \right)^2
                  + \left( \frac{\Omega_0}{\Omega} \right)^2
                    \cos^2\left(\frac{\Omega t}{2} \right)\,,
  \\
  \Abs{b(t)}^2 &= \left( \frac{\Omega_0}{\Omega} \right)^2
                    \sin^2\left(\frac{\Omega t}{2} \right)\,.
\end{align}
\end{subequations}
In the case of non-zero detuning, $\Delta \neq 0$, the population is only
transferred partially, but at higher frequency.
In the resonant case, $\Delta = 0$, the Rabi-frequency is $\Omega = \Omega_0$,
and there is complete population transfer at $t = \frac{\pi}{\Omega}$. The
complete coherent transfer of population from $\Ket{0}$ to $\Ket{1}$ is
therefore called a ``$\pi$-pulse''. However, according to
Eq.~\eqref{eq:rabi_amplitudes}, it also induces a phase factor of
$\ee^{\ii\frac{\pi}{2}}$,
\begin{equation*}
\text{$\pi$-pulse:}\qquad \Ket{0} \rightarrow \ii \Ket{1}\,.
\end{equation*}
Transferring the population up and down again in a full Rabi cycle, or
$2\pi$-pulse, yields a phase factor $-1$,
\begin{equation*}
\text{$2\pi$-pulse:}\qquad \Ket{0} \rightarrow -\Ket{0}\,.
\end{equation*}
In order to restore the original state with a phase of zero, two full cycles are
necessary.

For a time-dependent but slowly varying pulse shape $\Omega(t)$, the Rabi angle
$\Omega t$ in the argument of the sines and cosines in
Eq.~\eqref{eq:rabi_amplitudes} and Eq.~\eqref{eq:rabi_populations} generalizes
to
\begin{equation*}
  \Omega t \rightarrow \int_{0}^{t} \Omega(t') \dd t'\,.
\end{equation*}
Thus, the amount of population that is transferred depends only on the pulse
area, not the specific shape of the pulse.



