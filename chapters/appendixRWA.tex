\chapter{The Rotating Wave Approximation}
\label{AppendixRWA}

In many cases, we can simplify the Hamiltonian and make it analytically
tractable by transforming it from the \emph{lab frame}
\index{lab frame}
to the \emph{rotating frame}
\index{rotating frame}
oscillating at the laser frequency $\omega$. The rotating frame is defined by
a time-dependent unitary transformation $\Op{U}(t)$. Every state $\Ket{\Psi}$ in
the lab frame is transformed to the corresponding state in the rotating frame as
\begin{equation}
  \Ket{\tilde{\Psi}(t)} = \Op{U}(t) \Ket{\Psi(t)}
\end{equation}
In order to derive the Hamiltonian in the rotating frame, we demand the
Schrödinger equation to be fulfilled
\begin{equation}
  i \hbar \partdifquo{t} \Ket{\tilde{\Psi}} = \tildeOp{H}
  \Ket{\tilde{\Psi}}\,,
\end{equation}
which leads to
\begin{equation}
\begin{split}
  i \hbar \partdifquo{t} \Ket{\tilde{\Psi}}
    & = i \hbar \dotOp{U} \ket{\Psi}
        + \op{U} i \hbar \partdifquo{t} \ket{\Psi} \\
    & = i \hbar \dotOp{U} \ket{\Psi}
        + \op{U}\op{H}\ket{\Psi} \\
    & = i \hbar \dotOp{U} \op{U}\daggered \Ket{\tilde{\Psi}}
           + \op{U}\op{H}\op{U}\daggered\Ket{\tilde{\Psi}} \\
    & = \left( i \hbar \dotOp{U}\op{U}\daggered
        + \op{U}\op{H}\op{U}\daggered \right)\Ket{\tilde{\Psi}}.
\end{split}
\end{equation}
So, the transformed Hamiltonian is
\begin{equation} \label{eq:ham_tilde}
  \tildeOp{H} = i \hbar \dotOp{U} \op{U}\daggered
                   + \op{U}\op{H}\op{U}\daggered.
\end{equation}

% rotating wave approximation


