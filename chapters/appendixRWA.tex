% vim: ft=tex iskeyword=@,48-57,_,-,192-255,\: dictionary=bibkeys.lst,labels.lst:
\chapter{The Rotating Wave Approximation}
\label{AppendixRWA}

In many cases, we can simplify the Hamiltonian and make it analytically
tractable by transforming it from the \emph{lab frame}
to the \emph{rotating frame}
oscillating at the laser frequency $\omega$. The rotating frame is defined by
a time-dependent unitary transformation $\Op{U}(t)$. Every state $\Ket{\Psi}$ in
the lab frame is transformed to the corresponding state in the rotating frame as
\begin{equation}
  \Ket{\tilde{\Psi}(t)} = \Op{U}(t) \Ket{\Psi(t)}
\end{equation}
In order to derive the Hamiltonian in the rotating frame, we demand the
Schrödinger equation to be fulfilled
\begin{equation}
  \ii \hbar \partdifquo{t} \Ket{\tilde{\Psi}} = \tildeOp{H}
  \Ket{\tilde{\Psi}}\,,
\end{equation}
which leads to
\begin{equation}
\begin{split}
  \ii \hbar \partdifquo{t} \Ket{\tilde{\Psi}}
    & = \ii \hbar \dotOp{U} \ket{\Psi}
        + \op{U} \ii \hbar \partdifquo{t} \ket{\Psi} \\
    & = \ii \hbar \dotOp{U} \ket{\Psi}
        + \op{U}\op{H}\ket{\Psi} \\
    & = \ii \hbar \dotOp{U} \op{U}\daggered \Ket{\tilde{\Psi}}
           + \op{U}\op{H}\op{U}\daggered\Ket{\tilde{\Psi}} \\
    & = \left( \ii \hbar \dotOp{U}\op{U}\daggered
        + \op{U}\op{H}\op{U}\daggered \right)\Ket{\tilde{\Psi}}.
\end{split}
\end{equation}
So, the transformed Hamiltonian is
\begin{equation} \label{eq:ham_tilde}
  \tildeOp{H} = \ii \hbar \dotOp{U} \op{U}\daggered
                   + \op{U}\op{H}\op{U}\daggered.
\end{equation}

For a system of two dipole-coupled levels separated by energy $\hbar \omega_1$,
driven by a pulse $\epsilon(t)$, the Hamiltonian reads
\begin{equation}
  \Op{H} = \begin{pmatrix}
    0               & \mu \epsilon(t) \\
    \mu \epsilon(t) & \hbar \omega_1
  \end{pmatrix}\,,
\end{equation}
where $\mu$ is the dipole strength. We consider the pulse
\begin{equation}
  \epsilon(t)
  = E(t) \cos(\omega_L t)
  = \frac{E(t)}{2} \left( \ee^{\ii \omega_L t} + \ee^{-\ii \omega_L t } \right)\,,
\end{equation}
with frequency $\omega_L$ and a slowly varying shape $E(t)$.
The rotating frame for this pulse is defined by
\begin{equation}
  \Op{U}(t) = \begin{pmatrix}
    1 & 0                   \\
    0 & \ee^{i \omega_L t}
  \end{pmatrix}\,.
\end{equation}
Applying Eq.~\eqref{eq:ham_tilde} yields
\begin{equation}
  \tildeOp{H} = \begin{pmatrix}
    0                                      & \mu \epsilon(t) \ee^{-\ii \omega_L t }   \\
    \mu \epsilon(t) \ee^{+\ii \omega_L t } & \hbar(\omega_1 - \omega_L)
  \end{pmatrix}\,.
\end{equation}
The energy level has been shifted down by $\omega_L$, resulting from the term
$\ii \hbar \dotOp{U}\Op{U}\daggered$, and the couplings obtain a time-dependent
phase-factor, due to the term $\op{U}\op{H}\op{U}\daggered$. Up to this point,
the transformation is exact. For the off-diagonal terms, we now find
\begin{equation}
  \mu \epsilon(t) \ee^{\pm \ii \omega_L t }
  = \frac{\mu}{2} \epsilon(t) \left( 1 + \ee^{\pm 2 \ii \omega_L t} \right)
  \approx
    \frac{\mu}{2} \epsilon(t) \equiv \Omega(t)\,.
\end{equation}
The approximation is valid for $\omega_L \gg 1$, where the fast oscillations at
twice the laser frequency average out, leaving only the slowly varying pulse
shape. With the detuning $\Delta \equiv \hbar(\omega_1 - \omega_L)$, the
RWA-Hamiltonian is therefore
\begin{equation}
  \Op{H}_{\text{RWA}} = \begin{pmatrix}
    0         & \Omega(t) \\
    \Omega(t) & \Delta
  \end{pmatrix}\,.
\end{equation}

In chapter~\ref{chap:robust}, a Cesium atom was considered where the ground
state $\Ket{0}$ is excited to the Rydberg state \Ket{r} by a two-photon
transition via the intermediary level \Ket{i}, see
Fig.~\ref{fig:RydRobust_1q_levels}. In the lab frame, the
Hamiltonian reads
\begin{equation}
  \Op{H} = \begin{pmatrix}
    0                   & \mu_B \epsilon_B(t) & 0                  \\
    \mu_B \epsilon_B(t) & \hbar \omega_i      & \mu_R \epsilon_R(t) \\
    0                   & \mu_R \epsilon_R(t) & \hbar \omega_r
  \end{pmatrix}\,,
\end{equation}
with the red and blue laser field
\begin{equation}
  \epsilon_{R,B}(t) = E_{R,B}(t) \cos(\omega_{R,B} t).
\end{equation}
The appropriate two-color RWA is defined by
\begin{equation}
  \Op{U} = \begin{pmatrix}
    1  & 0                    & 0                                   \\
    0  & \ee^{\ii \omega_B t} & 0                                   \\
    0  & 0                    & \ee^{\ii (\omega_B + \omega_R) t}
  \end{pmatrix}\,.
  \label{eq:U_two_color_rwa}
\end{equation}
The transformed Hamiltonian (with $\mu_{R,B}$ absorbed in $\epsilon_{R,B}$) is
\begin{equation}
\begin{split}
  \tildeOp{H}
  & =
  \begin{pmatrix}
    0                                   & \epsilon_B(t) \ee^{-\ii \omega_B t}                           & 0                                                               \\
    \epsilon_B(t) \ee^{\ii \omega_B t}  & \hbar(\omega_i - \omega_R)                                    & \epsilon_R(t) \ee^{-\ii (\omega_R + \omega_B)t +\ii \omega_B t} \\
    0                                   & \epsilon_R(t) \ee^{\ii (\omega_R + \omega_B)t -\ii \omega_Bt} & \hbar(\omega_r - \omega_R -\omega_B)
  \end{pmatrix} \\
  & \approx
  \begin{pmatrix}
   0           & \Omega_B(t) & 0           \\
   \Omega_B(t) & \Delta_1    & \Omega_R(t) \\
   0           & \Omega_R(t) & \Delta_2
  \end{pmatrix}\,,
\end{split}
\end{equation}
with
\begin{equation}
  \Omega_{R,B}(t) = \frac{1}{2} \mu_{R,B} E_{R,B}(t)
\end{equation}
and the single- and two-photon detuning $\Delta_1$ and $\Delta_2$, respectively.
This corresponds to Eq.~\eqref{eq:Rydberg_H_1q}.

In general, the appropriate RWA-transformation $\Op{U}$ can be
read off from a level diagram; \Op{U} is always diagonal with entries $\ee^{\ii
\omega_i t}$, where $\omega_i$ is the amount by which the $i$'th energy level is
shifted down. The shifts result from shortening each transition by the pulse
frequency that is to be eliminated. For example, starting from the diagram in
Fig~\ref{fig:RydRobust_1q_levels}, shortening the blue transition means
that both $\Ket{i}$ and $\Ket{r}$ are shifted down by $\omega_B$. Then,
shortening the red transition means that $\Ket{r}$ is shifted down by an
additional amount $\omega_R$, resulting in~Eq.~\eqref{eq:U_two_color_rwa}.
