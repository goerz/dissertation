% vim: ft=tex iskeyword=@,48-57,_,-,192-255,\: dictionary=bibkeys.lst,labels.lst:
\chapter{Quantum Gates with Superconducting Transmon Qubits}
\label{chap:transmon}

Superconducting circuits have emerged has one of the most promising
environments for quantum information processing. Unlike most other
implementations (e.g.\ trapped atoms like the Rydberg gate discussed in
chapter~\ref{chap:robust}), they are macroscopic, in the sense that the relevant
degrees of freedom are the quantities of electrical circuits, such as
charge, current, or flux. However, unlike in classical electrical circuits,
these variables are now fully quantized and described by a wave function. The
quantum behavior results from the electrons forming Cooper pairs and condensing
to a \emph{collective} quantum state as the system is cooled below some
(material-dependent) critical temperature $T_c$ \cite{TinkhamBook}.

From a technical perspective, superconducting qubits have the significant
advantage of building upon fabrication techniques used in standard integrated
circuits. A superconducting layer of usually aluminum or niobium is layered on
a silicon wafer
\cite{DevoretQIP2004, FrunzioITAS2005}.
The circuit elements are then patterned using the standard tools of optical or
electron-beam lithography and chemical etching: a photo-sensitive ``resin''
material is added to the chip and the illuminated with UV light or an electron
beam through a mask. Then, the developed resin is removed and the
exposed superconducting material is etched away, or alternatively, electrical
leads are deposited in the resin gaps \cite{HirstSCHandbook2003}.
For superconducting circuits relevant to quantum computing, one of the central
components is the \emph{Josephson junction}, described in
section~\ref{sec:jj}. It consists of two layers of superconducting material
separated by an insulating layer. This insulating layer in generally created by
oxidation of the superconducting material \cite{DevoretQIP2004,
FrunzioITAS2005}. Using these well-established production techniques,
a quantum computer based on a superconducting architecture could readily
produced industrially, provided that the more fundamental challenges in building
a large-scale quantum processor can be met, i.e., implementing universal quantum
gates and scaling the circuit to a significant number of qubits while
maintaining coherence.

This chapter reviews the fundamentals of superconducting qubits, with emphasis
on the \emph{transmon} design. Two transmons can be coupled via a shared
transmission line, allowing for the exchange of virtual photons. Using
perturbation theory, an effective model can be derived in which the qubit-cavity
coupling has been eliminated. This motivates the two-transmon Hamiltonian used
in chapters~\ref{chap:pe} and~\ref{chap:3states}.
We discuss some of the gate mechanisms that have been
used to implement two-qubit gates using transmons and explore possibilities for
optimal control.


\section{The Josephson Junction}
\label{sec:jj}
\index{Josephson junction}

\begin{figure}[htbp]
  \centering
  \includegraphics{jj}
  \caption{Josephson junction. The junction consists of two superconducting
  layers of e.g.\ aluminum, separated by an insulating layer, e.g.\ aluminum
  oxide. The junction combines the properties of a capacitor, characterized by
  $C_J$ or the charging energy $E_C \propto \frac{1}{C_J}$, and an inductive
  tunneling current, characterized by $I_0$ or the Josephson energy
  $E_J \propto I_0$.
  }
  \label{fig:jj}
\end{figure}

\begin{table}
  \centering
  \begin{tabular}{ll}
  \toprule
  \multicolumn{2}{c}{junction parameters} \\
  \midrule
  $I_0$        & critical current                                     \\
  $C_j$        & junction capacitance \\
  $Q_r$        & offset charge \\
  $E_c$        & charging energy, $E_c = \frac{1}{2}\frac{(2e)^2}{C_J}$ \\
  $E_J$        & Josephson energy, $E_J = \phi_0 I_0$ \\
  $L_{J0}$     & effective inductance, $L_{J0} = \frac{\phi_0}{I_0}$ \\
  \midrule
  \multicolumn{2}{c}{dynamic quantities} \\
  \midrule
  $\Phi(t)$     & branch flux of junction, see Eq.~(\ref{eq:branchflux}); $\phi(t) \in \Real$ \\
  $I(t)$        & Josephson current, see Eq.~(\ref{eq:jjcurrent}); $I(t) \in \Real$ \\
  $\delta(t)$   & ``phase'', $\delta(t) = \frac{2\pi \Phi(t)}{\Phi_0} = \frac{\Phi(t)}{\phi_0}$; $\delta \in \Real$ \\
  $\theta(t)$   & condensate phase diff., $\theta(t) = \delta(t)\!\!\mod 2 \pi$; $\theta(t) \in [0, 2 \pi)$ \\
  $L_J(\delta)$ & Josephson inductance, $L_J(\delta) = L_{J0} \cos^{-1} \delta$ \\
  $n(t)$        & number of tunneled cooper pairs; $n(t) \in \Integer$ \\
  \midrule
  \multicolumn{2}{c}{fundamental constants} \\
  \midrule
  $\Phi_0$     & flux quantum, $\Phi_0 = \frac{h}{2e}$ \\
  $\phi_0$     & reduced flux quantum, $\phi_0 = \frac{\Phi_0}{2\pi}$ \\
  \bottomrule
  \end{tabular}
  \caption{Summary of quantities and constants used for the characterization of
  a Josephson junction. The first group contains static parameters that depend on
  the geometric properties of the junction at production. An exception is $E_J$
  which can also be made tunable by splitting the junction in two and running an
  external flux through the resulting loop. The second group are the dynamic
  quantities from the which the tunneling energy and the
  capacitive energy of the junction derive, and which allow to formulate the
  Hamiltonian.  The ``phase'' $\delta(t)$ is more properly known as the
  ``gauge-invariant phase difference'' across the junction.
  \label{tab:jj_params}
  }
\end{table}


The Josephson junction consists of two superconducting leads separated by a thin
insulating layer, as shown in Fig.~\ref{fig:jj}. This structure acts as
a capacitor with capacitance $C_j$. As known from classical electrodynamics, if
one of the leads (capacitor plates) holds a charge of $Q$, the energy stored in the
capacitor is
\begin{equation}
  E_{\text{capacitor}} = \frac{1}{2} \frac{Q^2}{C_J}\,.
  \label{eq:E_capacity}
\end{equation}
In addition to the capacitive effect, Cooper pairs of electrons can tunnel
through the insulating layer (owing to the quantum mechanical nature of the
circuit), resulting in a \emph{Josephson current}
\index{Josephson current}%
that is described by \cite{JosephsonAP1965}
\begin{equation}
  I(t) = I_0 \sin \delta(t)
  = I_0 \sin \frac{2 \pi \Phi(t)}{\Phi_0}
  = I_0 \sin \frac{\Phi(t)}{\phi_0}\,,
  \label{eq:jjcurrent}
\end{equation}
where $I_0$ is the maximum current the junction can support, and $\delta(t)$ is
the phase difference of the wave function on each side of the junction, which is
directly related to the branch flux~\cite{devoret1995quantum}
\begin{equation}
  \Phi(t) = \int_{-\infty}^{t} V(t') \dd t'
  \label{eq:branchflux}
\end{equation}
\index{branch flux}%
of the junction element. All the
quantities typically used to describe the junction are listed in
Table~\ref{tab:jj_params}.

If the junction holds an initial charge difference (``offset charge'') of $Q_r$
and $n$ Cooper pairs of electrons (charge $2e$) tunnel through the junction, the
capacitive energy of the junction according to Eq.~(\ref{eq:E_capacity}) is
\begin{equation}
  E_{jj}^{\text{capacitive}} = E_c \left( n - \frac{Q_r}{2e} \right)^2\,,\qquad
  E_c = \frac{1}{2} \frac{(2e)^2}{C_J}\,.
\end{equation}
The offset charge $Q_r/2e$ will generally be much larger than $n$, and
cannot easily be controlled during the production process.
The energy due to the tunneling current is
\begin{equation}
  E_{jj}^{\text{inductive}}
  = \int\limits_{-\infty}^{t} I(t') V(t') \dd t'
  = \int\limits_{0}^{\Phi(t)} I_0 \sin\left(\frac{\Phi}{\phi_0} \right) \dd \Phi
  = -E_J \cos \delta\,,
\end{equation}
with $E_J = I_0\phi_0$, $\delta= \frac{\Phi}{\phi_0}$,
cf.~Table~\ref{tab:jj_params}, and $V(t) = \frac{\dd
\Phi(t)}{\dd t}$, cf.~Eq.~\eqref{eq:branchflux}.
Together, both terms yield the Hamiltonian
\begin{equation}
  \Op{H}_{jj}
  = E_c \left( \Op{n} - \frac{Q_r}{2e} \right)^2
    - E_{J} \cos \Op{\theta}\,.
 \label{eq:H_JJ}
\end{equation}
where $\Op{n}$ and $\Op{\theta}$ are quantum variables obeying the canonical
commutator relationship $[\Op{\theta}, \Op{n}] = \ii \hbar$.

\section{The Cooper Pair Box}
%
The Hamiltonian in Eq.~(\ref{eq:H_JJ}) is dominated by the value of $Q_r$,
which takes a random value during the production process and is subject to
significant noise fluctuations~\cite{NakamuraPRL2002}.
This makes the naked Josephson junction ill-suited as a qubit. To obtain
a well-defined system, there are three ``traditional'' superconducting qubit
designs that add further elements to the Josephson junction
circuit~\cite{DevoretArXiv0411174}: the Cooper
pair box (``charge qubit'') the flux qubit, and the phase
qubit.
\index{Cooper pair box}%
\index{charge qubit}%

The most straightforward of these is the charge qubit, obtained by biasing the
Josephson junction with a voltage source and placing it in series with
a capacitance $C_g$~\cite{BouchiatPS1998, NakamuraN1999}. The form of the
resulting Hamiltonian is identical to the naked Josephson junction in
Eq.~\eqref{eq:H_JJ}, but the offset charge, respectively the charging energy is
now a well-defined quantity, determined by $C_g$ and the bias voltage
$U$:
\begin{equation}
  \Op{H}_{CPB}
  = 4 E_C \left( \Op{n} - n_g \right)^2
    - E_{J} \cos \Op{\theta}\,,
 \label{eq:H_cooper1}
\end{equation}
with
\begin{equation}
  4 E_C = \frac{(2e)^2}{2\left( C_J + C_g \right)}\,,\qquad
  n_g = \frac{Q_r}{2e} + \frac{C_g U}{2e}\,.
  \label{eq:charging_energy}
\end{equation}
$E_J$ can be made tunable by splitting the Josephson junction and running
a magnetic flux through the resulting loop~\cite{VionCooper2003}.
Eq.~\eqref{eq:H_cooper1} can be written entirely in the charge
basis~\cite{SchusterPhD2007},
\begin{equation}
  \Op{H}_{CPB} = 4 E_C \left( \Op{n} - n_g \right)^2
          - \frac{E_J}{2} \sum_n \left(
            \KetBra{n}{n+1} + \KetBra{n+1}{n}
          \right)\,.
  \label{eq:H_cooper2}
\end{equation}
\begin{figure}[tb]
  \centering
  \includegraphics{charge_dispersion}
  \caption{Lowest five eigenvalues of the charge qubit Hamiltonian, obtained
  from diagonalization of Eq.~\eqref{eq:H_cooper2}, depending on the offset
  charge $n_g$, for different ratios of the qubit's Josephson and charging
  energies, cf.~\cite{JKochPRA07}.}
  \label{fig:charge_dispersion}
\end{figure}
The qubit is encoded in the lowest two eigenstates of $\Op{H}_{CPB}$.
The eigenvalues depend parametrically on the offset charge $n_g$; this
dependence is named the \emph{charge dispersion}.
\index{charge dispersion}
The separation of qubit levels as well as the charge dispersion depend on the
ratio $E_J/E_C$ , as shown in Fig.~\ref{fig:charge_dispersion}. Fluctuations in
$Q_r$ (and thus $n_g$) lead to dephasing of the qubit. Therefore, the qubit is
usually operated at the \emph{charge degeneracy point} $n_g = 0.5$, where the
charge dispersion is zero to first order. At $n_g=0.5$, small values of
$E_J/E_C$ result in the lowest two levels being separated by approximately $E_J$
from each other, and by approximately $8 E_C$ from the next higher level, making
the charge qubit a two-level system in very good approximation.
Higher values of $E_J/E_C$ lead to less sensitivity with respect to charge
noise, as the curves flatten out, but also decrease the anharmonicity of the
levels. Both in order to maintain a well-defined two-level system, and for
technical reasons related to the read-out of the qubit~\cite{CottetPhD2002},
$E_J/E_C < 1$ in the traditional charge qubit.


\section{The Transmon Qubit}
\index{transmon}

\subsection{A Charge-Insensitive Cooper Pair Box}

\begin{figure}[tb]
  \centering
  \includegraphics{transmon_photo_overlay}
  \caption{Optical image of transmon inside of a resonator, adapted
  from~\cite{SchusterPhD2007}. The circuit consists of five superconducting
  islands: the lower ground plane (1), the lower transmon island (2), the upper
  transmon island (3), the coplanar wave guide center pin (4), and the upper
  ground plane (5). The zoom shows the split Josephson junction (6). The islands
  (4) and (1) are connected with a bias voltage.
  }
  \label{fig:transmon_photo}
\end{figure}
The transmon qubit~\cite{JKochPRA07} extends the charge qubit by shunting it
with an additional capacitance $C_B$ and placing it in a coplanar wave guide
resonator. We refer to this resonator also as the transmission line, or
simply the cavity.
\index{coplanar wave guide resonator}%
All components are shown in Fig.~\ref{fig:transmon_photo}. The capacitance $C_B$
is between the islands (2) and (3). The capacitance $C_g$ is between island (3)
and (4). A full circuit analysis~\cite{JKochPRA07, SchusterPhD2007,ChowPhD2010}
shows that the Hamiltonian takes the same form as that of the charge qubit,
Eq.~\eqref{eq:H_cooper1}. However, $C_B$ increases the denominator of the
charging energy, Eq.~\eqref{eq:charging_energy}, lowering $E_C$ and thus
increasing the ratio $E_J/E_C$ to values of several tens or hundreds.
The design of the transmon exploits that with increasing $E_J/E_C$, the
charge dispersion decreases exponentially, while the anharmonicity only
decreases algebraically~\cite{CottetPhD2002}. Thus, one obtains a weakly
anharmonic ladder that is robust against charge noise, cf.~the right panel of
Fig.~\ref{fig:charge_dispersion}.
Choosing $E_J \gg E_C$ limits $\theta$ in Eq.~\eqref{eq:H_cooper1} to small
values, allowing to neglect $n_g$ and to expand the cosine potential in a Taylor
series as $\cos \theta \approx 1 - \frac{\theta^2}{2} + \frac{\phi^4}{24}$.
Rewriting the charge and phase states in terms of ladder operators $\Op{b}$,
$\Op{b}^\dagger$,
\begin{equation}
  \Op{n} = \frac{\ii}{2} \left( \frac{E_J}{2 E_C}\right)^{\frac{1}{4}}
           \left( \Op{b}^{\dagger} - \Op{b} \right)
  \qquad
  \Op{\theta} = \left( \frac{2 E_C}{E_J}\right)^{\frac{1}{4}}
           \left( \Op{b}^{\dagger} + \Op{b} \right)
\end{equation}
results in the \emph{Duffing oscillator} (up to a constant)
\index{Duffing oscillator}%
\begin{equation}
  \Op{H}_{T}
  = \sqrt{8 E_C E_J}\, \Op{b}^{\dagger}\Op{b}
    - \frac{E_C}{12} \left( \Op{b}^\dagger + \Op{b}\right)^4\,.
    \label{eq:duffing}
\end{equation}
Leading order perturbation theory brings this
to~\cite{JKochPRA07,GirvinLecture2011}
\begin{equation}
  \Op{H}_T = \omega_q \Op{b}^\dagger \Op{b}
             + \frac{\alpha}{2} \Op{b}^\dagger\Op{b}^\dagger \Op{b}\Op{b}\,,
\end{equation}
with the qubit frequency $\omega_q \approx \sqrt{8 E_J E_C} - E_C$ and the
anharmonicity $\alpha = -E_C$.
\index{transmon!qubit frequency}
\index{transmon!anharmonicity}

Besides contributing to the charging energy, the waveguide cavity is also
central for control and readout of the transmon. The interaction between the
qubit  with the ladder operator $\Op{b}$, $\Op{b}^{\dagger}$ and the cavity
with ladder operators $\Op{a}$, $\Op{a}^\dagger$ is described by
by~\cite{RicherMaster2013}
\begin{equation}
  H_I = g \left(\Op{b}^\dagger  + \Op{b}\right)
          \left(\Op{a}^\dagger  + \Op{a}\right)\,,
  \label{eq:qubit_cavity_interaction}
\end{equation}
\index{transmon-cavity interaction}%
where $g$ is the interaction strength. The transmon system is operated in the
\emph{strong coupling regime} where $g$ is much larger than the qubit and cavity
\index{strong coupling regime}%
decay rates $\gamma$ and $\kappa$. This allows photons to be exchanged between
the qubit and the cavity. If  $g \ll \omega_q, \omega_c$,
a rotating wave approximation yields a Jaynes-Cummings
Hamiltonian, in which excitations are preserved~\cite{BishopPhD2010}.
\index{Jaynes-Cummings Hamiltonian}
\begin{equation}
  H_I = g \left(\Op{b}^\dagger \Op{a}  + \Op{b} \Op{a}^{\dagger}\right)\,,
\end{equation}
with
\begin{equation}
  \gamma, \kappa \ll g \ll \omega_q, \omega_c\,.
  \label{eq:tm_jc_condition}
\end{equation}
The total Hamiltonian of the transmon and the transmission line is
\begin{equation}
 \Op{H}_T  =
   \omega_{q} \Op{b}\daggered \Op{b}
   + \frac{\alpha}{2} \Op{b}\daggered \Op{b}\daggered \Op{b}
   \Op{b}
   + g (\Op{b}\daggered \Op{a} + \Op{b} \Op{a}\daggered)
   + \omega_c \Op{a}\daggered\Op{a}\,.
   \label{eq:H_T}
\end{equation}
It is also possible to consider the transmon in the \emph{ultra-strong coupling
regime}, where the rotating wave approximation breaks
down~\cite{BeaudoinPRA2011}.
\index{ultra-strong coupling regime}
The eigenstates of Eq.~\eqref{eq:H_T} mix the bare eigenstates of the qubit and
the cavity; the degree of mixing is determined by the ratio of the detuning
$\omega_c - \omega_q$ and the coupling strength $g$. In the \emph{resonant
regime},
\index{resonant regime}%
$\omega_c = \omega_q$, the eigenstates of $\Op{H}_T$ are equal superposition of
qubit and cavity eigenstates. This regime is used for measurement and
characterization of the transmon device~\cite{JohnsonPhD2011}. For the
implementation of quantum gates, the qubit is usually tuned to the
\emph{dispersive regime},
\index{dispersive regime}%
\begin{equation}
  \Abs{\omega_c - \omega_q} \gg g\,.
  \label{eq:dispersive_condition}
\end{equation}
The eigenstates are now very close to the
bare qubit and cavity eigenstates. The slight mixing shifts the energy of qubit
and the cavity by a small amount $\chi$, see
section~\ref{sec:tm_effective_description}.

The parameters of the qubit and the coupling to the cavity can be engineered
with great flexibility.  Qubit frequencies can be readily set anywhere from 2 to
\SI{15}{GHz}, and couplings can range from a few kilohertz to nearly
\SI{1}{GHz}~\cite{HouckNP2012}. Moreover, when implemented with a split
Josephson junction as depicted in Fig.~\ref{fig:transmon_photo}, the qubit
frequencies can be dynamically tuned by applying a magnetic flux through the
junction loop.


\subsection{Coupled Transmon Qubits}

Connecting two transmon qubits to the same transmission line~\cite{MajerN2007}
creates an effective coupling between them. Each qubit is
characterized by its own frequency, anharmonicity, and coupling strength to the
cavity, $\omega_1$, $\alpha_1$, $g_1$, respectively $\omega_2$, $\omega_2$,
$\alpha_2$ in Hamiltonian~\eqref{eq:H_T}. Together, the Hamiltonian for the two
qubits and the shared transmission line with frequency $\omega_c$ reads
\begin{equation}
\begin{split}
 \Op{H}_{2T}
 & =
       \sum_{q=1,2} \left[ \omega_{q} \Op{b}_{q}\daggered \Op{b}_{q}
     + \frac{\alpha_q}{2} \Op{b}_{q}\daggered \Op{b}_{q}\daggered \Op{b}_{q}
     \Op{b}_{q}
     + g_q (\Op{b}_{q}\daggered \Op{a} + \Op{b}_{q} \Op{a}\daggered)
     \right]
 + \\ &\quad
     + \omega_c \Op{a}\daggered\Op{a}
     + \epsilon^*(t) \Op{a} + \epsilon(t) \Op{a}\daggered\,.
     \label{eq:tm_fullham}
\end{split}
\end{equation}
\index{transmon!full Hamiltonian}%
We have included a driving term with the external field $\epsilon(t)$ on the
transmission line. The field $\epsilon(t)$ is taken as complex to account for
the possibility of a rotating wave approximation. As described in
appendix~\ref{AppendixRWA}, this shifts $\omega_1$, $\omega_2$, $\omega_c$ by
the frequency of the lab frame $\epsilon(t)$ and  allows to describe the field
$\epsilon(t)$ by its shape only.

The interaction with the cavity allows for the exchange of \emph{virtual
photons}~\cite{MajerN2007}, where an excitation moves from qubit (1) to qubit
(2) without populating the cavity. Driving $\epsilon(t)$ at the qubit frequency
results in a cavity-mediated excitation of the qubit. Thus, the shared
transmission line resonator allows for extensive control over the two-transmon
system.

In addition to the coupling via the transmission line, the qubits could also be
coupled directly, either capacitively or inductively~\cite{ChowPhD2010}, adding
a static interaction term
$J (\Op{b}_1^{\dagger} \Op{b}_2 + \Op{b}_1 \Op{b}_2^{\dagger})$
to Eq.~\eqref{eq:tm_fullham}, where $J$ is the strength of the interaction.
In this case, the interaction must be
activated by dynamically tuning the qubit frequencies into resonance, see
section~\ref{sec:tm_gate_mechs}.


\section[Effect.\ Description of the Transmon in the Dispersive Regime]{Effective Description of the Transmon in the Dispersive Regime}
\label{sec:tm_effective_description}

Understanding how the cavity mediates an effective interaction between the
qubits is possible by partially diagonalizing the
Hamiltonian~\eqref{eq:tm_fullham} to remove the qubit-cavity coupling.
First, Eq.~(\ref{eq:tm_fullham}) is rewritten in the energy basis,
\begin{equation}
  \Op{b} = \sum_{i} \sqrt{i} \KetBra{i-1}{i}\,,
\end{equation}
to read
\begin{equation}
\begin{split}
 \Op{H}_{2T} &=  \underbrace{%
            \sum_{q=1,2}\sum_{i} \omega_{i}^{(q)} \Op{\Pi}_i^{(q)}
          + \omega_c \Op{a}\daggered\Op{a}
          }_{\Op{H}_0}
       + \underbrace{%
            \sum_{q=1,2}\sum_{i} g_{i}^{(q)} \left(
               \Op{C}^{+\,(q)}_{i} \Op{a}
              +\Op{C}^{-\,(q)}_{i} \Op{a}\daggered
            \right)
         }_{\Op{H}_I}
        + \\ &\quad
          + \epsilon^*(t) \Op{a} + \epsilon(t) \Op{a}\daggered\,,
\end{split}
\end{equation}
\index{transmon!effective Hamiltonian}%
with the qubit energy levels
\begin{equation}
  \omega_i^{(1,2)} = i \omega_{1,2} + \frac{\alpha_{1,2}}{2} (i^2 - i)\,,
\end{equation}
the qubit-cavity coupling
\begin{equation}
  g_i^{(1,2)} = \sqrt{i}\, g_{1,2}\,,
\end{equation}
and the projectors  and jump operators
\begin{equation}
\Op{\Pi}_i^{(1,2)} = \KetBra{i}{i}_{1,2}\,,\quad
\Op{C}_i^{+\,(1,2)} = \KetBra{i}{i-1}_{1,2}\,,\quad
\Op{C}_i^{-\,(1,2)} = \KetBra{i-1}{i}_{1,2}\,.
\end{equation}

The unitary transformation that decouples qubits and cavity takes the form
\begin{equation}
  \Op{H}' = e^{\Op{S}} \Op{H} e^{-\Op{S}}\,.
  \label{eq:swt}
\end{equation}
It is known as a Schrieffer-Wolff-transformation, with
\index{Schrieffer-Wolff-transformation}%
\begin{equation}
\Op{S}
= - \sum_{q=1,2}\sum_i
  \frac{g_i^{(q)}}{\omega_i^{(q)}-\omega_{i-1}^{(1)}-\omega_c}
  \left(  \Op{a}\daggered \Op{C}_{i}^{-\,(1)}
        - \Op{a} \Op{C}_{i}^{+\,(1)}
  \right)\,.
\label{eq:S}
\end{equation}
\Op{S} is constructed such that $[\Op{S}, \Op{H}_0]
= - \Op{H}_I$, cf.~Ref.~\cite{PolettoPRL2012}. Therefore, to first order, it
removes the unwanted interaction term $\Op{H}_I$.
Eq.~(\ref{eq:swt}) is evaluated using the Hadamard-Lemma,
\index{Hadamard-Lemma}%
\begin{equation}
  \Op{H}' =
  e^{\Op{S}} \Op{H} e^{-\Op{S}}
  = \Op{H} + [\Op{S}, \Op{H}] + \frac{1}{2} [\Op{S}, [\Op{S}, \Op{H}]] + \dots
\end{equation}
After some algebra~\cite{BasilewitschBA2013, RicherMaster2013}, one obtains the
transformed Hamiltonian
\begin{equation}
\begin{split}
\op{H}'
 &=   \omega_c \Op{a}\Op{a}\daggered
      + \epsilon^*(t)\Op{a} + \epsilon(t) \Op{a}\daggered
      + \sum_{q=1,2}\sum_{i=0}^{N_q-1} (\chi^{(q)}_i - \chi^{(q)}_{i+1})
               \Op{a}\daggered\Op{a} \Op{\Pi}_i^{(q)}
 \;+ \\ &\quad
   +  \sum_{q=1,2}\sum_{i=0}^{N_q-1} \left[
      (\omega^{(q)}_i+\chi^{(q)}_i)\Op{\Pi}^{(q)}_i
      + g^{\eff\,(q)}_i \epsilon(t)(\Op{C}^{+\,(q)}_i+\Op{C}^{-\,(q)}_i)
      \right]
 \;+ \\ &\quad
   + \sum_{ij}J^{\eff}_{ij}
     (\Op{C}^{-\,(1)}_i\Op{C}^{+\,(2)}_j+\Op{C}^{+\,(1)}_i\Op{C}^{-\,(2)}_j)\,,
\end{split}
\label{eq:dispersive_ham}
\end{equation}
truncated to $N_q$ transmon levels, with the ``effective'' quantities
\begin{subequations}
\begin{gather}
g^{\eff\,(1,2)}_i
= \frac{g^{(1,2)}_i}{(\omega^{(1,2)}_i-\omega_{i-1}^{(1,2)}-\omega_c)}\,,
\label{eq:tm_g_eff}
\\
  J^{\eff}_{ij}
  =   \frac{1}{2} g^{\eff\,(1)}_i g^{(2)}_j
    + \frac{1}{2} g^{\eff\,(2)}_j g^{(1)}_i\,,
  \label{eq:tm_J_eff}
 \\
\chi^{(1,2)}_i
= \frac{(g^{(1,2)}_i)^2} {(\omega^{(1,2)}_i-\omega^{(1,2)}_{i-1}-\omega_c)}\,.
\label{eq:tm_chi}
\end{gather}
\label{eq:tm_eff_quantities}
\end{subequations}
All terms in which $g_i^{\eff\,(1,2)}$ appears in higher than quadratic order
have been neglected.
The qubit levels obtain a \emph{Lamb-shift} $\chi_i$. Moreover, the term
proportional to $\Op{a}^\dagger \Op{a} \Op{\Pi}_i$ can be interpreted either as
a qubit-dependent shift of the cavity levels, or a further
\emph{ac Stark shift} of the qubit levels depending on the number of photons in
\index{Stark shift}%
the cavity.
The mediated interaction between the two qubits is given by $J_{ij}^{\eff}$.
Lastly, we see that $\epsilon(t)$ can drive the qubit transitions with a
reduced transition strength $g_i^{\eff}$. The Schrieffer-Wolff-transformation is
a perturbative treatment into orders of $g^{\eff}$, which must be
small, corresponding to the dispersive
condition~\eqref{eq:dispersive_condition}. Thus, Eq.~\eqref{eq:swt} is also
known as the dispersive transformation, and is closely related to the method of
\emph{adiabatic elimination}~\cite{ZhuPRB2013b}.

Assuming that the frequency of $\epsilon(t)$ is far detuned from $\omega_c$,
such that the cavity is never populated,
($\langle\Op{a}\daggered\Op{a}\rangle = 0$), it can be fully integrated out and
one obtains an effective Hamiltonian for the reduced system,
\begin{equation}
\begin{split}
\op{H}_{\red}
 &=   \sum_{q=1,2}\sum_{i=0}^{N_q-1} \left[
        (\omega^{(q)}_i+\chi^{(q)}_i)\Op{\Pi}^{(q)}_i
        + g^{\eff\,(q)}_i \epsilon(t)(\Op{C}^{+\,(q)}_i+\Op{C}^{-\,(q)}_i)
      \right]
 + \\ & \quad
   + \sum_{ij}J^{\eff}_{ij}
     \left(\Op{C}^{-\,(1)}_i\Op{C}^{+\,(2)}_j
           +\Op{C}^{+\,(1)}_i\Op{C}^{-\,(2)}_j
     \right)\,.
   \label{eq:H_red_dispersive}
\end{split}
\end{equation}
The derivation of effective models using Schrieffer-Wolff transformations,
perturbation theory, and adiabatic elimination theory is an extensive field of
study. Some of the methods are reviewed in
Ref.~\cite{RicherMaster2013}.
%There are a variety of alternative formulations
%of the dispersive transformation that forgo rewriting Eq.~\eqref{eq:tm_fullham}
%in the energy basis; these are especially useful when truncating the transmon to
%a two-level system (in the absence driving the qubit). More generally, they lead
%to complicated additional terms accounting for higher levels.
%Also, for strong qubit-cavity coupling, it may be necessary to go to
%higher-order perturbation theory~\cite{ZhuPRB2013b,BeaudoinPRA2011}.

Depending on the parameter regime, often a series of additional Schrieffer-Wolff
transformations on Eq.~\eqref{eq:dispersive_ham} is employed to further reduce
the degrees of freedom, e.g.\ to decouple well-separated
subspaces~\cite{PolettoPRL2012}. The validity of the dispersive and any further
transformation to a given order must be verified for each choice of parameters.

Approximately, Eq.~\ref{eq:H_red_dispersive} takes the form of an effective
Jaynes-Cummings Hamiltonian on the qubits~\cite{PolettoPRL2012}
\begin{equation}
\begin{split}
  \Op{H}_{\red}
  &
  \approx
    \sum_{q=1,2} \left(
        \left( \omega'_q - \frac{\alpha_q}{2}\right)
        \Op{b}_q^{\dagger} \Op{b}_q
        + \frac{\alpha_q}{2} \left( \Op{b}_q^{\dagger} \Op{b}_q \right)^2
    \right)
  + \\ \qquad &
  + J^{\eff} \left( \Op{b}_1^\dagger \Op{b}_2
                  + \Op{b}_1 \Op{b}_2^\dagger
            \right)
  + \epsilon(t) \left( \Op{b}_1 + \Op{b}_1^\dagger
                    + \lambda \Op{b}_2 + \lambda \Op{b}_2^\dagger \right)\,,
\end{split}
\label{eq:tm_jc_eff}
\end{equation}
where the Lamb shifts have been absorbed in $\omega'_q$ and $\lambda$ expresses
the relative effective driving strength between the two qubits. While
Eq.~\ref{eq:tm_jc_eff} matches the level-dependence of the effective
quantities in Eq.~\eqref{eq:tm_eff_quantities} only approximately, it is still
useful in practical applications. It is valid for weak anharmonicities and low
qubit excitation~\cite{RicherMaster2013}. Moreover, deviations from the exact
theory may be compensated by measuring the effective parameters experimentally,
instead of deriving them from first principles.
The Hamiltonian is used as a simplified model for transmon
qubits in chapters~\ref{chap:pe} and~\ref{chap:3states}.


\section{Gate Mechanisms for the Transmon}
\label{sec:tm_gate_mechs}

%\cite{MajerN2007}
A variety of approaches have been used implement entangling two-qubit gates on
transmon qubits. We give a brief overview over some of the techniques that have
been realized experimentally.

The conceptually most straightforward method is the \emph{Direct Resonant iSWAP
(DRI) Gate}~\cite{DewesPRL2012}.
\index{DRI gate}%
The transmons are coupled statically with a capacitor, yielding an
interaction on the order of \SI{4}{MHz}. Since there is no drive on the
transmission line resonator during the implementation of the gate, the cavity
can be integrated out and the transmon truncated to two levels. The effective
Hamiltonian therefore reads simply as
\begin{equation}
  \Op{H} = -\frac{\omega_1}{2} - \frac{\omega_2}{2} + g\, \SigmaY\SigmaY\,.
\end{equation}
The interaction is ``activated'' by tuning the qubit frequencies into resonance
on a time scale $\frac{1}{g} \ll \tau \ll \frac{1}{\omega_{1,2}}$,
using a magnetic flux through the split Josephson junction. After a duration of
$T=\frac{1}{8g}$, it generates a $\sqrtISWAP$ up to single qubit
transformations.

The \emph{higher-level resonance-induced dynamical CPHASE (DP) gate}~\cite{DiCarloN09}
explicitly exploits the effective interaction mediated by the cavity. It models
the system with a generalized form of the
Jaynes-Cummings-Hamiltonian of Eq.~\eqref{eq:tm_fullham} as
\begin{equation}
  \Op{H} = \omega_c \Op{a}^\dagger \Op{a}
          + \sum_{q=1,2} \left[
            \sum_{i=0}^{N_q-1}
            \omega^{(q)}_i \KetBra{i}{i}_q
            + \left(\Op{a} + \Op{a}^{\dagger}\right) \sum_{i,j=0}^{N_q-1}
            g_{jk}^{(q)} \KetBra{i}{j}_q
          \right]\,,
\end{equation}
with all anharmonicities absorbed in $\omega^{(q)}_i$ and with a coupling
matrix $g_{jk}^{(q)}$. The qubit frequencies again are flux-tunable. For the
parameter choices in Ref.~\cite{DiCarloN09}, $\omega_c = \SI{6.9}{GHz}$ and
$g_1=\SI{199}{MHz}$, $g_2=\SI{183}{MHz}$, the logical level $\Ket{02}$ decreases
more rapidly than level $\Ket{11}$ when tuning the qubit frequencies. For
a specific flux value, these two levels would become degenerate. However, the
cavity also mediates a coupling $\Ket{11} \leftrightarrow \Ket{02}$,
cf.~Eq.~\eqref{eq:tm_jc_eff}, causing an avoided crossing at the degeneracy point.
\index{avoided crossing}%
The resulting large shift in $\Ket{11}$ induces a relative phase and thus
implements a CPHASE gate on a time scale of \SI{30}{ns}, without applying
a field to the waveguide resonator. However, this neglects the time required to
bring tune the qubits into the interaction regime. As the avoided crossing is
approached, the tuning speed must be sufficiently slow to remain adiabatic,
inversely proportional to the splitting between $\Ket{11}$ and $\Ket{02}$.

Using the flux-tunability of the qubit frequency has the disadvantage of reduced
coherence times due to flux noise and the risk of unwanted interactions during
the tuning process~\cite{ChowNJP2013}. An alternative is the use of shaped
microwave pulses on the transmission line resonator.

The \emph{resonator-sideband-induced (RSI) gate}~\cite{LeekPRB2009}
\index{RSI gate}%
employs sideband transitions~\cite{BlaisPRA2007} common in trapped-ion
implementations of quantum computing~\cite{CiracPRL95, HaffnerPhysRep08}.
Unlike the DRI or DP gates, the system is actively driven with pulses on the
transmission line.
The basic idea is to entangle the qubit state with the first excitation of the
cavity, and then to transfer this entanglement onto the two-qubit state. The
generation of entanglement can be understood by considering the initial state
$\Ket{000}$, where the quantum numbers indicate the excitation of the first
qubit, second qubit, and the cavity. A series of pulses is applied, consisting
of
\begin{enumerate}
  \item a $\pi$ pulse (see appendix~\ref{AppendixRabi}) on the second qubit,
  \item a $\frac{\pi}{2}$ pulse on the sideband of the first qubit, and
  \item a $\pi$-pulse on the sideband transition of the second qubit.
\end{enumerate}
The sideband transition excites or de-excites both the qubit and the cavity
simultaneously via a two-photon transition. This transforms the
qubit-qubit-cavity state as (neglecting global phases)
\begin{equation*}
  \Ket{000} \rightarrow
  \Ket{010} \rightarrow
  \frac{1}{\sqrt{2}} \left( \Ket{010} + \Ket{111} \right) \rightarrow
  \frac{1}{\sqrt{2}} \left( \Ket{010} + \ii \Ket{100} \right)\,.
\end{equation*}
In total, it brings the initial separable two-qubit state to an entangled Bell
state,
$$\Ket{00} \rightarrow \frac{1}{\sqrt{2}} \left( \Ket{01} + \ii \Ket{10} \right).$$
Note that in the last step, there exists no right-qubit blue sideband transition
for the first term $\Ket{010}$, which enables the transfer of the entanglement
to the qubits. The duration of the full pulse sequence for the parameters in
Ref.~\cite{LeekPRB2009} is $T=\SI{150}{ns}$.
As described in Ref.~\cite{Schmidt-KalerN03}, the mechanism can be
extended to implement a CNOT gate. Generally, many of the gate mechanisms for
trapped ions can be translated to superconducting qubits, with the cavity
taking the place of the vibrational excitation of the ions.

%%% Cross-Resonator gate
%%% * start from effective Hamiltonian
%%% * diagonalize -> shifted frequencies w1' w2'
%%% * drive qubit 1 at w_2 can be shown to generate effective \SigmaZ\SigmaX
%%%   interaction
%%% Not relevant here, because:
%%% * flux qubits
%%% * two-level approximation
%%% * single-site addressability

Two further gate implementations for transmon qubits exploit specific resonance
conditions with levels outside of the logical subspace. Both of these start from
an effective model, Eq.~\eqref{eq:tm_jc_eff}, where the qubit-cavity interaction
has been transformed into an effective qubit-qubit interaction, and an effective
driving term for the qubit transitions. Thus, we describe the state of the
system in the basis $\Ket{ij}$, where $i$ and $j$ are the excitation quantum
numbers of the left and right qubit, respectively.
While in general, the effective interaction is weak, building resonances into
the higher qubit levels can significantly enhance the generation of
entanglement.
In the case of the \emph{Bell-Rabi (BR) Gate}~\cite{PolettoPRL2012}, the
\index{BR gate}%
transmon is engineered such that the qubit-qubit detuning is close to the
anharmonicity of one of the qubits, $\omega_1 - \omega_2 \approx \alpha_1$.
This brings the levels $\Ket{11}$ and $\Ket{20}$ into resonance. Driving at
a frequency slightly detuned by $\delta$ from the center between both qubits,
\begin{equation}
  \omega_d = \frac{1}{2} \left(\omega_1 + \omega_2 \right) - \delta\,,
  \label{eq:br_gate_drive}
\end{equation}
the levels $\Ket{00}$, $\Ket{11}$, $\Ket{21}$ are close to zero in the rotating
frame at $\omega_d$, and far detuned from the remaining levels. Using
a Schrieffer-Wolff transformation similar to the one discussed in
section~\ref{sec:tm_effective_description}, the coupling between this
``low-energy manifold'' and the remaining Hilbert space can be eliminated. In
the frame defined by the transformation, it can be shown~\cite{PolettoPRL2012}
that there is a resulting two-photon transition between the state \Ket{00} and
\Ket{11}.  With a specific choice of the drive-detuning $\delta$, this
implements a gate that is equivalent to $\sqrtISWAP$.

The \emph{Microwave-Activated CPHASE (MAP) Gate}~\cite{ChowNJP2013}
\index{MAP gate}
uses an alternative resonance condition. Here, the qubits are engineered such
that $\Ket{12}$ and $\Ket{03}$ align, which translates to the condition
\begin{equation}
  \omega_1 - \omega_2 = 2 \alpha_2\,.
  \label{eq:map_gate_condition}
\end{equation}
The effective coupling between these two levels
shifts them, thus lifting the degeneracy of the two transitions
$\Ket{01} \leftrightarrow \Ket{02}$ and
$\Ket{11} \leftrightarrow \Ket{12}$. The system is driven with a detuning of
$\delta$ from these two transitions,
\begin{equation}
  \omega_d = \omega_2 + \alpha_2 - \delta\,.
\end{equation}
With the resonance condition in Eq.~\eqref{eq:map_gate_condition},
$\omega_d$ takes the same form as Eq.~\eqref{eq:br_gate_drive}. In the rotating
frame with $\omega_d$, the combined Hamiltonian (drift Hamiltonian and drive
Hamiltonian for a field with amplitude $\Omega$) can be fully diagonalized. The
field-dressed level \Ket{11} obtains a Stark shift. The shift induces a CPHASE
gate, with the total interaction energy
\begin{equation}
  \zeta \approx \zeta_0 +  \frac{\Omega^2}{2 \delta} \zeta_2\,,
  \label{eq:MAP_Stark}
\end{equation}
where $\zeta_0$ is the static interaction and the term proportional to $\zeta_2$
is the field-induced interaction, both defined in Ref.~\cite{ChowNJP2013}.
For the parameters chosen there, a fully entangling gate is reached after
\SI{510}{ns}.
A detailed discussion of the effective Hamiltonian and the gate mechanism for
both the BR and MAP gate may be found in Ref.~\cite{RicherMaster2013}.

\section{A Holonomic Phasegate}
\label{sec:holonomic}

We might wonder whether we can generalize the idea of using Stark shifts
\index{Stark shift}%
for the implementation of gates to include the cavity. Instead of inducing the
shift in the MAP gate by driving an off-resonant transition to higher qubit
levels, it has been proposed to use a transition to a cavity level,
implementing a \emph{Driven Resonator-Induced CPHASE (RIP)
Gate}~\cite{ChowNJP2013,GambettaAPS2012}.
\index{RIP gate}%
As an alternative picture, we could think of the resonance induced dynamical
phase (DP) gate discussed in section~\ref{sec:tm_gate_mechs}, but instead of
flux-tuning the qubit frequencies into resonance, the necessary shifts would be
obtained by driving the system with an off-resonant microwave field near the
cavity frequency.

We consider the full Hamiltonian of two transmon qubits coupled via a shared
transmission line (cavity), Eq.~\eqref{eq:tm_fullham}, that is driven at
a frequency $\omega_d$ detuned by a few tens of MHz from the cavity frequency
$\omega_c$,
\begin{equation}
  \epsilon(t) = \epsilon_0 S(t) \cos(\omega_d t), \quad S(t) \in [0,1]\,.
  \label{eq:tm_epsilon_t}
\end{equation}
The Hamiltonian can be transformed to a rotating frame at $\omega_d$, see
appendix~\ref{AppendixRWA}, with the basis transformation
\begin{equation}
  \Op{U}_{\text{RWA}}(t)
  = \sum_{ijn} \ee^{\ii (i+j+n) \omega_d t}
    \KetBra{ijn}{ijn}\,,
\end{equation}
where $i$, $j$, $n$ are the quantum numbers for the excitation of the left qubit,
right qubit, and the cavity, respectively. Assuming Eq.~\eqref{eq:tm_epsilon_t},
the field in the rotating wave approximation (RWA) is
\begin{equation}
  \Omega(t) = \frac{1}{2} \epsilon_0 S(t)\,.
  \label{eq:tm_Omega_t}
\end{equation}
Going to the RWA allows to obtain the field-dressed
energy levels by diagonalization of $\Op{H}_{\text{RWA}}(t)$ for a fixed value
of field amplitude $\epsilon_{0}$. A two-qubit interaction
\begin{equation}
  E_{\gamma}(\epsilon_0) = E_{00}(\epsilon_0) - E_{01}(\epsilon_0)
                          - E_{10}(\epsilon_0) + E_{11}(\epsilon_0)
  \label{eq:E_gamma}
\end{equation}
results if the energy levels that define the logical subspace are
shifted non-linearly. Keeping the field at amplitude $\epsilon_0$
for duration $T$ implements a diagonal gate
\begin{equation}
  D = \diag\left\{
        \ee^{\ii \phi_{00}},
        \ee^{\ii \phi_{01}},
        \ee^{\ii \phi_{10}},
        \ee^{\ii \phi_{11}}
     \right\}\,,
\end{equation}
where the $\phi_{00} = \ee^{\ii E_{00}(\epsilon_0) T}$, and equivalently for
$\phi_{01}$, $\phi_{10}$, $\phi_{11}$. As discussed in
chapter~\ref{chap:quantum}, this gate is a CPHASE gate with the entangling phase
\begin{equation}
  \gamma(\epsilon_0,T) = E_{\gamma}(\epsilon_0) T\,,
  \label{eq:hol_gamma}
\end{equation}
cf.~Eq.~\eqref{eq:entangling_phase}, up to single-qubit operations.

However, the implementation of the gate assumes that the population is in the
field-dressed logical eigenstates at every point in time. This can be ensured by
switching the pulse shape $S(t)$ on and off adiabatically: at $t=0$, the pulse
is off and the system is in one of the logical eigenstates, e.g.~$\Ket{00}$. The
pulse is then slowly switched on, reaching its peak value $\epsilon_0$
at $t=T/2$, after which it is slowly switched off again, reaching zero at $t=T$.
If the switch-on and switch-off is sufficiently slow, then
according to the adiabatic theorem,
\index{adiabatic theorem}%
the system will remain in the field-dressed logical eigenstate with energy
e.g.  $E_{00}(\Omega(t))$ at every point in time. Specifically, the derivative
of $\Omega(t)$  must be sufficiently small to avoid jumping over any avoided
crossing, as described by Landau-Zener's formula~\cite{ZenerPRSA1932}.
\index{Landau-Zener formula}%
% http://de.wikipedia.org/wiki/Adiabatisches_Theorem_der_Quantenmechanik
At final time $T$, the system returns to its original state, but with
a geometric Berry phase~\cite{BerryPRSA1984},
\index{Berry phase}%
e.g.
\begin{equation}
  \phi_{00} = \int_{0}^{T} E_{00}(\Omega(t)) \dd t\,.
\end{equation}
Thus, following the terminology of Ref.~\cite{ZanardiPLA1999}, we refer to this
realization of a diagonal gate as a \emph{holonomic phasegate}.
\index{holonomic phasegate}%

\subsection{Entanglement Creation}

\begin{figure}[p]
  \centering
  \includegraphics{holonomic_entanglement}
  \caption{Entanglement (concurrence) generated after $T=\SI{200}{ns}$, over the
  peak amplitude of the pulse, for different choices of pulse driving frequency
  $\omega_d$. The top and bottom panels shows results for a cavity frequency of
  \SI{860}{MHz} left and right of the qubits, respectively. The remaining
  parameters are listed in Table~\ref{tab:hol_params}. The gates implemented at
  the select points labeled 1--3 are analyzed in
  Table~\ref{tab:hol_entanglement_tab}.
  }
  \label{fig:holonomic_entanglement}
\end{figure}
\begin{table}[p]
  \centering
  \begin{tabular}{llrl}
  \toprule
  cavity frequency      &  $\omega_c$            & 6.00,~8.11 &GHz \\
  left qubit frequency  &  $\omega_1$            & 6.86       &GHz \\
  right qubit frequency &  $\omega_2$            & 7.25       &GHz \\
  qubit anharmonicity   &  $\alpha_1, \alpha_2$  & -300       &MHz \\
  qubit-cavity coupling &  $g_1, g_2$            &   70       &MHz \\
  \bottomrule
  \end{tabular}
  \caption{Parameters for the transmon Hamiltonian
  Eq.~\eqref{eq:tm_fullham} for a holonomic phase gate.
  }
  \label{tab:hol_params}
\end{table}
\begin{table}[p]
  \sisetup{round-mode=places,round-precision=2}
  \centering
  {\begin{tabular}{lSSSScc}
  \toprule
   point  & {$1-C$}        & {pop.\ loss}       & {$\Avg{i,j}_{\max}$} & {$\Avg{n}_{\max}$} & {$n_q$} & {$n_c$} \\
  \midrule
   1 & \num{1.56E-03} & \num{1.695483E-02} & 1.5453329107       & 7.7023907042       & 10      & 25 \\
   2 & \num{1.43E-02} & \num{9.891985E-03} & 3.0381299437       & 12.918444204       & 11      & 30 \\
   3 & \num{3.19E-03} & \num{3.602988E-03} & 1.0434286055       & 48.4647983895      & ~6      & 70 \\
  \bottomrule
  \end{tabular}}
  \caption{%
  Properties of gates implemented for the parameters at labeled points in
  Fig.~\ref{fig:holonomic_entanglement}. For each point, the entanglement error
  $1-C$ is given, cf.~Fig.~\ref{fig:holonomic_entanglement} for the values of
  the concurrence $C$. Furthermore, the loss of population from the logical
  subspace at final time $T$, the peak expectation value $\Avg{i,j}_{\max}$ at
  $t=\frac{T}{2}$ for the excitation of either one of the qubits, and the peak
  expectation value $\Avg{n}_{\max}$ for the excitation of the cavity are
  listed. Lastly, $n_q$ and $n_c$ are the number of qubit and cavity levels that
  must be taken into account to reach numerical convergence.
  }
  \label{tab:hol_entanglement_tab}
\end{table}
The effective entangling energy obtained at final time $T$ depends not only on
the peak amplitude of the pulse, but also on the choice of the system parameters
and the driving frequency of the microwave pulse. This is explored in
Fig.~\ref{fig:holonomic_entanglement}, showing the entanglement generated by
a Blackman pulse of duration $T=\SI{200}{ns}$ with varying peak amplitude
$\epsilon_0$. The Blackman
shape is nearly identical to a Gaussian centered around
$\frac{T}{2}$ with width $\sigma=\frac{T}{6}$, but is exactly zero at
$t=0$ and $t=T$, see Eq.~\eqref{eq:blackman}.
The qubit parameters are listed in Table~\ref{tab:hol_params}.
Even in the field-free case, $\epsilon_0=0$, the static coupling
between qubit and cavity induces a Lamb-shift in the qubit levels. To
first order, these shifts are described by $\chi_{i}^{(1,2)}$ in
Eq.~\eqref{eq:H_red_dispersive}. Considering only the levels of the logical
subspace, the cavity-mediated interaction energy is
$\zeta \equiv E_{\gamma}(\epsilon=0)$ with $E_{\gamma}$ defined in
Eq.~\eqref{eq:E_gamma}. At final time $T$, this results in the entanglement
\begin{equation}
  C(\zeta, T) = \sin\left[\frac{1}{2} \zeta\, T \right]\,,
\end{equation}
representing the y-intercept in Fig.~\ref{fig:holonomic_entanglement}.
Note that for the DP gate, the qubit parameters are flux-tuned to ensure
$C(\zeta, T) = 1$. Here, the qubit parameters are not tuned,
but the interaction due to the pulse-induced Stark shifts (from the combination
of all terms proportional to $\Op{a}$, $\Op{a}^{\dagger}$ in
Eq.~\eqref{eq:H_red_dispersive}, and higher order) yields the entanglement for
the realization of the gate. For a two-qubit gate, it must add to the always-on
entanglement due to $\zeta$, whereas for single qubit gates, it must cancel
$\zeta$.  The field-free
parameters in Table~\ref{tab:hol_params} are chosen such that the entanglement
created due to $\zeta$ at time $T$ is less than $0.5$, to ensure that the
realization of single-qubit gates is not harder than that of two-qubit gates.

The magnitude of the Stark shift is proportional to $\Abs{\epsilon_0}^2$,
i.e., it depends quadratically on the field. This reflected in the parabolic
shape of the curves in Fig.~\ref{fig:holonomic_entanglement}.

Generally, the magnitude and sign of the Lamb and Stark shifts depends on the
magnitude and sign of the qubit-qubit detuning, the qubit-cavity
detuning, and, the cavity-drive detuning.  Instead of an exact
theoretical analysis, based on  Eq.~\eqref{eq:H_red_dispersive} and higher order
terms (a procedure that quickly becomes too tedious to perform by hand),
we can obtain empirical insight into how these quantities affect the generation
of entanglement from Fig.~\ref{fig:holonomic_entanglement}.

The two panels consider the situation where the cavity frequency is left or
right of the qubits, while the qubits are at fixed frequencies $\omega_1$ and
$\omega_2$, see Table~\ref{tab:hol_params}. In both cases, the absolute value of
the qubit-cavity detuning is the same, but the sign of the detuning changes.
This leads to a change in the always-on interaction $\zeta$, resulting in less
entanglement in the field-free case when $\omega_c > \omega_{1,2}$.

Furthermore, when higher levels of the transmon are taken into account, the
Hamiltonians for the two choices of $\omega_c$ are in fact \emph{not} symmetric
with respect to the qubits:
With a negative anharmonicity, higher qubit transitions are closer to
the cavity frequency if $\omega_c < \omega_{1,2}$ and increasingly farther
detuned if $\omega_c > \omega_{1,2}$. Since avoided crossings with higher
levels can be relevant for the resulting shifts in the levels of the logical
subspace, this can have considerable impact on the implementation of the gate.

The choice of the drive frequency $w_d$ as either smaller or larger than the
cavity frequency determines whether the Lamb and Stark shifts interfere
constructively or destructively; driving left of cavity (solid blue and orange
lines) results in constructive interference, whereas driving right of
the cavity (dashed red and green lines) results in destructive interference.
This is independent of the sign of the qubit-cavity detuning, as the same
behavior is observed in both panels.

In the case of constructive interference, we observe that the rate at which
entanglement is created is proportional to $\zeta$ , from the greater slope of
the solid blue and orange curve in the top panel, compared to the bottom panel.
This may be expected from Eq.~\eqref{eq:H_red_dispersive} by the observation
that both the Lamb shift and the Stark shift contain $\chi_i^{(q)}$ as the
leading term. Interestingly, in the case of destructive interference
($\omega_d > \omega_c$) for a driving frequency close to the cavity (dashed red
curve), we do not observe the same behavior; this curve is simply shifted down
in the bottom panel. Presumably, this choice of parameters causes
higher order terms in the Stark shift to cancel the leading order term.

The implication of this is that for small values of $\zeta$, the fastest way to
realize an entangling gate (respectively, using the smallest pulse amplitude)
may be to choose a drive frequency such that the entanglement from the Lamb and
the Stark shift interfere destructively. For small pulse amplitudes,
respectively short durations, the same drive frequency could be used to
implement both single qubit gates ($\epsilon_0 = \SI{125}{MHz}$ for the red
curve in the bottom panel), while stronger drives, respectively longer durations,
would allow the realization of an entangling gate (point 3 at $\epsilon_0
= \SI{425}{MHz}$ in the bottom panel). In contrast, for the top panel one would
implement entangling gates at points labeled 1 or 2, but switch the drive
frequency to $\omega_d > \omega_c$ for single-qubit gates.

Setting the drive frequency closer to that of the cavity increases the Stark
shift significantly, cf.~\eqref{eq:MAP_Stark}, as shown by the greater slope of
the orange and red curves relative to the blue and green curves, in both panels.
However, getting too close to the cavity will drive up cavity population
significantly. Also, it will cause non-adiabatic effects. That is, the
dynamics might jump over at least some of the avoided crossings. These are hinted
at by kinks in the curves of Fig.~\ref{fig:holonomic_entanglement}. In more
detail, we consider the numbered points in Fig.~\ref{fig:holonomic_entanglement}.
The properties of the dynamics at these
points are listed in Table~\ref{tab:hol_entanglement_tab}.

\begin{figure}[tb]
  \centering
  \includegraphics{adiabatic_popdyn}
  \caption{Exemplary population dynamics for an adiabatic pulse. The dynamics
  shown are for the propagation of the logical eigenstate \Ket{00} for the
  parameters corresponding to point (3) in
  the bottom panel of Fig.~\ref{fig:holonomic_entanglement}. Panels (a-c) show
  the expectation value of the cavity (a), the right qubit (b), and the left
  qubit (c), in the bare basis. The standard deviation for each expectation
  value is shown in gray. In panel~(d), population in the (non-field-dressed)
  logical eigenstate \Ket{00}. The pulse shaped, normalized to the peak
  amplitude $\epsilon_0 = \SI{425}{MHz}$, is shown in light blue.
  }
  \label{fig:adiabatic_popdyn}
  % pop 00 from params2d40_T200 (''point 3'')
\end{figure}
The population dynamics of the state $\Ket{00}$ for the parameters at
point 3 are shown in Fig.~\ref{fig:adiabatic_popdyn}.
The dynamics are exemplary for an adiabatic evolution.
Note that excitations are the projection of the
field-dressed states on the non-dressed basis. This shows intuitively the effect
of the off-resonant drive: it shifts the qubit and cavity wavepackets out of
their equilibrium position proportionally to the square of the field
amplitude shown in light blue in panel~(d).
Most importantly, the adiabaticity of the dynamics result in the population,
panel~(d), returning exactly to the initial state at final time $T$, up to
a Berry phase.  Non-adiabatically jumping over an avoided crossing would be
reflected in the plot of the qubit and cavity excitation by kinks and
asymmetries; the excitation would no longer smoothly follow the pulse shape. As
a consequence, the system would generally no longer return to the initial state.
The non-adiabaticity can be measured by the loss from the logical subspace,
\begin{equation}
  p_{\loss} = 1-\frac{1}{4} \Tr\left[ \tildeOp{U}^\dagger \tildeOp{U} \right]\,,
  \label{eq:pop_loss}
\end{equation}
where $\tildeOp{U}$ is the projection of the full time evolution operator onto
the logical subspace.
For the three numbered points in Fig.~\ref{fig:holonomic_entanglement}, the
value of $p_{\loss}$ is listed in Table~\ref{tab:hol_entanglement_tab}.
Points 1 and 2 show a significantly larger loss than point 3. Accordingly, for
point 3, there are no discernible nonadiabatic effects visible in the population
dynamics, while for points 1 and 2, the propagation of at least one of the
logical basis states has some visible deviation from the ideal dynamics.

The shift of the qubit and cavity excitation is also inverse proportional to
the detuning of the drive from the respective frequency.  Thus, in
Fig.~\ref{fig:adiabatic_popdyn}, the cavity, panel~(a), is strongly excited,
whereas the right qubit, panel~(b), is only slightly
excited, as it is far off-resonant. The left qubit, panel~(c), is even
further detuned from the drive, and is thus receives even less excitation.
The data for the peak expectation value for the qubit and cavity excitation
in Table~\ref{tab:hol_entanglement_tab} illustrates this further. For points
1 and 2, $\omega_{1}, \omega_2 < \omega_d < \omega_c$, and both qubit and cavity
population increase as the pulse amplitude increases from
$\epsilon_0 = \SI{125}{MHz}$ for point 1 to $\epsilon_0 = \SI{250}{MHz}$ for
point 2. However, since for point 3, the cavity drive is much farther detuned
from the qubits, $\omega_{1}, \omega_2 < \omega_c < \omega_d$, the qubit
excitation is much lower, even though the pulse amplitude increases to
$\epsilon_0 = \SI{425}{MHZ}$.

Another observation based on the data of Table~\ref{tab:hol_entanglement_tab} is
that for points 1 and 2, the number of qubit and cavity levels that need to be
taken into account in the dynamics is higher than the excitation would imply.
This illustrates that higher levels of the transmon can be relevant to the
realization of the gate, even without being populated.
The same situation occurs in the DP gate mechanism in
section~\ref{sec:tm_gate_mechs}, where the $\Ket{02}$ level causes the shift in
the $\Ket{11}$ level, without ever being populated.
When the cavity and drive frequency are to the right of the qubits, far less
qubit levels need to be taken into account.

Beyond the parameters explored in Table~\ref{tab:hol_params} and
Fig.~\ref{fig:holonomic_entanglement}, increasing the qubit-cavity
detuning has the effect of reducing the entanglement generated by the Stark
shift. Decreasing it to values significantly smaller than those used in
Table~\ref{tab:hol_params} leads to strongly non-adiabatic dynamics.
Similarly, lowering the value of the qubit-cavity coupling $g$ reduces the
generation of entanglement, but enlarging it leads to undesirably large values
of $\zeta$.

\enlargethispage{\baselineskip}
The empirical observations on the parameter-dependence of the entanglement
creation are essential for a good choice of parameters for the realization of
a holonomic phasegate, and moreover they are relevant for guiding a more
rigorous analytical derivation. Both points 1 and 3 in
Fig.~\ref{fig:holonomic_entanglement} are worthwhile for further consideration.
Placing the cavity frequency to the right of the qubits has the significant
benefits of providing a lower value of $\zeta$, allowing to implement single and
two-qubit gates with the same drive frequency, showing more adiabatic behavior
despite stronger pulse amplitudes, and exciting the qubit levels far less.
However, the strong pulses imply a large cavity population. Placing the cavity
left of the qubits allows to implement the gate with far less pulse amplitude
and cavity excitation, but runs the risk of population loss from the logical
subspace due to non-adiabatic effects.


\subsection{Simplex Optimization of a Holonomic Gate}
\label{subsec:hol_simplex}

Since the points in Fig.~\ref{fig:holonomic_entanglement} only sample
a discrete set of parameter choices, none of the gates considered there are
exact perfect entanglers, cf.\ the entanglement error in
Table~\ref{tab:holonomic_simplex_results}.
Furthermore, the gate duration has been fixed at $T=\SI{200}{ns}$. In order to
find exact perfect entanglers and also to minimize the gate duration, we can use
a simplex optimization as discussed in chapter~\ref{chap:numerics} to optimize
the gate duration and the peak amplitude of the Blackman shape, for the two most
promising types of parameters of Fig.~\ref{fig:holonomic_entanglement},
$\omega_d < \omega_c=\SI{6.0}{GHz} < \omega_1, \omega_2$ and
$\omega_d > \omega_c=\SI{8.11}{GHz} > \omega_1, \omega_2$.
These correspond to the most promising parameter sets from
Fig.~\ref{fig:holonomic_entanglement}: the orange and blue curve in the top
panel, and the green and red curve in the bottom panel.
We also explore the option of using smaller pulse-cavity detunings.

The figure of merit for the optimization must encode three requirements:
\begin{enumerate}
  \item The obtained gate must be a perfect entangler, as measured by the
  non-local phase $\gamma$ defined in Eq.~\eqref{eq:hol_gamma}, which should
  take the value $\pi$.
  \item The gate duration $T$ should be as short as possible.
  \item The dynamics should be adiabatic, as measured by population loss
        from the logical subspace, Eq.~\eqref{eq:pop_loss}.
\end{enumerate}
This allows to write the functional
\begin{equation}
  F(\epsilon_0, T) =  (1-p_{\loss})^{n} \frac{\Abs{\gamma / \pi}}{T^{1/m}},
  \qquad n = 10,
  \quad  m = 1, 2\,.
  \label{eq:hol_simplex_F}
\end{equation}
which takes its maximum value if all three requirements are fulfilled. To
weight the three conditions, the exponents $l$ and $m$ are used. Adiabaticity is
the most difficult to achieve, and is weighted most strongly by setting $n=10$.
For the gate duration, $m=2$ has proved the most effective choice, although
$m=1$ can be used to put more emphasis on a shorter gate duration.

\begin{table}[tb]
  \centering
  \sisetup{round-mode=places,round-precision=2}
  \begin{tabular}{cccSSScc}
  \toprule
  {$\delta$} & {$T$} & {$\epsilon_0$} & \multirow{2}{*}{pop.\ loss}  &  \multirow{2}{*}{$\Avg{i,j}_{\max}$} & \multirow{2}{*}{$\Avg{n}_{\max}$} & \multirow{2}{*}{$n_q$} &   \multirow{2}{*}{$n_c$}       \\
  {   [MHz]} & {[ns]}& {[MHz]} & &  &  &   &          \\
  \midrule
  \multicolumn{8}{c}{$\omega_d < \omega_c = \SI{6.0}{GHz} < \omega_1, \omega_2$} \\
  % dynamic_scan2/params1.dat
  \midrule                                                                                                           %           1-C     1-chi [pi]
  -20 &   423  &    ~19 & \num{1.321366E-01} &   \num{2.338191}  &   \num{21.749016}  &       12 &        50 \\      %  1.587029E-01   3.764600E-01
  -30 &   214  &    ~44 & \num{4.529893E-03} &   \num{1.150207}  &    \num{4.811057}  &        9 &        40 \\      %  0.000000E+00   0.000000E+00
  -40 &   165  &    121 & \num{7.501945E-03} &   \num{1.362307}  &    \num{7.987227}  &       10 &        40 \\      %  0.000000E+00   0.000000E+00     d40b: m=2
  -40*&   146  &    127 & \num{1.090988E-02} &   \num{1.401790}  &    \num{8.213258}  &       10 &        60 \\      %  0.000000E+00   0.000000E+00     d40a: m=1
  -50 &   311  &    144 & \num{5.465042E-03} &   \num{1.146033}  &    \num{5.204558}  &        5 &        40 \\      %  8.072068E-04   0.000000E+00
  \midrule
  \multicolumn{8}{c}{$\omega_d > \omega_c = \SI{8.11}{GHz} > \omega_1, \omega_2$} \\
  \midrule                                                                                                           %            1-C     1-chi [pi]
  ~20 &   291  &    ~67 & \num{2.302942E-02} &   \num{1.014221}  &    \num{9.915150}  &       5  &       40  \\      %   0.000000E+00   0.000000E+00
  ~30 &   308  &    174 & \num{9.689395E-03} &   \num{1.020742}  &   \num{18.242242}  &       5  &       50  \\      %   0.000000E+00   6.000000E-05
  ~40 &   220  &    372 & \num{1.096306E-02} &   \num{1.046223}  &   \num{36.311382}  &       6  &       70  \\      %   0.000000E+00   1.218000E-02    d40b: m=2
  ~40*&   125  &    471 & \num{5.942816E-02} &   \num{1.099855}  &   \num{64.824770}  &       6  &      100  \\      %   1.440705E-05   6.000000E-05    d40a: m=1
  ~50 &   168  &    670 & \num{1.272611E-02} &   \num{1.092632}  &   \num{65.635537}  &       6  &      110  \\      %   0.000000E+00   1.300000E-04g
  \bottomrule
  \end{tabular}
  \caption{Simplex optimization of pulse time $T$  and peak pulse amplitude
  $\epsilon_0$ for the realization of a holonomic phasegate, for different
  pulse-cavity detunings $\delta \equiv \omega_d - \omega_c$ and for two
  different cavity frequencies
  (left and right of the qubits, cf.~the two panels in Fig.~\ref{fig:holonomic_entanglement}).
  All parameters are listed in Table~\ref{tab:hol_params}.
  The figure of merit for the optimization is Eq.~\eqref{eq:hol_simplex_F} with
  $n=10$ and $m=2$. We list the loss of population from the logical subspace at
  final time $T$, the peak expectation values $\Avg{i,j}_{\max}$,
  $\Avg{n}_{\max}$ of the qubit and cavity excitation, and the number $n_q$,
  $n_c$ of qubit and cavity levels that must be included for numerical
  convergence. In the optimizations labeled by an asterisk, the figure of merit
  is taken with $m=1$, putting more emphasis on short gate durations.
  }
  \label{tab:holonomic_simplex_results}
\end{table}
The results of the simplex optimization are shown in
Table~\ref{tab:holonomic_simplex_results}. The obtained gates are comparable to
those obtained from systematic variation in
Fig.~\ref{fig:holonomic_entanglement}
and Table~\ref{tab:hol_entanglement_tab}. Nearly all of the gates obtained by
the simplex optimization are now exact perfect entanglers.

For $\omega_c = \SI{6.00}{GHz}$ and $\omega_d = \omega_c - \SI{40}{MHz}$, the
optimization yielded considerable
improvement of the quality of the gate, compared to point 1 in
Table~\ref{tab:hol_entanglement_tab}. The gate duration is slightly shorter,
while population loss has decreased and entanglement has increased.
For $\omega_c = \SI{8.11}{GHz}$ the results are less positive;
for $\omega_d = \omega_c + \SI{40}{MHz}$, there is in fact a larger loss of
population than for point 3 in Table~\ref{tab:hol_entanglement_tab} (although
the gate is now a perfect entangler). This illustrates a shortcoming of the
simplex algorithm, which is sensitive to local traps. A slight change in
parameters can cause a sudden non-adiabatic loss of population, producing
a local minimum for the optimization functional.

The results for varying values of $\delta=\omega_d-\omega_c$ confirm that
when placing the drive frequency closer to the cavity, lower pulse amplitude
produce higher entanglement, but also, it is easier to lose population from the
logical subspace. For $\delta = \SI{-20}{MHz}$, the gate fails completely:
there is large excitation of the qubit, 13\% loss of population, and the
concurrence only reaches 0.84. This explains the somewhat counter-intuitive
long gate durations for both small and large values of $\delta$. For small
values, the loss of adiabaticity can be countered by making the pulse longer,
and thus change more slowly. For large values, it is more difficult to obtain
entanglement, so that the pulse needs to be active for a longer duration before
a concurrence of 1.0 is reached. For $\delta = \SI{-50}{MHz}$ ,
$1-C = \num{8e-4}$. The best result is obtained for $\delta = \SI{-40}{MHz}$
(shortest gate duration) or $\delta = \SI{-30}{MHz}$ (least loss of population).
For $\omega_c = \SI{8.11}{GHz}$, moving the drive frequency closer to the cavity
is less detrimental; a perfect entangler is still achieved, albeit at
a relatively high loss of population.

Shorter gates can be obtained by putting more emphasis on the minimization of
$T$, as was done for the two optimizations labeled with an asterisk. However,
this is at the cost of larger pulse amplitudes and increased loss of population.
Also for $\omega_c=\SI{8.11}{GHz}$ and $\delta=\SI{40}{GHz}$, the gate falls
short of being an exact perfect entangler, $1-C = \num{1e-5}$.

Lastly, the results in Table~\ref{tab:holonomic_simplex_results} confirm the
observations of Fig.~\ref{fig:holonomic_entanglement} and
Table~\ref{tab:hol_entanglement_tab}, that choosing
$\omega_c > \omega_{1}, \omega_2$ require far stronger pulses, causing very
large cavity excitation. On the other hand, far less qubit levels need to be
taken into account.


\section[Optimization of Transmon Quantum Gates]{Optimization of Transmon Quantum Gates with Krotov's Method}

\subsection{Optimization for CPHASE and CNOT}
\label{subsec:tm_oct_direct}

\begin{table}[tb]
  \centering
  \begin{tabular}{llrl}
  \toprule
  cavity     frequency  &  $\omega_c$            & 8.3   &GHz \\
  left qubit frequency  &  $\omega_1$            & 6.5   &GHz \\
  right qubit frequency &  $\omega_2$            & 6.6   &GHz \\
  qubit anharmonicity   &  $\alpha_1, \alpha_2$  & -300  &MHz \\
  qubit-cavity coupling &  $g_1, g_2$            & 100   &MHz \\
  \bottomrule
  \end{tabular}
  \caption{Parameters for the transmon Hamiltonian
  Eq.~\eqref{eq:tm_fullham}~\cite{GirvinTransmonParams}
  }
  \label{tab:tm2013_params}
\end{table}

In order to explore the potential for more advanced optimal control of two-qubit
quantum gates for two transmons with a shared transmission line, we use Krotov's
method as presented in chapter~\ref{chap:numerics}
to optimize the full system Hamiltonian in Eq.~\eqref{eq:tm_fullham} for the
parameters given in Table~\ref{tab:tm2013_params}. The qubit parameters are
assumed fixed, and the control is entirely through the microwave field on the
transmission line resonator. This distinguishes the approach from recent
applications of similar control techniques to flux-tunable qubits where the
control parameter was the qubit-cavity detuning~\cite{EggerSST2014}.

The optimization functional to be minimized is
\begin{equation}
  J = 1 - \frac{1}{16} \Abs{\sum_{i=1,4}
      \Braket{\phi_k | \Op{O}^{\dagger} \Op{U}(T,0;\epsilon) | \phi_k}
      }^2 + \frac{\lambda_a}{S(t)} \int_{0}^{T} (\Delta \epsilon(t) )^2 \dd t\,,
  \label{eq:tm_J}
\end{equation}
where $\{\Ket{\phi_k}\}$ are the states \{\Ket{000}, \Ket{010}, \Ket{100},
\Ket{110}\} that span the logical two-qubit subspace. The three quantum numbers
indicate the excitation of the first qubit, second qubit, and the cavity.
$\Op{U}(T,0;\epsilon)$ is the time evolution operator for the propagation under
the pulse $\epsilon(t)$, $\Op{O}$ is the target quantum gate. We optimize both
for $\Op{O} = \text{CPHASE}$ and $\Op{O}= \text{CNOT}$.
The shape function $S(t)$ ensures a smooth switch-on and switch-off, $\lambda_a$
is an arbitrary scaling parameter. Starting from a guess pulse driving at the
frequency of the second qubit and small amplitude, Krotov's method iteratively
updates the pulse according to
\begin{equation}
  \Delta\epsilon(t)
    =
  \frac{S(t)}{\lambda_a} \Im \left[
    \sum_{k=1}^{4}
    \Bigg\langle
      \chi_k^{(i)}(t)
    \Bigg\vert
      \Bigg(\Op{a} + \Op{a}^{\dagger} \Bigg)
    \Bigg\vert
      \phi_k^{(i+1)}(t)
    \Bigg\rangle
  \right]\,,
\end{equation}
cf.~Eq.~\eqref{eq:krotov_first_order_update}, in order to minimize
Eq.~\eqref{eq:tm_J}. The states $\{\Ket{\phi_k}\}$ are forward propagated with the
updated pulse $\epsilon^{(i+1)}(t)$, while the states $\{\Ket{\chi_k}\}$ are
backward-propagated with the pulse $\epsilon^{(i)}(t)$ of the current iteration,
with the boundary condition given by Eq.~\eqref{eq:chi_JTsm}.

\begin{figure}[p]
  \centering
  \includegraphics{transmon2013_gate_error_cphase}
  \caption{Results for the unconstrained optimization of a CPHASE gate for
  different gate durations. The panels show the peak amplitude of the optimized
  pulse (a), the peak expectation value of the qubit number operator (b) and the
  cavity number operator (c).
  Panel~(d) shows the gate error without dissipation, and with spontaneous decay
  of the cavity for a lifetime of $\tau=\SI{100}{\micro\second}$ and
  $\tau=\SI{25}{\micro\second}$.
  For the non-dissipative case, the loss of population from the logical subspace
  at final time $T$ is shown.
  }
  \label{fig:transmon2013_gate_error_cphase}
\end{figure}
\begin{figure}[p]
  \centering
  \includegraphics{transmon2013_gate_error_cnot}
  \caption{Results for the unconstrained optimization of a CNOT gate for
  different gate durations, cf.~Fig.~\ref{fig:transmon2013_gate_error_cphase}.
  }
  \label{fig:transmon2013_gate_error_cnot}
\end{figure}
The results of the optimization for a CPHASE gate with pulse durations
between \SI{100}{ns} and \SI{1000}{ns} are shown in
Fig.~\ref{fig:transmon2013_gate_error_cphase}.
% pulse amplitude
The gate error $1-F_{\avg}$ with $F_{\avg}$ according to Eq.~\eqref{eq:Favg} is
shown in blue in panel~(d). The gate error is limited to values $> \num{1e-3}$,
i.e., slightly above the quantum error correction limit. For gate durations $\ge
\SI{200}{ns}$, the gate error is entirely due to loss of population from the
logical subspace at time $T$, evaluated according to Eq.~\eqref{eq:pop_loss}.
It is shown as the black dashed line in panel~(d) of
Fig.~\ref{fig:transmon2013_gate_error_cphase}.
For $T=\SI{100}{ns}$, the gate error is not simply due to the inability to bring
return the population to the logical subspace: for short gate durations, there
is not sufficient time to exploit the cavity-mediated qubit interaction to
generate the necessary entanglement for the desired gate.
Panels (b) and (c) show the peak expectation value of the qubit and cavity
population during the propagation of
any of the logical basis states. For the qubit, it shows that population is is
mostly within the first three levels of the transmon. The system would
\emph{not} accurately described by truncating it to two levels, however. In
fact, the number of levels that must be included in the simulation for full
convergence is a multiple of the mean excitation number. The cavity is strongly
driven to large excitations, which is the cause for the limited gate fidelity.
The gate error in panel~(d) and the peak cavity excitation in panel~(b) are
strongly correlated. In cases where the cavity population reaches large values
during the implementation of the gate, it becomes extremely hard to bring the
cavity to the ground state at final time $T$. This results in the loss of
population from the logical subspace, and limits the gate fidelity. The large
cavity population also makes the gate sensitive to spontaneous decay of the
cavity levels. The yellow and red curve in panel~(d) show the gate error if the
cavity decays weakly, with a lifetime of $\tau = \SI{100}{\micro\second}$, and
more strongly, with a lifetime of $\tau = \SI{25}{\micro\second}$, exploring the
realistic range for current transmission line resonator. In both cases, the gate
error increases to values of \num{1e-2} or worse.

The system of two coupled transmons can implement not only a CPHASE gate, but
but a wide range of other entangling quantum gates. In
Fig.~\ref{fig:transmon2013_gate_error_cnot}, the results for an optimization
towards a CNOT gate for a few gate durations is shown. The pulse amplitudes in
panel~(a) are similar to those obtained in the CPHASE optimization. Also, the
reachable gate error is on the order of $10^{-3}$. The qubit excitation again
shows significant population of the third level of the anharmonic transmon
ladder. For short gate durations, the peak cavity population is significantly
higher than for CPHASE, resulting in considerably worse fidelities at
\SI{200}{ns} and \SI{300}{ns}. A quantum speed limit, where the gate error is
not determined just by the loss of population from the logical subspace alone is
again found for $T<\SI{200}{ns}$.

\begin{figure}[tp]
  \centering
  \includegraphics{tm2013_spectra}
  \caption{Spectra of select pulses resulting from the optimization for
  CPHASE~(top) and CNOT~(bottom), for a gate duration of $T=\SI{200}{ns}$~(left)
  and $T=\SI{1000}{ns}$~(right),
  cf.~Fig~\ref{fig:transmon2013_gate_error_cphase}
  and Fig.~\ref{fig:transmon2013_gate_error_cnot}. The bare qubit and cavity
  frequencies are indicated by the solid black vertical lines, and the qubit
  transitions $\Ket{1} \rightarrow \Ket{2}$ by the dashed lines.
  }
  \label{fig:tm2013_spectra}
\end{figure}
While the optimization yields complicated dynamics that make a detailed
understanding of the gate mechanisms far from trivial, it is instructive to
consider the spectra of the optimized pulses. For the optimized pulses of CPHASE
and CNOT, for each gate duration of $T=\SI{200}{ns}$ and $T=\SI{1000}{ns}$, the
spectra are shown in Fig.~\ref{fig:tm2013_spectra}. Intuitively, for
$T=\SI{200}{ns}$, the implementation of the gate is much harder than for
$T=\SI{1000}{ns}$ and requires greater pulse amplitude. Consequently, there are
more spectral components and broader peaks in the spectra for the short gate
durations. For CPHASE at $T=\SI{200}{ns}$, the spectrum is dominated by the
frequencies corresponding to the anharmonic transition $\Ket{1} \rightarrow
\Ket{2}$ for the both left and right qubit. This matches the peak qubit
excitation of $\Avg{i}=2$ in panel~(b) of
Fig.~\ref{fig:transmon2013_gate_error_cphase}.
The two qubit frequencies and the cavity frequencies are also present, together
with high-frequency sideband transitions. For example, the first two of these
transitions at $\SI{10.0}{GHz}$ and $\SI{10.1}{GHz}$ correspond to the
transitions
\begin{equation}
  \Ket{1}_q \otimes \Ket{0}_c
  \; \longleftrightarrow \;
  \Ket{0}_q \otimes \Ket{2}_c\,,
  \qquad q = 1,2
  \label{eq:tm_spec_sideband}
\end{equation}
that transfers two cavity excitation into one qubit excitation of the left or
right qubit, or vice versa.
For $T=\SI{1000}{ns}$, the spectrum for the CPHASE gate is relatively close to
the spectrum of the original guess pulse, driving mainly on the frequency of the
second qubit. However, the cavity frequencies and the sideband transition of
Eq.~\eqref{eq:tm_spec_sideband} is also present. The absence of the frequency
driving the $\Ket{1} \leftrightarrow \Ket{2}$ transition of the qubits corresponds
to the peak qubit population of $\Avg{i}=1$ for $T=\SI{1000}{ns}$ in panel~(c)
of Fig.~\ref{fig:transmon2013_gate_error_cphase}.
The optimized pulses for a CNOT gate are roughly similar to those for CPHASE.
However, the combination of driving the right qubit frequency and the left
anharmonic transition appear relevant, as they are present both for short and
long gate duration. Counterintuitively, the significant drive of the cavity
frequency for CNOT at $T=\SI{1000}{ns}$ does \emph{not} correspond to a large
peak cavity population in panel~(c) of
Fig.~\ref{fig:transmon2013_gate_error_cnot}. This illustrates that the dynamics
of the optimized pulses are very non-trivial and that the high-frequency
sideband frequencies play a significant role. For example, it is conceivable
that the transition in Eq.~\eqref{eq:tm_spec_sideband} acts in conjunction with
the cavity drive to manipulate the qubit state.
Lastly, the spectra also illustrate the Stark shift of the cavity level,
respectively the Lamb shifts on the qubit levels that is induced by the
qubit-cavity interaction: the qubit frequencies are shifted very slightly to the
left of the bare frequency, whereas the cavity frequency is very slightly
shifted right.

% XXX punish cavity population, may lead to completely different dynamics


\subsection{Optimization for a Holonomic Phasegate}

Given the limited success of an unconstrained direct optimization for a CPHASE
or a CNOT gate, we may attempt to set up the optimization for a specific gate
mechanism. Therefore, we return to the holonomic phasegate
and attempt to use Krotov's method to improve upon the results of
section~\ref{sec:holonomic}.
The first term in Eq.~\eqref{eq:tm_J} must be replaced with a final
time functional that reflects the objectives of the holonomic mechanism. While
the functional in Eq.~\eqref{eq:hol_simplex_F} used in the simplex
optimization is not suitable for Krotov's method, since it would be extremely
hard to evaluate the boundary condition for the backward-propagated states,
Eq.~\eqref{eq:chi_boundary}, it is relatively straightforward to derive a more
suitable functional. The two conditions that we consider are:
\begin{enumerate}
  \item The gate should be diagonal; every one of the logical eigenstates should
  be mapped onto itself, up to a Berry phase.
  \item The Berry phases should yield maximum entanglement, $\gamma = \pi$,
  according to Eq.~\eqref{eq:entangling_phase}.
\end{enumerate}

The two conditions enter as two terms in the final time functional,
\begin{equation}
  J_T = J_{ss} + w_{\gamma} J_{\gamma}\,,
\end{equation}
with a factor $w_{\gamma}$ to weight their relative importance.
With
\begin{equation}
  \tau_{00} = \Braket{00|\Op{U}| 00}\,,
\end{equation}
and equivalently for $\Ket{01}$, $\Ket{10}$, and $\Ket{11}$, the first term
takes the form
\begin{equation}
  J_{ss} = 4 -\Abs{\tau_{00}}^2 -\Abs{\tau_{01}}^2
             -\Abs{\tau_{10}}^2 -\Abs{\tau_{11}}^2\,,
\end{equation}
i.e.\ four simultaneous state-to-state transitions~\cite{PalaoPRA03}.
As shown in appendix~\ref{AppendixHol}, the second term can be written as
\begin{equation}
  J_{\gamma} = 2 + \tau_{00}\tau_{01}^{*}\tau_{10}^{*}\tau_{11}
                 + \tau_{00}^{*}\tau_{01}\tau_{10}\tau_{11}^{*}\,.
\end{equation}
The boundary condition for the backward propagation may also be found in
appendix~\ref{AppendixHol}.

%%% XXX second order?

\begin{figure}[tb]
  \centering
  \includegraphics{hol_oct_success}
  \caption{Success of optimization for a holonomic phasegate using Krotov's
  method for two parameter sets, cf.~Fig.~\ref{fig:holonomic_entanglement}.
  All remaining parameters are given in Table~\ref{tab:hol_params}.
  The success is evaluated in terms of the loss of population from the logical
  subspace at final time $T$, and the concurrence $C$ of the obtained gate
  $\Op{U}$.
  }
  \label{fig:hol_oct_success}
\end{figure}
The results of an optimization for the two parameter sets corresponding to the
solid orange curve in the top panel and the dashed red curve in the bottom panel
of Fig.~\ref{fig:holonomic_entanglement}, used also in the simplex optimization
in section~\ref{subsec:hol_simplex}, is shown in
Fig.~\ref{fig:hol_oct_success}. The gate durations are varied between \SI{40}{ns}
and \SI{150}{ns}, the approximate shortest gate duration obtained in the simplex
optimization. A fully entangling gate can be achieved for $T>\SI{90}{ns}$
($\omega_c = \SI{6.0}{GHz}$), respectively $T > \SI{120}{ns}$ ($\omega_c
= \SI{8.1}{GHz}$). However, there is considerable loss of population from the
logical subspace, comparable to the results obtained by simplex optimization,
cf.~\ref{tab:holonomic_simplex_results}. As observed both in
section~\ref{subsec:hol_simplex} and section~\ref{subsec:tm_oct_direct},
it is the inability to return the state entirely to the logical subspace that
fundamentally limits the optimization success to values above the error
correction threshold.


\begin{figure}[p]
  \centering
  \includegraphics{hol_oct_120left_popdyn}
  \caption{Population dynamics for logical eigenstate \Ket{00} under the
  optimized pulse for $T=\SI{120}{ns}$ and $\omega_c = \SI{6.0}{GHZ}$, cf.\ the
  corresponding point in the solid red line of Fig.~\ref{fig:hol_oct_success}.
  The figure follows the same conventions as Fig.~\ref{fig:adiabatic_popdyn}.
  }
  \label{fig:hol_oct_120left_popdyn}
\end{figure}
The population dynamics for $T=\SI{120}{ns}$ and $\omega_c = 6.0$, cf.~the
respective point on the solid red line in Fig.~\ref{fig:hol_oct_success} is
shown in Fig.~\ref{fig:hol_oct_120left_popdyn} and should be compared to
Fig.~\ref{fig:adiabatic_popdyn}. The dynamics are considerably non-adiabatic,
showing direct driving of both the qubits and the cavity. However, the highly
symmetric shape of the pulse, the population dynamics and the excitation of both
qubits and cavity correspond to what one would hope to achieve by applying
optimal control to the holonomic gate. Since adiabaticity cannot be achieved on
the given time-scale, optimal control corrects the non-adiabatic effects by
ensuring that for a symmetric pulse, any jumps over an avoided crossing that
happen during the switch-on also happen during the switch-off. However, the
optimized pulse is only partially successful in this endeavor, as evidenced by
the non-zero excitation of the left qubit at final time $T$, cf.\ panel~(c).

\begin{figure}[p]
  \centering
  \includegraphics{hol_oct_spectra}
  \caption{Spectra of  pulses for $T=\SI{120}{ns}$
  The top panel corresponds to the pulse shown in
  panel~(d) of Fig.~\ref{fig:hol_oct_120left_popdyn}, for
  $\omega_d < \omega_c = \SI{6.0}{GHZ} < \omega_1, \omega_2$
  (cf.\ yellow curves in Fig.~\ref{fig:hol_oct_success})
  The bottom panel  shows the spectrum of the optimized pulse for
  $\omega_d > \omega_c = \SI{8.1}{GHZ} > \omega_1, \omega_2$
  (cf.\ red curves in Fig.~\ref{fig:hol_oct_success}). The spectrum is shown in
  the rotating frame at $\omega_d$, corresponding to zero in the spectrum.
  The bare frequencies of the qubit and the cavity are indicated by vertical
  dashed lines. The central peaks extend to a value $\approx 40$.
  }
  \label{fig:hol_oct_spectra}
\end{figure}
The spectrum of the optimized pulse in panel~(d) of
Fig.~\ref{fig:hol_oct_120left_popdyn} is shown in the top panel of
Fig.~\ref{fig:hol_oct_spectra}, together with the spectrum of the optimized
pulse for $\omega_c = \SI{8.1}{GHz}$, also at $T=\SI{120}{ns}$ (bottom panel).
In both cases, the optimized pulse does not only contain the central frequency
for the off-resonant drive at $\omega_d$, but also frequencies that resonantly
drive the cavity and both qubits. In the top panel, these are shifted
significantly from their bare values, whereas for the bottom panel, the shifts
are not discernible. This corresponds to the results of
Fig.~\ref{fig:holonomic_entanglement}. The mirrored frequencies are due to the
pulse being complex-valued in the RWA and correspond to the time-dependence of
the complex phase.

For all the optimizations presented in this section, the CPHASE, CNOT, and the
holonomic gate, the fidelities of the optimized gates are limited by loss of
population from the logical subspace. To achieve high fidelity gates, more
advanced approaches will be required. First, it appears that the Krotov-gradient
is not sufficiently sensitive to the small but still significant loss of
population. Switching the optimization to a method that uses second order
information as soon as convergence stagnates, such as the GRAPE/LBFGS method
presented in chapter~\ref{chap:numerics} provides a possible solution. More
fundamentally, however, it might be more promising to avoid the problematic and
unwanted excitation of high qubit and cavity levels altogether, by placing
constraints on the optimization. These could take the form of state-dependent
constraints~\cite{PalaoPRA08} to avoid high excitations, or using spectral
constraints~\cite{JosePRA13, ReichJMO2014} to enforce simpler spectra in the
optimized pulses or to remove e.g.\ cavity driving from the optimization of the
CPHASE and CNOT, and resonant driving from the optimization of the qubit.
A further option is to reduce the complexity of the model by considering an
effective model of the cavity-mediated qubit-qubit interaction, as discussed in
section~\ref{sec:tm_effective_description}. This is the approach we take in
chapters~\ref{chap:pe} and~\ref{chap:3states}. However, we have found that an
unconstrained optimization in the effective model will generally cause the
correspondence between the full and the effective model to break down. Again,
this issue might be addressed by placing constraints on the optimization.

Limiting the excitation of the qubits and the cavity will also improve the
robustness of the implemented gates with respect to decoherence,
cf.~Figs.~\ref{fig:transmon2013_gate_error_cphase}
and~\ref{fig:transmon2013_gate_error_cnot}, as the lower levels are least
affected by spontaneous decay and dephasing. Going further,
once the optimization of the transmon qubits has been brought to yield
fidelities below the quantum error correction limit, the techniques of
chapter~\ref{chap:robust} and the following chapters~\ref{chap:pe}
and~\ref{chap:3states} may be used to address the ultimate goal of designing
quantum gates for the transmon that are as robust as possible against the
effects of decoherence.

% \section{Dissipation}
% [1] A. A. Houck, H. E. Tureci, and J. Koch, Nat. Phys. 8, 292 (2012).
% (overview of decoherence)

