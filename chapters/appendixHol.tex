\chapter{Optimization Functional for a Holonomic Phasegate}
\label{AppendixHol}

We formulate a functional for the holonomic phasegate of
chapter~\ref{chap:transmon}, i.e.\ an  arbitrary diagonal perfect entangler. The
functional can be expressed both in Hilbert space or Liouville space. We derive the
boundary condition for the backward propagation in Krotov's method,
Eq.~\eqref{eq:chi_boundary}.

\section{Hilbert Space Functional}

The functional is split in two parts, corresponding to the two requirements that
the gate must be diagonal, and that it should be a perfect entangler,
\begin{equation}
  J_T = J_{ss} + w_{\gamma} J_{\gamma}.
\end{equation}
The two terms may be weighted with $w_{\gamma}$.

The first part of the functional states that every logical eigenstate should be
mapped onto itself, irrespective of any phase.
\begin{equation}
\begin{split}
  J_{ss}&= 4 - \Abs{\Braket{00 | \Op{U} | 00}}^2
             - \Abs{\Braket{01 | \Op{U} | 01}}^2
           + \\ & \quad
             - \Abs{\Braket{10 | \Op{U} | 10}}^2
             - \Abs{\Braket{11 | \Op{U} | 11}}^2
\end{split}
\end{equation}
$J_{ss}$ takes values in $[0,4]$ and optimizes for an arbitrary diagonal gate.

The second requirement of having a perfect entangler is fulfilled if the
non-local phase, cf.~Eq.~\eqref{eq:entangling_phase} reaches a value of $\pi$,
$\gamma = \phi_{00} - \phi_{10} - \phi_{01} + \phi_{11} = \pi$
where $\phi_{00}$ is the phase obtained by the state \Ket{00} and equivalently for
the other states of the logical basis.

We find
\begin{equation}
  \Braket{00|\Op{U}|00}
  \Braket{01|\Op{U}\daggered|01}
  \Braket{10|\Op{U}\daggered|10}
  \Braket{11|\Op{U}|11}
  = e^{i \gamma}\,.
\end{equation}
Together with
\begin{equation}
  2 \cos(\gamma) = e^{i \gamma} + e^{-i \gamma}\,,
  \label{eq:2cos}
\end{equation}
and the condition that $\cos(\gamma) = -1$ for $\gamma = \pi$,
this determines the functional $J_{\gamma}$,
\begin{equation}
\begin{split}
  J_{\gamma} &= 2 +
    \Braket{00|\Op{U}|00}
    \Braket{01|\Op{U}\daggered|01}
    \Braket{10|\Op{U}\daggered|10}
    \Braket{11|\Op{U}|11} \\
    & \quad \quad
    +
    \Braket{11|\Op{U}\daggered|11}
    \Braket{10|\Op{U}|10}
    \Braket{01|\Op{U}|01}
    \Braket{00|\Op{U}\daggered|00}\,.
\end{split}
\end{equation}
$J_{\gamma}$ takes values in $[0, 4]$.


In order to derive the boundary condition for the backward propagation of
Krotov's method, we use the abbreviation
\begin{equation}
 \tau_{00} \equiv \Braket{00|\Op{U}|00},
\end{equation}
and equivalently for $\Ket{01}$, $\Ket{10}$, and $\Ket{11}$.
The two terms of the functional are then
\begin{subequations}
\begin{align}
  J_{ss} &= 4 - \tau_{00}\tau_{00}^* - \tau_{01}\tau_{01}^*
          - \tau_{10}\tau_{10}^* - \tau_{11}\tau_{11}^*\,,\\
  J_{\gamma} &= 2 + \tau_{00} \tau_{01}^* \tau_{10}^* \tau_{11}
                 + \tau_{00}^* \tau_{01} \tau_{10} \tau_{11}^*\,.
\end{align}
\end{subequations}

In order to calculate the boundary conditions for the backward propagated
states,
\begin{equation}
  \Ket{\chi(T)}_i = - \frac{\partial J_T}{\partial \Bra{i(T)}};
  \quad
  \Ket{i(T)} \equiv \Op{U} \Ket{i};
  \quad
  i = \left\{ 00, 01, 10, 11 \right\}\,,
  \label{eq:eta_hol1}
\end{equation}
we use
\begin{equation}
  \frac{\partial \tau_i}{\partial \Bra{j(T)}} = 0,
  \qquad
  \frac{\partial \tau_i^*}{\partial \Bra{j(T)}} = \delta_{ij} \Ket{i};
  \qquad
  i,j = \left\{ 00, 01, 10, 11 \right\}.
\end{equation}
This yields
\begin{subequations}
\begin{align}
  \Ket{\chi(T)}_{00}
  &= \left( \tau_{00} - w_{\gamma}\tau_{01}\tau_{10}\tau_{11}^* \right)
     \Ket{00}\,, \\
  \Ket{\chi(T)}_{01}
  &= \left( \tau_{01} - w_{\gamma}\tau_{00}\tau_{10}^*\tau_{11} \right)
     \Ket{01}\,, \\
  \Ket{\chi(T)}_{11}
  &= \left( \tau_{11} - w_{\gamma}\tau_{00}^*\tau_{01}\tau_{10} \right)
     \Ket{11}\,, \\
  \Ket{\chi(T)}_{10}
  &= \left( \tau_{10} - w_{\gamma}\tau_{00}\tau_{01}^*\tau_{11} \right)
     \Ket{10}\,.
\end{align}
\end{subequations}

\section{Liouville Space Functional}

Based on the work presented in chapter~\ref{chap:3states}, an optimization
functional for a holonomic gate in Liouville space can be formulated.
The optimization success is tracked by the three states $\Op{\rho}_1$,
$\Op{\rho}_2$, $\Op{\rho}_3$ defined in Eq.~\eqref{eq:3st_rhos}.
The optimization functional contains one term for each of these matrices,
\begin{equation}
  J_T = w_1 J_1 + w_2 J_2 + w_3 J_3 \,.
\end{equation}
with weights $w_1$, $w_2$, $w_3$.

When the optimization target is reached,
\begin{equation}
  \Op{U} \Op{\rho}_1 \Op{U}\daggered = \Op{\rho}_1\,,
  \qquad
  \Op{U} \Op{\rho}_3 \Op{U}\daggered = \Op{\rho}_3\,.
\end{equation}
The state $\Op{\rho}_3$ ensures that no population is lost from the logical
subspace, $\Op{\rho}_1$ optimizes for a diagonal gate.
Therefore, we choose
\begin{equation}
\begin{split}
  J_{1,3} &= 1-\frac{1}{\trace\left[\Op{\rho}_{1,3}^2\right]}
            \Re\left[
            \LBraket{\Op{\rho}_{1,3} | \DynMap\Op{\rho}_{1,3}}
            \right]
          \\
          &= 1-\frac{1}{\trace\left[\Op{\rho}_{1,3}^2\right]}
            \frac{1}{2} \left[
              \LBraket{\Op{\rho}_{1,3} | \DynMap\Op{\rho}_{1,3}}
             +\LBraket{\DynMap\Op{\rho}_{1,3} | \Op{\rho}_{1,3} }
            \right]\,,
\end{split}
\end{equation}
using $\DynMap$ to indicate the dynamical map, and a braket-notation for the
Hilbert-Schmidt product,
\begin{equation}
  \LBraket{\Op{A} |\Op{B}} \equiv \trace\left[\Op{A}\daggered \Op{B}\right]\,.
\end{equation}

The remaining state
$\Op{\rho}_2$ is used to track the relative phases and to optimize the
entanglement.
If $\Op{U}$ is diagonal as in Eq.~\eqref{eq:general_diagonal_gate}, then
\begin{equation}
  \Op{U} \Op{\rho}_2 \Op{U}\daggered
  = \sum_{i,j} e^{i(\phi_i-\phi_j)} \Ket{i}\!\Bra{j};
  \quad
  i,j = \left\{ 00, 01, 10, 11 \right\}.
\end{equation}
With $
  \gamma =  \left( \phi_{00} - \phi_{01} \right)
          + \left( \phi_{00} - \phi_{10} \right)
          + \left( \phi_{11} - \phi_{00} \right)
$ and
$\Op{P}_{i,j} \equiv \Ket{i}\!\Bra{j}$, we find
\begin{align}
  e^{i \gamma}  &= \LBraket{\Op{P}_{00,01} | \Op{U}\Op{\rho}_2\Op{U}\daggered}
                 \LBraket{\Op{P}_{00,10} | \Op{U}\Op{\rho}_2\Op{U}\daggered}
                 \LBraket{\Op{P}_{11,00} | \Op{U}\Op{\rho}_2\Op{U}\daggered}\\
  e^{-i \gamma} &= \LBraket{\Op{U}\Op{\rho}_2\Op{U}\daggered | \Op{P}_{00,01}}
                 \LBraket{\Op{U}\Op{\rho}_2\Op{U}\daggered | \Op{P}_{00,10}}
                 \LBraket{\Op{U}\Op{\rho}_2\Op{U}\daggered | \Op{P}_{11,00}}
\end{align}
With Eq.~(\ref{eq:2cos}) and the generalization
$\Op{U}\Op{\rho}_2\Op{U}\daggered \rightarrow \DynMap\Op{\rho}_2$, we obtain
the functional for $\Op{\rho}_2$,
\begin{equation}
\begin{split}
  J_2 &= 2 + \LBraket{\Op{P}_{00,01} | \DynMap\Op{\rho}_2}
             \LBraket{\Op{P}_{00,10} | \DynMap\Op{\rho}_2}
             \LBraket{\Op{P}_{11,00} | \DynMap\Op{\rho}_2} \\
      & \qquad
          + \LBraket{\DynMap\Op{\rho}_2 | \Op{P}_{00,01}}
            \LBraket{\DynMap\Op{\rho}_2 | \Op{P}_{00,10}}
            \LBraket{\DynMap\Op{\rho}_2 | \Op{P}_{11,00}}\,.
\end{split}
\end{equation}
$J_1$ and $J_3$ take values in $[0,1]$, whereas $J_2$ takes values in $[0,4]$.


Equivalently to Eq.~(\ref{eq:eta_hol1}), the boundary condition for the
backward-propagated states in Krotov's method is
\begin{equation}
  \Op{\sigma}_i(T) = - \frac{\partial J_T}
                         {\partial \Lbra{\DynMap\Op{\rho}_i}};
  \quad
  i = \left\{ 1,2,3\right\}\,,
\end{equation}
which yields
\begin{align}
  \Op{\sigma}_1(T)
   & = \frac{w_1}{2} \Op{\rho}_1 \,,\\
  \Op{\sigma}_2(T)
   & =
     - \LBraket{\DynMap\Op{\rho}_2 | \Op{P}_{00,10}}
       \LBraket{\DynMap\Op{\rho}_2 | \Op{P}_{11,00}}
       \, w_2 \, \Op{P}_{00,01}
       \nonumber\\
     & \quad
     - \LBraket{\DynMap\Op{\rho}_2 | \Op{P}_{00,01}}
       \LBraket{\DynMap\Op{\rho}_2 | \Op{P}_{11,00}}
       \, w_2 \, \Op{P}_{00,10}
       \\
     & \quad
     - \LBraket{\DynMap\Op{\rho}_2 | \Op{P}_{00,10}}
       \LBraket{\DynMap\Op{\rho}_2 | \Op{P}_{00,01}}
       \, w_2 \, \Op{P}_{11,00} \,,
       \nonumber
   \\
  \Op{\sigma}_3(T)
   & = \frac{w_3}{2} \Op{\rho}_3\,.
\end{align}
