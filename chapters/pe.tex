% vim: ft=tex iskeyword=@,48-57,_,-,192-255,\: dictionary=bibkeys.lst,labels.lst:
\chapter{Optimization for a Perfect Entangler}
\label{chap:pe}

The implementation of a universal quantum computer requires at least one
entangling two-qubit gate~\cite{DeutschPRSA1995, ZhangPRL2003}, together with
arbitrary single-qubit gates. However, as
discussed in chapter~\ref{chap:quantum}, this need not be one of the
``standard'' gates, such as CNOT.
The construct of the Weyl chamber, see section~\ref{sec:C_LI} in
chapter~\ref{chap:quantum} shows that for any two-qubit gate, there is an
infinite number of equivalent quantum gates that differ only by additional local
operations. For example the CPHASE gate is equivalent in this sense to CNOT, see
Eq.~\eqref{eq:CNOT_CPHASE_equivalent}.

In an optimization context, this has far-reaching consequences. Not every
Hamiltonian allows for the implementation of arbitrary two-qubit gates. For
example, the trapped Rydberg atoms discussed in chapter~\ref{chap:robust} can
only implement diagonal gates, with a fixed global phase. The optimization
target must be chosen to properly reflect this. Reachability is not the only
consideration, however. Equally important is the question whether one two-qubit
gate may be ``easier'' to implement than some other (possibly equivalent)
two-qubit gate, in terms of the minimum required pulse duration, the required
pulse energy, or the difficulty of keeping the population in the logical
subspace in a multi-level system. For example, a Hamiltonian containing among
other terms an exchange interaction $(\SigmaPlus\SigmaMinus
+ \SigmaMinus\SigmaPlus)$ will likely find it easier to implement the
$\sqrtISWAP$ gate than the locally equivalent M-gate. The latter is
generated by a term $(3\SigmaX\SigmaX + \SigmaY\SigmaY)$,
cf.~Table~\ref{tab:appendixGates}, that would be part of the Lie algebra only
indirectly. Likewise, the implementation of iSWAP would likely require more time
than $\sqrtISWAP$, even though both gates generate the same amount of
entanglement.

Using an optimization functional such as Eq.~\eqref{eq:JTsm} or
Eq.~\eqref{eq:JTre} that targets a specific quantum gate is therefore
unnecessarily restrictive. A far better approach is to formulate a more general
optimization functional, and allow optimal control to select the specific gate
that most effectively fits the objective. The geometric theory of local
invariants has been combined with optimal control~\cite{ReichDipl10,
MullerPRA11}, using a functional the minimizes the Cartesian distance to
a target point in the Weyl chamber $(c_1, c_2, c_3)$, see
Eq.~\eqref{eq:cartan_decomposition}, respectively a set of local invariants
$(g_1, g_2, g_3)$, see Eq.~\eqref{eq:li_from_c}.

This chapter, adapted from~\cite{PE1, PE2}, addresses the realization of perfect
entanglers, i.e., quantum gates that are capable of transforming some separable
states into maximally entangled states. This recognizes the fundamental role of
entanglement, as the resource that two-qubit gates contribute to the gate model
of universal quantum computing. Moreover, since nearly 85\% of gates in $\SU(4)$
are perfect entanglers~\cite{MuszPRA2013, WattsE2013}, optimizing for an
arbitrary perfect entangler gives a much broader target.

Extending the idea of optimization in the Weyl chamber, a functional for an
arbitrary perfect entangler has been formulated~\cite{PE1}. As shown in
Fig.~\ref{fig:weylchamber}, the perfect entanglers form a polyhedron bounded by
the planes
\begin{equation}
  \label{eq:pe_planes}
  c_1 + c_2 = \frac{\pi}{2}, \quad
  c_1 - c_2 = \frac{\pi}{2}, \quad
  c_2 + c_3 = \frac{\pi}{2}.
\end{equation}
Thus, an optimization could simply minimize the distance to the nearest wall of
that polyhedron.

Both the underlying local-invariants functional and the newly developed perfect
entanglers functionals are reviewed in section~\ref{sec:pe_functional}. An
optimization in the Weyl chamber, either towards a specific point or towards
a general perfect entangler is most useful for quantum computing implementations
that can generate a large number of gates, such as the superconducting circuits
presented in chapter~\ref{chap:transmon}. In order to obtain some insight in
the structure of the dynamics generated by a typical Hamiltonian,
section~\ref{sec:pe_controllability} shows the controllability for an effective
description of two qubits, truncated to a two levels.
Section~\ref{sec:pe_transmon_oct} then applies the method to a specific example
of transmon qubits, illustrating the power of an optimization towards a perfect
entangler in comparison to the optimization towards a local equivalence class

While the perfect entanglers functional is tested on a Hilbert space
description, the ultimate motivate for using a more general functionals is to
counter decoherence. First, dissipation might affect the processes implementing
different gates differently. In such a situation, a direct optimization might
result in very different fidelities. Optimizing for a general perfect entanlger,
on the other hand, would automatically yield the gate that is implementable with
highest fidelity under dissipation.
In a situation where dissipation cannot be circumvented, the only option is to
perform the gate on a faster than the time scale of the decoherence. The
equation of the quantum speed limit then becomes crucial. Since the speed limit
may differ for various gates, the perfect entanglers functional allows to
identify the optimal gate in such cases. Thus, even without going to the
formalism of Liouville space, the possibility to implement \emph{fast} gates
provides an answer to the problem of implementing gates in an open quantum
system.
The message of this chapter therefore connects that that of
chapter~\ref{chap:robust}, in that the key to finding robust quantum gates with
optimal control is to encode all relevant requirements, and \emph{only} the
relevant requirements in the optimization functional.


\section{The Perfect Entanglers Functional}
\label{sec:pe_functional}

\subsection{Formulation in $c$-space}
\label{subsec:pe_in_c_space}

The Cartan decomposition
\index{Cartan decomposition}%
\begin{equation}
  \Op{U} = \Op{k}_1 \Op{A}_U \Op{k}_2\,, \quad
  \Op{A}_u = \exp\left[\frac{\ii}{2}
                       \sum_{j=1}^{3} c_j \Op{\sigma}_j \Op{\sigma}_j \right]\,,
\end{equation}
cf.~Eq.~\eqref{eq:cartan_decomposition}, allows to determine the ``true''
two-qubit part $\Op{A}_U$ for an arbitrary two-qubit gate $\Op{U}$. The
``local'' components $\Op{k}_1, \Op{k}_2$ can be written in terms of
single-qubit gates
\begin{equation}
  \Op{k}_i = \Op{U}_i^{(1)} \otimes \unity + \unity \otimes \Op{U}_i^{(2)}\,,
  \quad
  \Op{U}_i^{(1,2)} \in \SU(2)\,,
\end{equation}
whereas $\Op{A}_U$ allows for no such decomposition. The entanglement power of
$\Op{U}$ is provided only by the non-local $\Op{A}_U$.

Therefore, the optimization for a general perfect entangler starts from the
prerequisite of eliminating all non-local operations from the figure of merit.
Replacing the target gate $\Op{O}$ and achieved gate $\Op{U}$ in the
functional~\eqref{eq:JTre} by only their non-local components $\Op{A}_O$,
$\Op{A}_U$ yields an optimization functional $\errLec = 1 - F_{\lec}$ with
\begin{equation}
  F_{\lec} = \frac{1}{4} \Re\left(
                \tr\left[ \Op{A}_{O}^\dagger \Op{A}_U \right]
             \right)\,.
 \label{eq:Flec_A}
\end{equation}
The functional $\errLec$ takes its minimum value of zero if and only if $\Op{O}$
and $\Op{U}$ have the same non-local component, as determined by the Weyl
chamber coordinates $(c_1, c_2, c_3)$. Rewriting Eq.~\eqref{eq:Flec_A}
explicitly in these coordinates yields~\cite{PE1}
\begin{equation}
   F_{\lec}
   = \cos\frac{\Delta c_1}{2}
     \cos\frac{\Delta c_2}{2}
     \cos\frac{\Delta c_3}{2}\,, \qquad
   \Delta c_{i} \equiv c_{i,O} - c_{i,U}\,.
  \label{eq:Flec}
\end{equation}
\index{local equivalence class functional!in $c$-space}%
Thus, the optimization can be interpreted as minimizing the geometric distance
$(\Delta c_1, \Delta c_2, \Delta c_3)$ to a target point in the Weyl chamber.

This may now be extended further to obtain a functional that optimizes for
a general perfect entangler, by minimizing the distance not to a specific point
in the Weyl chamber, but to the polyhedron of perfect entanglers.
For any gate $\Op{U}$, we identify the closest of the planes bounding the
perfect entanglers, Eq.~\eqref{eq:pe_planes}. Then, the distance of
$(c_{1,U}, c_{2,U}, c_{3,U})$ and its projection onto that plane is minimized,
resulting in~\cite{PE1}
\begin{equation}
  F_{\PE}(\Op{U})=\begin{cases}
    \cos^2\frac{c_{U,1}+c_{U,2}-\frac{\pi}{2}}{4} &  c_1+c_2\leq\frac{\pi}{2}\\
    \cos^2\frac{c_{U,2}+c_{U,3}-\frac{\pi}{2}}{4} &  c_2+c_3\geq\frac{\pi}{2}\\
    \cos^2\frac{c_{U,1}-c_{U,2}-\frac{\pi}{2}}{4} &  c_1-c_2\geq\frac{\pi}{2}\\
    1                                             &  \text{otherwise.}
  \end{cases}
  \label{eq:FPE}
\end{equation}
Both $F_{\lec}$ and $F_{\PE}$ take values in $[0,1]$ and can thus be interpreted
as fidelities.
\index{perfect entangler functional!in $c$-space}

Generally, the logical two-qubit subspace is embedded in a larger Hilbert space,
such that while the dynamics in the total Hilbert space are unitary, the
dynamics in the subspace may not be. In this case,
a closest unitary $\Op{U}$ can be derived from the non-unitary (projected) gate
$\tildeOp{U}$: If $\tildeOp{U}$ has the singular value decomposition
\index{singular value decomposition}%
\begin{equation}
  \tildeOp{U} = \Op{V} \, \Op{\Sigma} \, \Op{W}^{\dagger}\,,
\end{equation}
then the closest unitary is
\begin{equation}
  \Op{U} = \arg \min_{u} \Norm{\tildeOp{U} - u}
         = \Op{V} \, \Op{W}^\dagger\,.
\end{equation}
\index{closest unitary}%
The distance between $\Op{U}$ and $\tildeOp{U}$ is a measure of unitarity.
\index{unitarity measure}%
The total local-equivalence-class and perfect-entangler fidelities then becomes
\begin{align}
  F_{\lec}(\tildeOp{U})
  &
  = F_{\lec}(\Op{U}) - \Norm{\tildeOp{U} - \Op{U}}\,,
  \label{eq:FUtilde-lec}
  \\
  F_{\PE}(\tildeOp{U})
  &
  = F_{\PE}(\Op{U}) -\Norm{\tildeOp{U} - \Op{U}}\,.
  \label{eq:FUtilde-pE}
\end{align}
The optimization goal is to find the $\tildeOp{U} = \Op{U}$ that minimizes
$1-F_{\lec}$ or $1-F_{\PE}$.


\subsection{Formulation in $g$-space}
\label{subsec:pe_in_g_space}

The formulation of the local equivalence class and perfect entanglers
functionals in terms of the Weyl chamber coordinates $(c_1, c_2, c_3)$,
Eq.~\eqref{eq:FUtilde-lec} and Eq.~\eqref{eq:FUtilde-pE}, have the disadvantage
that there is no way to evaluate them \emph{analytically} for a given gate
$\Op{U}$. This is because calculation of the Weyl coordinates themselves relies
on numerical diagonalization of $\Op{U}$, as discussed in
chapter~\ref{chap:quantum}. Therefore, use of the functionals $F_{\lec}$ and
$F_{\PE}$ is restricted to optimization method that are not gradient-based, such
as the CRAB algorithm presented in section~\ref{subsec:simplex} of
chapter~\ref{chap:numerics}. An application of the functional using this
algorithm can be found in Ref.~\cite{PE2}.

For the other optimal control approaches discussed in
chapter~\ref{chap:numerics}, the gradient of the optimization functional needs
to be evaluated. We must therefore use an equivalent
functional, based not on the Weyl space coordinates $(c_1, c_2, c_3)$, but on
the local invariants $(g_1, g_2, g_3)$, see
Eq.~\eqref{eq:local_invariants}, which depend analytically on $\Op{U}$ and
therefore can be differentiated either with respect to the field (GRAPE) or with
respect to the states (Krotov).

For the local invariants, an appropriate functional can be postulated
as~\cite{ReichDipl10, MullerPRA11}
\begin{equation}
  J_{\LI}(\Op{U}) = (\Delta g_1)^2 + (\Delta g_2)^2 + (\Delta g_3)^2\,,
  \label{eq:J_T_LI_g}
\end{equation}
\index{local equivalence class functional!in $g$-space}%
where $\Delta g_i$ is the Euclidean distance between local invariant $g_i$ of
the obtained unitary $\Op{U}$ and the optimal gate $\Op{O}$.
Eq.~\eqref{eq:J_T_LI_g} takes its minimum value zero if and only if the gate
$\Op{U}$ and $\Op{O}$ are identical up to local transformations. This makes it
slightly more general than Eq.~\eqref{eq:Flec}, as there are some gates
that are locally equivalent but have different Weyl-chamber coordinates, such as
the points $Q$ and $M$ in Fig.~\ref{fig:weylchamber}.

\begin{figure}[tb]
  \centering
  \includegraphics{fpe_topology}
  \caption{%
  Values for the perfect-entanglers $J_{\PE}$ as defined in
  Eq.~\eqref{eq:J_T_PE_g} for sampling points in the Weyl chamber. On the left,
  values for points in the regions $W_0$, $W_0^*$, and $W_1$ (outside of the
  perfect entanglers polyhedron). The values are in the range
  $-2 \le J_{\PE} \le 2$.
  On the right, values $-0.7 \lesssim J_{\PE} \lesssim 0.7$ inside the
  perfect-entanglers polyhedron. The functional takes slightly positive values
  in the center top region, and slightly negative values near the outer bottom
  regions, behind both the $M$ point and the $Q$ point (not visible).
  }
  \label{fig:fpe_topology}
\end{figure}
In order to derive a functional based on the local invariants $(g_1, g_2, g_3)$
for an arbitrary perfect entangler, Eq.~\eqref{eq:li_from_c} is used to rewrite
Eq.~\eqref{eq:FPE}, resulting in~\cite{PE1}
\begin{equation}
  J_{\PE}(\Op{U}) =  g_3 \sqrt{g_1^{2} + g_2^{2}} - g_1\,.
  \label{eq:J_T_PE_g}
\end{equation}
\index{perfect entangler functional!in $g$-space}%
Eq.~\eqref{eq:J_T_PE_g} takes the value zero at the boundary to the perfect
entanglers. The values of $J_{\PE}$ for gates throughout the Weyl chamber is
shown in Fig.~\ref{fig:fpe_topology}. It is important to note that $J_{\PE}$ can
also take negative values. Typically, an optimization will start from a guess
pulse that generates little entanglement, locally equivalent to a gate close to
the identity. In the Weyl chamber, these are defined as the regions $W_0$
(delimited by $O$, $L$, $Q$, and $P$) and $W_0^*$ (delimited by $A_1$, $L$, $M$,
and $N$). In these regions, the functional goes smoothly from its maximum value
of 2 to 0 on the boundary of the perfect entanglers. The sign of the functional
is reversed if the optimization were to start from near the SWAP gate at $A_3$.
This region is denoted as $W_2$, delimited by $A_3, A_2, P, N$. Typically, this
is not a problem. Thinking of entanglement as a resource that is generated by an
interaction acting over a certain period of time, continuing to apply the same
interaction after the entanglement has reached its maximum value of 1 will now
disentangle the entangled states. For example, a
$$\SigmaX\SigmaX + \SigmaY\SigmaY + \SigmaZ\SigmaZ$$ interaction will generate
gates along the $O-P-A_3$ line (see appendix~\ref{AppendixGates}), building up
entanglement as it goes towards the $\sqrt{\text{SWAP}}$ perfect entangler and
then removing at again until the non-entangling SWAP gate is reached. Thus,
guess pulses would be expected to start in $W_2$ only for extremely simple
systems; when this occurs, the sign of the functional must be reversed.

Inside the polyhedron of perfect entanglers, $J_{\PE}$ can also take non-zero
values; it is zero on the planes $L$--$N$--$A_2$ and
$L$--$P$--$A_2$, as well as on all boundaries of the polyhedron.
Thus, if an optimization is allowed to continue after passing through the
perfect-entanglers boundary, it will be pushed towards the center of the
sub-polyhedron delimited by the point $L$, $M$, $A_2$, and $N$ on the right; or
$L$, $Q$, $A_2$ and $P$ on the left.

Like before, we must take into account non-unitarity due to projection onto the
logical subspace. Just as for the value of the functional, $F_{\lec}$ and
$F_{\PE}$, the expression $\Norm{\tildeOp{U} - \Op{U}}$ used to evaluate
unitarity in section~\ref{subsec:pe_in_c_space} cannot
easily be differentiated. As an alternative, we minimize the loss of population
from the logical subspace,
\begin{equation}
p_{\loss} = \frac{1}{4} \Tr\left[ \tildeOp{U}^\dagger \tildeOp{U} \right]
\end{equation}
as an alternative measure of unitarity.
\index{unitarity measure}

In total, the optimization functional for the local invariants and perfect
entanglers then reads
\begin{align}
  J_{\LI}(\tildeOp{U})
  &
  = w J_{\LI}(\Op{U}) + (w-1) \left(
       1 - \frac{1}{4} \Tr\left[ \tildeOp{U}^\dagger \tildeOp{U} \right]
    \right)\,,
  \label{eq:J_LI_tilde}
  \\
  J_{\PE}(\tildeOp{U})
  &
  = w J_{\PE}(\Op{U}) + (w-1) \left(
      1 - \frac{1}{4} \Tr\left[ \tildeOp{U}^\dagger \tildeOp{U} \right]
   \right)\,.
\label{eq:J_PE_tilde}
\end{align}
In order to weight the relative importance of the Weyl-chamber optimization and
the unitarity, the factor $w \in [0,1]$ is used. This factor can be adaptively
changed during the optimization in order to improve convergence.

\section{Controllability of Superconducting Qubits}
\label{sec:pe_controllability}

Optimization towards an arbitrary perfect entangler is most meaningful
if the system dynamics allows the polyhedron of perfect entanglers to
be approached from more than one direction or, more generally, for
optimization paths in the Weyl chamber that explore more than one
dimension.  We therefore investigate the corresponding requirements on
a generic two-qubit Hamiltonian,
\begin{equation}
\begin{split}
  \Op{H}[u_{1}(t), u_{2}(t)]
  &
  =   \sum_{\alpha=1,2} \frac{\omega_{\alpha}}{2} \SigmaZ^{(\alpha)}
      + u_1(t) \left( \SigmaX^{(1)} + \lambda \SigmaX^{(2)}
      \right)
  + \\ & \quad
     + u_2(t) \left(\SigmaX^{(1)} \SigmaX^{(2)}
         +\SigmaY^{(1)} \SigmaY^{(2)}\right)\,.
\end{split}
  \label{eq:H_truncated}
\end{equation}
Here, $\Op{\sigma}_{i}^{(\alpha)}$ is the $i$'th Pauli operator
acting on the $\alpha$'th qubit of transition frequency
$\omega_{\alpha}$, $u_1(t)$ the
single-qubit control field, where $\lambda$ describes how strongly $u_1(t)$
couples to the second qubit relative to the first one, and $u_2(t)$ is the
two-qubit interaction control field.

The Hamiltonian in Eq.~\eqref{eq:H_truncated} is of the form typical for an
effective description of superconducting qubits, truncated to two levels.
In principle, truncation of the Hamiltonian can have significant influence on
controllability. For example, a two-level system is fully controllable, whereas
and infinite harmonic oscillator is not. For superconducting qubits, however, we
do not expect controllability to be enhanced by truncation. Therefore, an
analysis still provides valuable insight into how the Hamiltonian acts in the
Weyl chamber.
\index{controllability}

We analyze the solutions to the differential equation
\begin{equation}
  \frac{\partial}{\partial t}\Op{U}(t)
  = -\ii \Op{H}\left[u(t)\right] \Op{U} \left(t\right)\,,
  \quad \Op{U}(0) = \unity\,,
\label{eq:U_dgl}
\end{equation}
for the unitary transformations $\Op{U}$ generated by the
Hamiltonian~\eqref{eq:H_truncated}.  The reachable set of unitary
transformations for a Hamiltonian is given in terms of the
corresponding dynamical Lie algebra, see section~\ref{subsec:controllability} in
chapter~\ref{chap:quantum}. It can be generated by taking
the terms in \eqref{eq:H_truncated} as a basis (neglecting
orthonormalization for simplicity),
$$
  \SigmaZ^{(1)} \,,\,
  \SigmaZ^{(2)} \,,\,
  \SigmaX^{(1)}  + \lambda \SigmaX^{(2)} \,,\,
  \SigmaX^{(1)} \SigmaX^{(2)} + \SigmaY^{(1)} \SigmaY^{(2)}\,,
$$
and constructing the repeated Lie brackets of these operators. This quickly
yields all 15 canonical basis operators of $SU(4)$, consisting of the
single-qubit operators
  $\SigmaX^{(1)}$,
  $\SigmaX^{(2)}$,
  $\SigmaY^{(1)}$,
  $\SigmaY^{(2)}$,
  $\SigmaZ^{(1)}$, and
  $\SigmaZ^{(2)}$,
as well as the entangling operators
  $\SigmaX^{(1)}\SigmaY^{(2)}$,
  $\SigmaY^{(1)}\SigmaX^{(2)}$,
  $\SigmaY^{(1)}\SigmaZ^{(2)}$,
  $\SigmaZ^{(1)}\SigmaY^{(2)}$,
  $\SigmaX^{(1)}\SigmaZ^{(2)}$,
  $\SigmaZ^{(1)}\SigmaX^{(2)}$,
  $\SigmaX^{(1)}\SigmaX^{(2)}$,
  $\SigmaY^{(1)}\SigmaY^{(2)}$, and
  $\SigmaZ^{(1)}\SigmaZ^{(2)}$.
Hence the system is completely controllable, and any point in the Weyl
chamber can be reached.

\begin{figure}[tb]
  \centering
  \includegraphics{controllability}
  \caption{Sampling of reachable points in the Weyl
    chamber, obtained by solving Eq.~\eqref{eq:U_dgl}
    for  the Hamiltonian~\eqref{eq:H_truncated} with $\lambda=1$.
    In panel (a),
    $u_{1}(t)$ is a random pulse $\in [0,1]$, $u_{2}(t) \equiv 10^{-3}$,
    and $\omega_1 = 1.0 \neq \omega_2=1.1$.
    In panel (b), in addition
    $\omega_1 = \omega_2 = 1.0$.
    In panel (c),
    $u_{1}(t)$ and $u_{2}(t)$ are both independently random $\in [0,1]$, and
    $\omega_1 = \omega_2 = 0$.
    In panel (d), in addition $u_{1}(t) \equiv u_{2}(t)$.
    The red line in panel (d) is obtained for $u_1(t) \equiv 0$ and $u_{2}(t)$
    random $\in [0,1]$ ($\omega_1 = \omega_2 = 0$).
  }
  \label{fig:controllability}
\end{figure}
The complete controllability can be verified numerically, by solving
Eq.~\eqref{eq:U_dgl} for a random sequence of pulse values. The gates obtained
within 1000 steps are shown in panel (a) of Fig.~\ref{fig:controllability}, and
demonstrate full controllability, since there are points in all regions of the
Weyl chamber.
Continuing the procedure to infinity would eventually fill the entire chamber.
Neither setting $u_2(t)$ constant nor choosing $\lambda=0$ places any
restrictions on the controllability -- indeed it is sufficient if either
the single qubit terms or the interaction term is controllable.
While the controllability in this example was analyzed for
arbitrary values of the parameters,
the form of the Hamiltonian and the ratio between $\omega_{1,2}$ and $u_2$ fits
the description of superconducting transmon qubits, with qubit energies in the
GHz range and static qubit-qubit-coupling in the MHz range.

Introducing symmetries in the Hamiltonian~\eqref{eq:H_truncated} reduces the
controllability. First, we consider a situation in which
the two qubits operate at the same frequency $\omega_1 = \omega_2$. In this
case, the dynamic Lie algebra consists of only 9 instead of 15
operators. Consequently, not every
two-qubit gate can be implemented. However, the nine operators include
$\SigmaX^{(1)} \SigmaX^{(2)}$,
$\SigmaY^{(1)} \SigmaY^{(2)}$,
$\SigmaZ^{(1)} \SigmaZ^{(2)}$,
which are sufficient to reach every point in the Weyl chamber, cf.\
Eq.~\eqref{eq:cartan_decomposition}. This is illustrated
in panel (b) of Fig.~\ref{fig:controllability}. Despite
the reduced controllability, the Weyl chamber is more evenly filled after the
same 1000 propagation steps as in panel (a). This counterintuitive
finding is due to the lower dimension of the random walk, with no
resources being ``wasted'' on the missing six single-qubit directions.

The set of gates that can be implemented with Hamiltonian~\eqref{eq:H_truncated} is
more severely restricted if both qubits are
completely degenerate, $\omega_1 = \omega_2 = 0$. This is typical for
superconducting charge qubits operated at the  ``charge degeneracy
point''. Without any drift term, the Lie algebra consists of
only four generators,
$\SigmaZ^{(1)} \SigmaY^{(2)} + \SigmaY^{(1)} \SigmaZ^{(2)}$ and
$\SigmaY^{(1)} \SigmaY^{(2)} - \SigmaZ^{(1)} \SigmaZ^{(2)}$ in addition
to the two original terms.
The implications for controllability in the Weyl
chamber are not immediately obvious since three
generators can be sufficient to obtain full Weyl chamber
controllability. The
easiest approach is to perform a numerical analysis, the results of which are
shown in panel (c) of Fig.~\ref{fig:reducedControl}. Two independent randomized
pulses $u_1(t)$ and $u_2(t)$ were used. The reachable points
lie on a plane, which due to the reflection symmetries
appears as two triangular branches, indicated by the shaded triangles,
$O$--$(\frac{2\pi}{3},\frac{\pi}{3}, \frac{\pi}{3})$--$A_2$ and
$A_1$--$(\frac{\pi}{3},\frac{\pi}{3}, \frac{\pi}{3})$--$A_2$.
Note that almost none of the common two-qubit gates are included in this set.

If only a single pulse is available to drive both the single-qubit and
two-qubit terms, $u_1(t) \equiv u_2(t)$, and the qubits are
degenerate, $\omega_1 = \omega_2 = 0$, there is a single
generator for the dynamics. This situation is shown
in panel (d) of Fig.~\ref{fig:controllability}.
Although there is only a single generator for the dynamics, a
two-dimensional subset of the Weyl chamber can be reached. However,
the subset is no longer the full plane as it is for two independent
pulses, panel (c). Without single-qubit
control, the center of the plane is not longer reachable.
It is important to remember that while a single
generator yields points on a line in the Weyl chamber (not necessarily
a straight one), it can still fill
an arbitrary subset of the Weyl chamber, due to
reflections at the boundaries.
A similar example, restricted to the ground plane of the
Weyl chamber, has been analyzed in Ref.~\cite{ZhangPRA03}.

Lastly, if there is no control over the individual qubits at all,
$u_1(t) \equiv 0$, the only remaining generator is
$\SigmaX^{(1)} \SigmaX^{(2)}+ \SigmaY^{(1)} \SigmaY^{(2)}$.
This corresponds to the straight line
$O$--$A_2$ in the Weyl chamber, shown in red in panel (d) of
Fig.~\ref{fig:controllability}. The line is reflected back onto itself
at the $A_2$ point. Thus, in this case only a truly one-dimensional subset
of reachable gates in the Weyl chamber can be realized.

For a Hamiltonian that allows for  a one-dimensional search-space only,
optimal control calculations with a functional targeting all perfect
entanglers will not yield results better than direct gate
optimization.
In the above example, the perfect entangler that can be generated with the
shortest gate duration is the $\sqrtISWAP$ at the point $Q$.

In contrast, for Hamiltonians allowing for two or three search directions in
the Weyl chamber, especially panels (a) and (b) in
Fig.~\ref{fig:controllability}, the polyhedron of perfect
entanglers may be approached from several different angles.  Optimization with a
functional targeting all perfect entanglers is then non-trivial. In
such a search, the optimized solution will depend on additional constraints
in the functional and the initial guess field.

\section{Optimization of Transmon Qubits}
\label{sec:pe_transmon_oct}

\subsection{Model}

\begin{table}[tb]
  \centering
  \begin{tabular}{llrl}
  \toprule
  left qubit frequency           &  $\omega_1$  & 4.380 &GHz \\
  right qubit frequency          &  $\omega_2$  & 4.614 &GHz \\
  left qubit anharmonicity       &  $\alpha_1$  & -210  &MHz \\
  right qubit anharmonicity      &  $\alpha_1$  & -215  &MHz \\
  effective qubit-qubit coupling &  $J^{\eff}$  & -3.0  &MHz \\
  relative coupling strength     &  $\lambda$   & 1.03  &~\\
  \bottomrule
  \end{tabular}
  \caption{Parameters for the transmon Hamiltonian Eq.~({\ref{eq:PE_H2T}})}.
  \label{tab:pe_transmon_parameters}
\end{table}
Having verified that the Hamiltonian in Eq.~\eqref{eq:H_truncated} allows for
full controllability, we now extend the discussion to a more realistic
description of two transmons, i.e., anharmonic multi-level systems that interact
jointly with a cavity are described by  a generalized
Jaynes-Cummings Hamiltonian, as described in chapter~\ref{chap:transmon}.
The energy of each transmon qubit transition given by $\omega_1$, $\omega_2$ for
the first (``left'') and second (``right'') transmon respectively.
Higher levels are given as a Duffing oscillator with anharmonicity
$\alpha_1$, $\alpha_2$. Each qubit couples to the cavity with coupling strength
$g_1$ , $g_2$.
In the dispersive limit $\Abs{\omega_i -\omega_r} \gg |g{i}|$, $i=1,2$, with
$\omega_r$ the cavity frequency, the cavity can be eliminated and an effective
two-transmon Hamiltonian is obtained. The coupling between each transmon and the
cavity tuns into an effective qubit-qubit coupling
\begin{equation}
J^{\eff}
\approx
    \frac{g_1 g_2}{(\omega_1-\omega_r)}
  + \frac{g_1 g_2}{(\omega_2-\omega_r)}\,.
\end{equation}
In most current setups, $J^{\eff} \ll \Abs{\omega_2 - \omega_1}$,
and the two-transmon Hamiltonian can be approximated as \cite{PolettoPRL2012}
\begin{equation}
\begin{split}
  \Op{H}_{2T}
  &
  \approx
    \sum_{i=1,2} \left(
        \left( \omega_i + \frac{\alpha_i}{2}\right)
        \Op{b}_i^{\dagger} \Op{b}_i
        - \frac{\alpha_i}{2} \left( \Op{b}_i^{\dagger} \Op{b}_i \right)^2
    \right)
  + \\ \qquad &
  + J^{\eff} \left( \Op{b}_1^\dagger \Op{b}_2
                  + \Op{b}_1 \Op{b}_2^\dagger
            \right)
  + \Omega(t) \left( \Op{b}_1 + \Op{b}_1^\dagger
                    + \lambda \Op{b}_2 + \lambda \Op{b}_2^\dagger \right)
\end{split}
\label{eq:PE_H2T}
\end{equation}
where $\Omega(t)$ is the driving field that couples to the cavity. Typical
parameters are given in Tab.~{\ref{tab:pe_transmon_parameters}}. A two-level
truncation of this Hamiltonian corresponds to Eq.~\eqref{eq:H_truncated}.

\subsection{Krotov's Method}

The optimization is performed using Krotov's method,
section~\ref{subsec:Krotov} of chapter~\ref{chap:numerics}, for both the local
invariants functional~\eqref{eq:J_LI_tilde} and the perfect entanglers
functional~\eqref{eq:J_PE_tilde}. No dissipation is taken into account.
The basis states for the forward propagation are the Bell states
\begin{align}
  \Ket{\phi_1} &=  \frac{1}{\sqrt{2}} \left(    \ket{00} - \ii \ket{11}\right)\,,&
  \Ket{\phi_2} &= -\frac{1}{\sqrt{2}} \left(\ii \ket{01} -     \ket{10}\right)\,,\\
  \Ket{\phi_3} &= -\frac{1}{\sqrt{2}} \left(\ii \ket{01} +     \ket{10}\right)\,,&
  \Ket{\phi_4} &=  \frac{1}{\sqrt{2}} \left(    \ket{00} + \ii \ket{11}\right)\,,
\end{align}
\index{magic basis}%
as required for the calculation of the local invariants,
cf.~Eq.\eqref{eq:qmagic}. The equations of motion are given by the standard
Schrödinger equation, see Eq.~\eqref{eq:fw_eqm} and Eq.~\eqref{eq:bw_eqm}.

The boundary conditions for the backward propagation,
\begin{equation}
  \Ket{\chi_k^{(i)}(T)}
   = - \krotovdifquo[J_T]{\Bra{\phi_k}}{\phi_k^{(i)}(T)}\,,
\end{equation}
cf.~Eq.~\eqref{eq:chi_boundary}, are straightforward to
calculate, although tedious. For $J_{\LI}$, they can be found in
Ref.\cite{ReichDipl10}. For $J_{\PE}$, they are listed in
appendix~\ref{AppendixPE}.

Both $J_{\LI}$ and $J_{\PE}$ are highly non-convex functionals. Therefore, they
require the second order contribution in the Krotov update equation. With the
standard constraint~\eqref{eq:g_a_delta}, this update equation is
\begin{equation}
\begin{split}
  \Delta\Omega(t)
  &=
  \frac{S(t)}{\lambda_a} \Im \left[
    \sum_{k=1}^{N}
    \Bigg\langle
      \chi_k^{(i)}(t)
    \Bigg\vert
      \Bigg(\dHdOmega \Bigg)
    \Bigg\vert
      \phi_k^{(i+1)}(t)
    \Bigg\rangle
 + \right. \\ & \qquad \left.
    + \frac{1}{2} \sigma(t)
    \Bigg\langle
      \Delta\phi_k(t)
    \Bigg\vert
      \Bigg(\dHdOmega\Bigg)
    \Bigg\vert
      \phi_k^{(i+1)}(t)
    \Bigg\rangle
  \right]\,,
\end{split}
\label{eq:krotov_second_order_update}
\end{equation}
\index{Krotov!second order update equation}%
cf.~Eq.~\eqref{eq:krotov_proto_update}, with $\sigma(t)$ given
by Eq.~\eqref{eq:sigma_A}.
The optimization is carried out
for different gate durations between \SI{25}{ns} and \SI{400}{ns}, using
a $\sin$-squared pulse of \SI{35}{MHz} peak amplitude as the guess pulse
$\Omega^{(0)}(t)$.

\subsection{Optimization Results}


\begin{figure}[tb]
  \centering
  \includegraphics{weyl_paths}
  \caption{Optimized gates in the Weyl chamber, for two transmon qubits,
  optimized with Krotov's method for the perfect-entangler (PE) functional in
  Eq.~(\ref{eq:J_PE_tilde}).
  The point at which each optimization enters the PE polyhedron, or the
  end point of the optimization if no PE can be obtained,
  is shown by a black dot and labeled with the gate duration.
  The entire optimization paths for  $T=50\,$ns and $T=400\,$ns are
  shown in light blue and dark purple, respectively, with the
  starting points labeled by 50$^*$ and 400$^*$.}
  \label{fig:transmon_weyl_paths}
\end{figure}
Figure~\ref{fig:transmon_weyl_paths} shows the results of the
optimization in the Weyl chamber. The point at which each optimization
enters the perfect entanglers
polyhedron is indicated by a black dot and labeled
with the gate duration. For $T<50\,$ns, no perfect entangler can be
reached -- defining heuristically the quantum speed limit (QSL) for this
transformation.
\index{quantum speed limit}%
In order to illustrate how the optimization proceeds,
the optimization paths for $T=50\,$ns, i.e., the gate at the
QSL, and a high-fidelity gate ($T=400\,$ns) are traced in light blue
and dark purple, respectively. Both optimizations start
in the $W_0^*$ region (near the $A_1$ point). The gate obtained with
the guess pulse for
$T=50\,$ns is significantly farther away from the surface of the
polyhedron of PE than that for the guess pulse with $T=400\,$ns.
Optimization for $T=400\,$ns therefore
moves directly towards the $W_0^*$ surface of the
PE polyhedron, whereas the optimization for $T=50$~ns enters the
ground plane and emerges in the $W_0$ region, before finally
reaching the $W_0$ surface of the polyhedron of perfect entanglers. The jump
from $W_0^*$ to $W_0$ is indicated by the light blue arrow. We find
the optimization to enter $W_0$ from $W_0^*$ for
durations $< 100\,$ns,  whereas for longer gate duration the optimizations
stay within $W_0^*$ entirely. The different optimization paths are a
result of the competition between the two objectives -- to reach a
perfect entangler, and to implement a gate that is unitary in the
logical subspace (the points shown in
Fig.~\ref{fig:transmon_weyl_paths} are the Weyl chamber
coordinates of the unitary $U$ closest to the actual time evolution
$\tilde{U}$). The latter objective is more difficult to realize
for shorter gate durations, resulting in a more indirect approach to
the polyhedron of perfect entanglers than one might expect when
considering that objective alone.


\begin{figure}[tb]
  \centering
  \includegraphics{tm_conv_LI}
  \caption{Comparison of optimization success for the
    PE functional compared to the local invariants (LI) functional
    for several points in the Weyl chamber. The
    optimization success using Krotov's method
    is measured in $c$-space, although the optimization functionals are
    defined in $g$-space (see text for details). For the
    LI-optimization, the results are fully converged. For the PE-optimization,
    the results are converged to a relative change below
    $10^{-2}$ (black solid curve) and $10^{-3}$ (gray dash-dash-dotted curve).
  }
  \label{fig:tm_conv_LI}
\end{figure}
It is instructive to compare the optimization success of the perfect entangler
functional, Eq.~(\ref{eq:J_PE_tilde}) with the optimization under the
local-invariants functional, Eq.~(\ref{eq:J_LI_tilde}) for a few select points
of the Weyl chamber. This is shown in Fig.~\ref{fig:tm_conv_LI}.
While the optimization was driven by the $g$-space formulation of the
functionals, the fidelities $F_{\lec}$ and $F_{\PE}$ define a more intuitive
figure of merit for the analysis.
As defined in Eq.~\eqref{eq:FUtilde-lec} and Eq.~\eqref{eq:FUtilde-pE}, the
fidelity is reduced by non-unitarity.

The results of Fig.~\ref{fig:tm_conv_LI} show how for different gate
durations, different gates are easiest to reach. In agreement
with the results of Fig.~\ref{fig:transmon_weyl_paths}, for durations
$< 50$~ns, the jump in the optimization error indicates a speed limit.
For short gate durations, $50~\text{ns} \le T \le 100~\text{ns}$, optimization
towards the point $Q$ in the Weyl chamber is most successful. This
matches the optimized gates for $T \le 100\,$ns in
Fig.~\ref{fig:transmon_weyl_paths} being near the $Q$ point. Also
correspondingly, the longer gate durations end near the $N$ point. The
failure to reach the point $Q$ for longer durations is due to the
symmetry structure of the Weyl chamber.
Namely, for the ground plane of the chamber, there is a mirror axis defined
by the line through $L$ and $A_2$, where mirrored points are in the same local
equivalence class. Both the $Q$-point and the $M$ point have local
invariants of $g_1 = \frac{1}{4}, g_2=0, g_3=1$. Since the
optimization was performed in $g$-space, these two points are not
distinguishable; indeed, for long gate
durations, the $Q$-optimization successfully reached the $M$
point.

In comparison with the local invariants optimization, the perfect
entanglers functional shows excellent performance. It automatically identifies
the optimal gate for a given gate duration and reaches significantly better
fidelities. This is due to the fact that the desired entangling power of $U$ can
usually be obtained in just a few tens of iterations of the algorithm, and the
remainder of the optimization
then focuses on improving the unitarity of the obtained gate $\tilde{U}$.
Most strikingly, we find that for the optimization towards a specific local
equivalence class, the convergence rate becomes extremely small as the optimum
is approached. All the results shown in Fig.~\ref{fig:tm_conv_LI} are
converged to a relative change below $10^{-4}$, such that no measurable
improvement can be expected within a reasonable number of iterations.
While in principle (due to the full controllability
of the system), the direct optimizations should yield arbitrarily
small gate errors, as long as the gate duration is above the quantum
speed limit, in practice this depends
on numerical parameters such as the weight $\lambda_a$ in Krotov's
method and may take a extremely large number of iterations or stagnate,
as we observe here. The perfect
entangler optimization shows remarkable robustness with respect to
this issue. We observed very little slow-down in convergence. The black
curve in Fig.~\ref{fig:tm_conv_LI} for the PE-optimization already yields
a significantly smaller optimization error than any of the LI-optimizations, but
is only converged to a relative change of $10^{-2}$. Even the gray
dash-dash-dotted curve, labeled PE$^*$, is only converged to a relative change
of $10^{-3}$, and thus the optimization would still yield considerably better
results if it were to be continued.

\begin{figure}[tb]
  \centering
  \includegraphics{tm_conv_favg}
  \caption{Analysis of the sources of error for PE of
    2 transmon qubits:
    Population loss from the logical subspace (light red squares),
    concurrence error of the closest unitary gate $\Op{U}$ in the logical
    subspace (blue circles), and average gate error,
    $\varepsilon_{avg}=1-F_{\avg}$,  with which $\Op{U}$ is
    implemented (red circles).
}
  \label{fig:tm_conv_favg}
\end{figure}
The values of the optimization error in Fig.~\ref{fig:tm_conv_LI} of $10^{-3}$
or $10^{-2}$ should not be understood to indicate a gate error above the quantum
error correction threshold.
Whereas the optimization error relates only to a figure of merit used for
optimization, the relevant physical quantity that would be determined in
an experiment is the average gate fidelity, Eq.~\eqref{eq:Favg}

Figure~\ref{fig:tm_conv_favg} shows the
generated entanglement as measured by the concurrence and the average
gate error, $\varepsilon_{avg}=1-F_{avg}$,
together with the population loss from the logical
subspace. $\Op O$ is taken to be the unitary that is closest
to the projection of the realized operation from the full Hilbert space
onto the logical subspace. For $T> 50\,$ns, the gate errors are at or
below $10^{-4}$.
For shorter gate durations, insufficient entanglement is generated,
cf.\ blue dashed curve in Fig.~\ref{fig:tm_conv_favg}. Once $T$ is
sufficiently large to generate the desired entanglement,
the only source of error is loss of population from the
logical subspace, shown in light red in
Fig.~\ref{fig:tm_conv_favg}. This loss does not depend on the choice
of the weight $w$ in Eq.~\eqref{eq:J_PE_tilde}. When the gate duration
is increased, optimization yields gates that are exponentially more
unitary, as indicated by the linear decrease of the average gate error
in our semi-log plot.
The difficulty to ensure unitarity on the logical subspace is typical
for weakly anharmonic ladders, as found in superconducting transmon or
phase qubits. Optimal control can be successfully employed to tackle
the problem of ensuring unitarity in the logical subspace,
in addition to generating entanglement, as exemplified in
Fig.~\ref{fig:tm_conv_favg}.
