\chapter{Interaction between an Atom and a Laser Field}
\label{AppendixLMI}

We consider a electromagnetic field with the vector and electrostatic potential
\begin{gather}
  \vec{A}(\vec{r}, t)
  = \frac{E_{0}}{\omega} \vec{e}_z \sin\left( ky - \omega t \right) \\
  \Phi(\vec{r}, t) = 0
\end{gather}
This corresponds to a cosine shape electromagnetic field propagating along the
$y$-axis, with the electric field oscillating in $z$-direction and the magnetic
field oscillating in $y$-direction,
\begin{gather}
  \vec{E}(\vec{r}, t) = - \frac{\partial}{\partial t} \vec{A}(\vec{r}, t)
                        - \vec{\nabla} \Phi(\vec{r}, t)
                      = E_0 \vec{e}_z \cos(ky - \omega t) \\
  \vec{B}(\vec{r}, t) = \vec{\nabla} \times \vec{A}(\vec{r}, t)
                      = \frac{\partial}{\partial y} A_z \vec{e}_x
                      = B_0 \vec{e}_x \cos(ky - \omega t)\,
\end{gather}
with $B_0 = \frac{E_0}{\omega} k = \frac{E_0}{c}$, where $k=\frac{\omega}{c}$,
$c$ is the speed of light and $\omega$ is the laser frequency.

The Hamiltonian for an atom's valence electron at position $\vec{r}$, with
electron mass $m$, electron-charge $q$, and $\vec{r}$ and $\vec{p}$ now being
operators, reads
\begin{equation}
\begin{split}
\Op{H}
  & = \frac{1}{2m} \left[
      \vecOp{p} - q \vecOp{A}(\vecOp{r}, t)
    \right]^2
    + \Op{V}(\vecOp{r})
    - \frac{q}{m} \vecOp{S} \cdot \vecOp{B}(\vecOp{r}, t)
    + \vec{\nabla} \Op{\Phi}(\vecOp{r}, t)
 \\ &
  = \Op{H}_0 - \frac{q}{m} \vecOp{p} \cdot \vecOp{A}
             - \frac{q}{m} \vecOp{S} \cdot \vecOp{B}
             + \frac{q}{2m} \left[ \vecOp{A}(\vecOp{r}, t)\right]^2\,,
  \label{eq:LMI_ham}
\end{split}
\end{equation}
with the electron's drift Hamiltonian
\begin{equation}
  \Op{H}_0 = \frac{\vecOp{p}^{\,2}}{2m} + \Op{V}(\vecOp{r})
\end{equation}
and the spin operator $\vecOp{S}$ coupling to the magnetic field.
The origin of the coordinate system is in the atoms nucleus.
Since $\Norm{\vec{A}^{\,2}} \ll \Norm{\vec{A}}$ for realistic laser field
amplitudes, we set the last term to zero.

The spatial dependence of the vector potential, $ky = \frac{2\pi y}{\lambda}$,
where $y$ is on the order of an atomic radius $a_0$ and $\lambda$ is the
wavelength of the laser is extremely small. We can therefore Taylor-expand the
vector potential as
\begin{equation}
\begin{split}
  \vec{A}(\vec{r}, t)
  & = \frac{E_0}{w} \vec{e}_z \sin(ky - \omega t)
  \\
  & = \frac{E_0}{2 \ii \omega} \vec{e}_z \left(
        \ee^{\ii k y} \ee^{-\ii \omega t} - \ee^{-\ii k y} \ee^{\ii \omega t}
      \right)
  \\
  & \approx
      \frac{E_0}{2 \ii \omega} \vec{e}_z \left(
        (1 + \ii k y) \ee^{-\ii \omega t} - (1 - \ii k y) \ee^{\ii \omega t}
      \right)
  \\
  & = \frac{E_0}{w} \vec{e}_z \sin(\omega t) + B_0 y \vec{e}_z \cos(\omega t)
\end{split}
\label{eq:LMI_taylor}
\end{equation}

Also, since an electron that is localized with a Bohr radius $a_0$
\index{Bohr radius}
must have a minimum momentum $\vecOp{p}$ such that $\frac{\hbar}{p} \le a_0$, and
$\vecOp{S}$ is on the order of $\hbar$, we can show
$\Norm{\vecOp{p} \cdot \vecOp{A}} \gg \Norm{\vecOp{S}\cdot \vecOp{B}}$,
\begin{equation}
  \frac{\Norm{\vecOp{S} \cdot \vecOp{B}}}{\Norm{\vecOp{p} \cdot \vecOp{A}} }
  \approx \frac{\hbar k E_0/\omega}{p E_0 / \omega}
  = \frac{\hbar k}{p}
  < \frac{a_0}{\lambda} \ll 1
\end{equation}
Therefore, we are justified in approximating
\begin{equation}
  \vec{B}(\vec{r}, t) \approx B_0 \vec{e}_x \cos{\omega t}\,.
\end{equation}

Inserting this and Eq.~\eqref{eq:LMI_taylor} into Eq.~\eqref{eq:LMI_ham} yields
\begin{equation}
  \Op{H}
  \approx
    \Op{H}_0
    - \frac{q}{m} \frac{E_0}{\omega} \Op{p}_z \sin(\omega t)
    - \frac{q B_0}{m} B_0 \Op{p}_z \Op{y} \cos(\omega t)
    - \frac{q}{2m} \Op{S}_x B_0 \cos(\omega t)\,.
\end{equation}
Furthermore,
\begin{equation}
  \Op{p}_z \Op{y}
   = \frac{1}{2} \left(\Op{p}_z \Op{y} - \Op{z} \Op{p}_y\right)
     +\frac{1}{2} \left(\Op{p}_z \Op{y} - \Op{z} \Op{p}_y\right)
  = \frac{1}{2} \Op{L}_x +\frac{1}{2} \left(\Op{p}_z \Op{y}
     - \Op{z} \Op{p}_y\right)\,,
\end{equation}
resulting in
\begin{equation}
\begin{split}
  \Op{H}
 &
  \approx
    \Op{H}_0
    - \frac{q}{m} \frac{E_0}{\omega} \Op{p}_z \sin(\omega t)
    - \frac{q}{2 m c} E_0 \cos(\omega t)
      \left[ \Op{p}_z \Op{y} - \Op{z} \Op{p}_y\right]
 + \\ & \quad
    - \frac{q}{2m} \left(\Op{L}_x + \Op{S}_x\right) B_0 \cos(\omega t)\,.
\end{split}
\end{equation}

The three interaction terms are interpreted as follows:
\begin{itemize}[noitemsep]
  \item
  \begin{equation}
  \Op{H}_{ED} = \frac{q}{m} \frac{E_0}{w} \Op{p}_z \sin(\omega t)
  \end{equation}
  is the \emph{electric dipole} interaction in momentum space.
  \index{dipole moment!electric}
  It can be rewritten to its more familiar form in coordinate space
  \begin{equation}
  \Op{H}_{ED} = q \Op{z} \, E_0 \cos(\omega t) = \Op{\mu} E_0 \cos(\omega t)\,,
  \end{equation}
  where $\Op{\mu}$ has been introduced as the dipole operator.
  \index{dipole operator}
  \item
  \begin{equation}
    \Op{H}_{EQ} = - \frac{q}{2 m c} E_0 \cos(\omega t)
                    \left[ \Op{p}_z \Op{y} - \Op{z} \Op{p}_y\right]
  \end{equation}
  describes the \emph{electric quadrupole} interaction.
  \index{quadrupole moment!electric}
  \begin{item}
  \begin{equation}
    \Op{H}_{MD} = - \frac{q}{2m} \left(\Op{L}_x
                  + \Op{S}_x\right) B_0 \cos(\omega t)\,.
  \end{equation}
  describes the \emph{magnetic dipole} interaction.
  \index{dipole moment!magnetic}
  \end{item}
\end{itemize}
Both the electric quadrupole and the magnetic dipole are negligible compared
to the electric dipole. Therefore, in the dipole-approximation the total
Hamiltonian becomes
\begin{equation}
  \Op{H} \approx \Op{H}_0 + \Op{\mu} E(t)\,.
  \label{eq:LMI_dipole_ham}
\end{equation}
The dipole approximation results from the assumption that the wavelength of the
laser is much larger than the width of the atom, and thus that the spatial
dependence of the field can be dropped, allowing to define the $z$-component of
the electric field as
\begin{equation}
  E(t)  = E_0 \cos(\omega t)\,.
\end{equation}

When Eq.~\eqref{eq:LMI_dipole_ham} is written in the energy representation given
by the eigenstates of $\Op{H}_0$, the selection rules for the dipole transitions
are obtained. That is, for certain quantum numbers the corresponding matrix
element of $\Op{\mu}$ will vanish. For a Hydrogen atom with eigenstates
$\Ket{nlm}$, the dipole is zero unless $\Delta l = 1$ and $\Delta m = 0, 1$.
\index{selection rules}

\enlargethispage{\baselineskip}
For example, we may consider
a Hamiltonian for a sub-system consisting of three levels $\Ket{0}$, $\Ket{1}$,
and $\Ket{2}$, with energies $E_0$, $E_1$, $E_2$. The dipole transition $\Ket{0}
\rightarrow \Ket{1}$ and $\Ket{2} \rightarrow \Ket{3}$ is allowed with
a resulting dipole moment of $\mu_{01} = \Braket{0|\Op{\mu}|1}$ and
$\mu_{12}=\Braket{1|\Op{\mu}|2}$, respectively, but
$\Ket{1} \rightarrow \Ket{3}$ is forbidden. This Hamiltonian
would we written in the energy representation as
\begin{equation}
  \Op{H} = \begin{pmatrix}
    E_0           & \mu_{01} E(t) &   0           \\
    \mu_{01} E(t) & E_1           & \mu_{12} E(t) \\
    0             & \mu_{12} E(t) & E_2           \\
  \end{pmatrix}\,,
\end{equation}
the form used for the Hamiltonians e.g.\ in chapters~\ref{chap:robust}
and~\ref{chap:3states}.


