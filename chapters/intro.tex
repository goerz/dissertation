% vim: ft=tex iskeyword=@,48-57,_,-,192-255,\: dictionary=bibkeys.lst,labels.lst:
\begin{savequote}[75mm]
It is often stated that of all the theories proposed in this century, the
silliest is quantum theory. In fact, some say that the only thing that quantum
theory has going for it is that it is unquestionably correct.
\qauthor{Michio Kaku (\textit{Hyperspace})}
\end{savequote}

\chapter{Introduction}
\label{chap:introduction}

The theory of quantum mechanics has been as perplexing as it has been
successful. Despite many of the philosophical questions remaining open from its
very inception, questions of realism, non-locality, and the role of chance in
our description of nature, quantum mechanics has been proven again and again to
to accurately describe the fundamentals of nature. As the theory has matured,
many scientists and engineers have set aside the fundamental concerns and moved
on to the applications of quantum mechanics. Without its detailed understanding,
we would have no lasers, no modern electronics, no MRI scanners. In fact, one
would be hard-pressed to find any technological advance of the last decades that
does not in some way rely on our understanding of quantum mechanics.

We are presently at the cusp of a second wave of quantum technology, based not
just on the understanding of quantum effects, but on the active control and
manipulation of quantum systems. This new ``quantum engineering'' has
considerable challenges, but also holds the promise of unprecedented new
possibilities. One possible application is the control of chemical
reactions, e.g.\ with shaped laser pulses \cite{BrumerShapiro}.
Another field of research attempting to make use of quantum control is renewable
energy, where a new generation of solar power cells could mimic the process of
photosynthesis, which has been shown to rely on quantum effects
\cite{ColliniScience09, SarovarNatPhys10}.  A third and very broad example (and
the main focus of this thesis) is the entire field of quantum information
processing, where logical operations are performed by manipulating the quantum
system on which the information is encoded \cite{NielsenChuang}.

In all of these examples, the core idea is to to rely on the wave character
of matter and use interferences to steer the system in some particular way.



The problem is well-described by Schrödinger's famous cat.
As is frequently remarked, the trick is not so much the cat as it is the box.


This thesis is organized as follows: The first two chapters give an introduction
into the mathematical framework and the numerical tools.
Specifically, Chapter~\ref{chap:quantum} outlines the basics of quantum
computing, followed by an overview over the description of open quantum systems.
Chapter~\ref{chap:numerics} discusses the numerical tools that have been used
and developed.
Chapter~\ref{chap:pe}
Chapter~\ref{chap:robust}
Chapter~\ref{chap:3states}
Chapter~\ref{chap:tmintro}
Chapter~\ref{chap:tmresults}
Finally, chapter~\ref{chap:outlook} gives an outlook.

