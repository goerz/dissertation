% vim: ft=tex iskeyword=@,48-57,_,-,192-255,\: dictionary=bibkeys.lst,labels.lst:
\chapter{Introduction}
\label{chap:intro}

Quantum mechanics has been one of the most fundamental and significant
achievements of modern physics. While to this day, the theory can seem
perplexing and at odds with the categories of everyday experience, it has
withstood any test put to it, and indeed allowed unprecedented insights into the
fundamentals of nature. Many of the philosophical questions that quantum
mechanics raises remain open. Is quantum mechanics just a mathematical tool
that happens to yield accurate predictions, or are wave functions in fact
``real'' objects?  Can it be true that particles are everywhere at once, until
a measurement pins them down? Is it acceptable to have a theory of nature that
involves blind chance at a fundamental level? Can the gap between the quantum
and the classical ever be bridged? All of these questions continue to be
explored by both philosophers and scientists.

At the same time, the theory itself has matured, and many scientists and
engineers have gone beyond such fundamental concerns and moved on to explore its
applications, sparking a technological revolution. Remarkably, while the
knowledge of quantum mechanics seems utterly irrelevant to the average person on
the street, it is at the core of an immeasurable number of today's technologies.
In fact, one would be hard-pressed to find any technological advance of the last
decades that does not in some way rely on our understanding of quantum
mechanics.  Without it, we would have no lasers, no MRI scanners, no modern
chemistry, and maybe most importantly, no modern electronics and thus no
information technology.

Presently, we are at the cusp of a second wave of quantum technology that is
based not just on the passive understanding of quantum effects, but on the
active control and manipulation of quantum systems \cite{DowlingPTRSA2003}.
This new ``quantum engineering'' has considerable challenges, but also holds the
promise of unprecedented new possibilities; for example the
control of chemical reactions with shaped laser pulses
\cite{BrumerShapiro}, or advances in renewable energy, where a new generation of
solar power cells could mimic the process of photosynthesis
\cite{ColliniScience09, SarovarNatPhys10}.  Probably the most far-reaching
example, and the main focus of this thesis is the field of quantum information
processing, where logical operations are performed by manipulating the quantum
system on which the information is encoded \cite{NielsenChuang}.

%%%%%%%%%%%%%%%%%%%%%%%%%%%%%%%%%%%%%%%%%%%%%%%%%%%%%%%%%%%%%%%%%%%%%%%%%%%%%%%%
\section{Coherent Control of Quantum Systems}

A common thread through the newly developing quantum technologies is the
concept of \emph{coherent control}.
\index{control!coherent}
The control problem can be formulated as
follows: Given a quantum system in a well-defined initial state, and a
Hamiltonian
\begin{equation}
  \Op{H} = \Op{H}_0 + \Op{H}_c[u(t)]
\end{equation}
that includes a control parameter $u(t)$, which choice of $u(t)$
ensures that the system evolves to some desired target state, or implements
a desired process? Instead of a single control $u(t)$, we might also have a set
of multiple control parameters $\{u_i(t)\}$.

For example, the interaction of an electronic or nuclear spin with a magnetic
field $\vec{B}(t)$ is described by the Hamiltonian
\begin{equation}
  \Op{H} = - \gamma \left(
                \Op{S}_x B_x(t)
              + \Op{S}_y B_y(t)
              + \Op{S}_z B_z(t)
              \right)\,,
\end{equation}
where $\gamma$ is the gyromagnetic ratio,
$\Op{S}_{x} = \frac{\hbar}{2} \Op{\sigma}_x$,
$\Op{S}_{y} = \frac{\hbar}{2} \Op{\sigma}_y$, and
$\Op{S}_{z} = \frac{\hbar}{2} \Op{\sigma}_z$
are the operators measuring the spin in the three spatial
directions, proportional to the Pauli matrices
\begin{equation}
  \Op{\sigma}_x = \begin{pmatrix}
    0 & 1 \\
    1 & 0
  \end{pmatrix}\,,
  \quad
  \Op{\sigma}_x = \begin{pmatrix}
    0 & -\ii \\
    \ii & 0
  \end{pmatrix}\,,
  \quad
  \Op{\sigma}_x = \begin{pmatrix}
    1 & 0 \\
    0 & -1
  \end{pmatrix}\,.
\end{equation}
\index{Pauli matrices}
The components of the magnetic field $B_x(t)$, $B_y(t)$, $B_z(t)$
are the control parameters, and we may ask which magnetic field will bring an
arbitrary initial state $\Ket{\Psi_{0}(t=0)}$ to an arbitrary target state
$\Ket{\Psi^{\tgt}(t=T)}$ at some final time $T$.
On a larger scale, such a control of spin
systems forms the basis of next-generation medical imaging technology or
spectroscopy of complex molecules, among other applications.
In the same way, we can ask which laser field will drive the electronic states
of atoms to bind them into ultra-cold molecules \cite{TomzaPRA2012}, which
voltages to the electrodes of an ion-trap will transport an ion over a large
distance \cite{FuerstNJP2014}, or which variation of the confinement potential of
a Bose-Einstein will split the condensate's wave function \cite{JaegerPRA14}.

In all cases, the solution to the control problem relies on the well-defined
phase relation between the basis states of the Hilbert space, exploiting the
interference between multiple pathways.  This is what defines the control as
\emph{coherent} and thus puts the control problem entirely in the quantum
domain.

The control problem is not limited to simple state-to-state transformations, but
can equally apply to the implementation of a desired unitary evolution $\Op{O}$
on an $N$-dimensional Hilbert space.
In this case, we start from a set of basis states $\{\Ket{\phi_{1}},\dots,
\Ket{\phi_N}\}$ and attempt to find a single control that maps each of
the basis states $\Ket{\phi_j}$ to $\Op{O} \Ket{\phi_j}$:
\begin{equation}
  \forall j: \quad
  \Op{U}(T,0;u(t)) \Ket{\phi_j} \stackrel{!}{=} \Op{O} \Ket{\phi_j}\,
  \label{eq:tgt_forall_j}
\end{equation}
where $\Op{U}(T, 0; u(t))$ is the time evolution operator induced by the
Hamiltonian $\Op{H}[u(t)]$ from $t=0$ to $t=T$.
\index{time evolution operator}
Note that the control $u(t)$ is
the same for each of the basis states. If $\{\Ket{\phi_j}\}$ is a complete
basis, then any state $\Ket{\Psi}$ can be expanded as $\Ket{\Psi} = \sum_{j}
\alpha_j \Ket{\phi_j}$, and thus we have implemented $\Op{U} \Ket{\Psi} = \Op{O}
\Ket{\Psi}$ for an arbitrary state. This type of control problem is especially
relevant to quantum computation where $\Op{O}$ is a \emph{quantum gate}.

We can go further and formulate the control problem on a more mathematical basis
by defining a functional $J_T\left(\{\Ket{\Psi_j(T)}\}\right)$ that becomes
minimal if (and only if) the control implements the desired target. For example,
for a state-to-state transition, one possible choice is
\begin{equation}
\begin{split}
  J_{T,\text{ss}}(\Ket{\Psi(T)})
  &
  = 1 - \Abs{%
         \Braket{\Psi(t=0) | \Op{U}^\dagger(T,0;u(t)) | \Psi^{\tgt}(t=T)}
        }^2
  \\ &
  = 1 - \Abs{\Braket{\Psi(T) | \Psi^{\tgt}(T)}}^{2}\,,
  \label{eq:J_T_ss}
\end{split}
\end{equation}
where $\Op{U}(T,0;u(t))$ is again the time evolution operator; $\Op{U}^\dagger$ acts
to the left in Eq.~\eqref{eq:J_T_ss}, as $\Ket{\Psi(t)}
= \Op{U}(t,0;u(t))\Ket{\Psi(t=0)}$.
The task has now become an \emph{optimization problem} of finding the $u(t)$
that minimizes $J_T$. The time evolution operator places an implicit constraint
on the optimization, in that the state $\Ket{\Psi(t)}$ must be a solution to the
proper equation of motion, usually the Schrödinger equation:
\index{Schrödinger equation!time-dependent}
\begin{equation}
  \ii \hbar \frac{\partial}{\partial t} \Ket{\Psi(t)} = \Op{H}[u(t) \Ket{\Psi(t)}\,.
  \label{eq:tdse_intro}
\end{equation}
The methods for solving this constrained optimization problem constitutes the
framework of \emph{optimal control}.
\index{control!optimal}

In simple cases, the solution to the control problem can be derived
analytically. In the above example of the two-level system of a single
spin, we can find a solution simply by solving the Schrödinger equation. We
write the initial state $\Ket{\Psi}$ in its Feynman-Vernon-Hellwarth (FVH)
representation \index{Feynman-Vernon-Hellwarth representation}
\cite{TannorBook, FeynmanJAP1957}, a three component vector $\vec{r}$ obtained
from projecting the density operator $\Op{\rho} \equiv \Ket{\Psi}\!\Bra{\Psi}$
onto the three Pauli matrices
\begin{equation}
  r_i = \Tr\left[\Op{\sigma}_i \Op{\rho}\right]\,.
  \label{eq:bloch_vector}
\end{equation}
The pulse is also represented as a vector $\vec{\Omega}$ with the three
components $B_x(t)$, $B_y(t)$, $B_z(t)$, and the FVH-vector now simply
precesses around $\vec{\Omega}$,
\begin{equation}
  \dot{\vec{r}} = \vec{\Omega}(t) \times \vec{r}\,.
  \label{eq:bloch_precession}
\end{equation}
The optimal solution for a state transfer from $\vec{r}_0(t=0)$ to
$\vec{r}_{\tgt}(t=T)$ is easy to see: choose $\vec{\Omega}$ such that it points
to the center of the geodesic circle connecting $\vec{r}_0$ and
$\vec{r}_{\tgt}$, and switch the magnetic field to an arbitrary constant
amplitude until the target state is reached.

More generally, analytical solutions are usually based on a direct application
of Pontryagin's maximum principle \cite{PontryaginBook},
\index{Pontryagin maximum principle}
a generalization of the Euler-Lagrange equation familiar from the variational
calculus of classical mechanics.
For spin systems, a number of non-trivial control problems have been solved
using a technique named \emph{geometric control},
\index{control!geometric}
such as the implementation of quantum gates \cite{KhanejaPRA2001}, dissipative
state-to-state transfer \cite{LapertPRL2010}, and control under inhomogeneous
magnetic fields \cite{AssematPRA2010}.

Usually, an analytic solution to the control problem is only possible as long as
the system or the optimization functional is reasonably ``simple''. One problem
is that the analytic solutions are often of \emph{bang-bang} type, i.e., they
switch instantaneously between zero and some constant amplitude. This type of
control would not be compatible with constraints on the smoothness or spectral
width of the control, reflecting that such controls may not be realistic. For
example, a laser field cannot be switched instantaneously.

Furthermore, analytical solutions are usually restricted to Hilbert spaces of very
small dimension. A standard approach is to try to derive a
reduced effective model of a physical system that is valid in a certain
parameter regime. For example, in a two-photon transition between two level via
one or more intermediary levels, the intermediate levels can be
\emph{adiabatically eliminated} if both driving fields are far off-resonant
\index{adiabatic elimination}
from the intermediary levels and if the pulse shapes are slowly varying. Another
example where the dynamics can be effectively limited to a subset
of the full system is population-transfer in a $\Lambda$-shaped
three-level system via the popular STIRAP mechanism
\cite{VitanovARPC2001}
\index{Stimulated Raman Adiabatic Passage (STIRAP)}
In both of these examples, the approximations only hold under severe
restrictions, specifically that the dynamics
must be \emph{adiabatic}, i.e., that the system is always in an instantaneous
eigenstate of the time-dependent Hamiltonian.
\index{adiabaticity}
This is only true if the controls vary on a sufficiently slow time scale, which
is in direct contradiction to the requirement to implement the target process
in as little time as possible.
In chapter~\ref{chap:robust}, we will see the limitations of such techniques in
the context of a quantum gate optimization.

To circumvent the limitations of analytical optimal control, we turn to
\emph{numerical optimal control}
\index{control!numerical}
which takes a different approach.  Starting from a set of sub-optimal
``guesses'', the control fields are
iteratively improved: The equation of motion is solved numerically, allowing
to evaluate the functional, and to modify each
$u_i(t)$ such that the value of the optimization functional decreases. This new
optimized set of controls is then used as the guess for another iteration.
Eventually, assuming there are no local minima in the optimization landscape,
the procedure will converge towards an optimal solution.

Numerical optimal control theory (OCT) holds the promise of providing
an extremely versatile tool for designing the controls necessary in a new
generation of quantum technology, going beyond simple models that can be
approached analytically, and instead meeting the demands imposed by real-life
implementations. Fundamentally, the method is only limited by the available
computational resources.
This does not imply that OCT can provide a complete black-box solution. The
efficient numerical simulation of the system dynamics relies on an suitable
model that identifies the relevant degrees of freedom.
Moreover, the iterative nature of the control scheme makes it necessary to start
from a reasonably well-chosen guess pulse in order to converge quickly.
Designing such a guess pulse still requires a thorough understanding of
least some possible control mechanisms. Lastly, the optimized pulses
resulting from OCT can be extremely complex, sometimes making their physical
implementation a challenge. From a theorist's perspective, understanding
the mechanisms employed by the pulse can be a non-trivial task, but it provides
the chance for an important interplay between analytical and numerical
solutions. Ideally, entirely new control mechanisms can be identified, or OCT
can find the most suitable out of several possible strategies, and the gained
knowledge can go into the design of better guess pulses, or even new analytic
schemes.  Thus, OCT does not substitute for a deep understanding of the physics
of a given quantum system, but it augments the framework of coherent control by
an important and powerful tool.

\enlargethispage{\baselineskip}
The methods of optimal control, both analytic and numerical, originate from
extensive work in mathematics and engineering
\cite{BellmanBook, PontryaginBook, NocedalBook, BoydBandenberghe, KrotovBook}.
In the context of quantum systems, these ideas were first applied
to the control molecular interactions \cite{RiceZhao, BrumerShapiro}, and later
to the control of spins in nuclear magnetic resonance \cite{SkinnerJMR2003,
KhanejaJMR05, TosnerJMR2009} as well as to a wide range of other quantum systems
\cite{BrifNJP2010}.
Some of the central algorithms used in numerical optimal control of quantum
systems are presented in chapter~\ref{chap:numerics}.


%%%%%%%%%%%%%%%%%%%%%%%%%%%%%%%%%%%%%%%%%%%%%%%%%%%%%%%%%%%%%%%%%%%%%%%%%%%%%%%%
\section{Quantum Computation}
\label{sec:intro_qc}

Quantum computation \cite{NielsenChuang} is a particularly far-reaching and
exciting type of quantum technology. As microchips become smaller and smaller,
following Moore's law \cite{MooreE1965}, they will reach sizes at which quantum
effects are dominant.  At this point, in order to achieve further improvements
in computing power, a paradigm shift is necessary, where quantum effects are no
longer seen as a perturbation to classical electronic circuits, but are actively
involved in the computational process.

The motivation for quantum computers is
more fundamental than that, however. The ``weirdness'' of quantum mechanics
promises radically more powerful ways of processing information
\cite{FeynmanIJTP1982}.
A key observation is that quantum mechanics cannot efficiently be simulated on
a classical computer~\cite{FeynmanIJTP1982}, giving the motivation for quantum
computing was to use one quantum system to simulate the behavior of another.
\index{quantum simulator}
This is possible because the theoretical framework of quantum mechanics is
entirely abstract from the diverse range of systems it models. Therefore,
a quantum system that is easily engineered and controlled in the lab could stand
in for another system that is less accessible. The far-reached applications of
a quantum simulator typically include open problems in solid states
physics, such as high-temperature superconductivity or quantum phase
transitions~\cite{SchaetzNJP2013, JohnsonEPJQT2014}, but extend to other fields
such as high-energy physics or cosmology as well~\cite{GeorgescuRMP2014}.

The reason that a classical computer cannot efficiently simulate a large quantum
system originates from the feature that distinguishes quantum states from
classical states: \emph{quantum superposition} and \emph{entanglement}. Whereas
a register of $n$ classical bits encodes exactly one of $2^n$ possible values
composed of $n$ binary digits $b_i$, $\text{reg} = \left| b_0 \dots b_n \right|$,
the quantum version of this register is described by a quantum state
\begin{equation}
\Ket{\Psi} = \sum_{i=0}^{2^n-1} a_{i} \,
              \left(\Ket{q_0^{(i)}} \otimes
                  \dots \otimes \Ket{q_{n-1}^{(i)}} \right)\,,
\end{equation}
with $q_j^{(i)} \in \{0,1\}$, i.e., a simultaneous superposition of all the
$2^n$ eigenstates of the $n$-quantum-bit-register, with complex coefficients $a_i$.
The amount of information required to describe the state of a quantum system
grows exponentially with the size of the system, but grows only linearly for
a classical system. This raises the question whether more generally, a quantum
system could also implement a \emph{universal} computer that would be
fundamentally more powerful than classical computers.

In theoretical computer science, the question of computability is addressed in
the model of the Turing machine~\cite{TuringPLMS1937} as a universal computer,
i.e., one that can compute any computable function.
As an important prerequisite to the possibility of a universal quantum computer,
a quantum version of such a Turing machine has been shown~\cite{DeutschPRSA1985}
to be at least as powerful as its classical counterpart, and to solve
at least some problems with greater efficiency~\cite{KayeLaflammeMosca}.

In practice, the concept of a universal quantum computer takes its inspiration
from the standard digital computer, implemented as a network of electronic
logical gates on a microchip. Such a computer can perform any manipulation of
bits in its memory. Similarly, a gate-based model of a quantum computer,
reviewed in chapter~\ref{chap:quantum} is implemented as a network of quantum
gates that perform any operation on the quantum bits in its memory.
Compared to classical circuits, there are some caveats imposed by the laws of
quantum mechanics~\cite{NielsenChuang}: the quantum network has to be
\emph{unitary}, i.e., fully reversible and the copying of
a quantum state is not allowed~\cite{WoottersN1982}.

At first glance, quantum superposition might seem to make it obvious that
a quantum computer could solve problems at exponentially greater efficiency
than a classical computer. In a quantum analogue of a classical circuit, one could
simply prepare the quantum superposition of all possible inputs, and the circuit
would simultaneously calculate all possible results, in a form of \emph{quantum
parallelism}.
\index{quantum parallelism}
Unfortunately, the resulting quantum state has to be measured in order to read
it out, collapsing the superposition of results to one result at random.
Designing quantum algorithms in which some of the computational power resulting
from quantum parallelism survives the measurement process requires great
ingenuity.
The most striking case in which such an algorithm has been devised is Shor's
method of factoring numbers~\cite{ShorSJC1997}.
\index{Shor's algorithm}
Given an integer that is the
product of two large prime numbers, the best classical algorithm to find the two
factors runs in near exponential time~\cite{PomeranceNAMS1996}, whereas Shor's
algorithm can perform the same task in polynomial time on a quantum computer.

In general, the understanding of all the ways quantum algorithms could outperform
classical algorithms is incomplete.  An underlying theme,
however, is that it can be possible to obtain some \emph{global}
property, i.e., a property shared by \emph{all} states in the
superposition. The question then becomes how to cleverly manipulate the input of
the quantum computer such that the resulting output has some interesting global
property, which will survive the measurement. This is exactly what Shor's
algorithm exploits. It turns out~\cite{MillerJCSS1976} that finding the prime
factors of a large integer $N$ can be mapped to finding the period of the function
\begin{equation}
  f(x) = a^{x} \mod N
\end{equation}
for random integers $a<N$. The period is found using a quantum Fourier transform
\index{Fourier transform!quantum}
that maps every eigenstate \Ket{i} of an $n$-dimensional Hilbert space as
\begin{equation}
  \Ket{i} \rightarrow
  \frac{1}{\sqrt{n}} \sum_{j=0}^{n-1}
  \exp\left[ 2 \pi \ii \frac{i j}{n} \right] \, \Ket{j}\,.
\end{equation}
This is analogous to the classical Fourier transform,
cf.~appendix~\ref{AppendixFFT}; as in the classical
case, each value in the spectrum gives a global property of the original
function. In the case of the quantum Fourier transform, a measurement has the
greatest likelihood for yielding the dominant frequency component of $f(x)$, and
thus ultimately solves the factoring problem.

The factoring of large integers is a highly relevant problem, because the
security of one of the most widely used encryption systems, the RSA
\index{RSA}
algorithm~\cite{RivestCACM1978}, relies it. Having a machine that could factor
large integers efficiently would instantly break all such encryption.

Beyond the quantum Fourier transform and the algorithms derived from it, there
is currently one other class of quantum algorithms that are fundamentally faster
than their classical counterparts. The core problem is finding one
specific value from a list of $N$ non-sorted entries. In the classical case,
there is no better solution than to look at all $N$ entries of the list,
stopping when the target element is found. Quantum mechanically, Grover's
quantum search algorithm~\cite{GroverPRL1997} can perform the same task in only
$\sqrt{N}$ steps.
\index{Grover's algorithm}
\index{quantum search}

While Grover's algorithm provides a less impressive advantage over the classical
algorithm, its applications are much farther reaching. There is a large class of
``hard'' problems where there is no way to find a solution in polynomial time,
but it is easy to verify that a given solution is the correct one.
In complexity theory, such problems belong to the category
``NP''~\cite{AroraBook}. These include many famous graph-theoretical
optimization problems, such as versions of the traveling salesman
problem~\cite{LawlerTSPBook} or the graph coloring
problem~\cite{JensenGraphColorBook}.
In the worst case, an NP problem can be solved by
enumerating all $N$ possible solutions and selecting the desired one, where $N$
depends exponentially on the size of the problem. Using Grover's algorithm in
such a case would give a very significant speedup $N \rightarrow \sqrt{N}$, even
though the solution remains exponential in principle.

The underlying idea of Grover's algorithm is that since a solution can easily be
identified, it is possible to implement an \emph{oracle}
\index{oracle}
quantum gate that selectively flips
the phase of the eigenstate associated with the correct solution. Once the
eigenstate has been ``tagged'' in this way in a superposition of all
eigenstates, the \emph{amplitude amplification}
\index{amplitude amplification}
technique~\cite{GroverPRL1998, BrassardISTCS1997} is employed, which increases
the relative amplitude of the tagged
state. This amplification has to be repeated on the order of $\sqrt{N}$ time
before the target state dominates over all other eigenstates, and can be
measured with high likelihood. Essentially, the quantum search algorithm finds
a needle in a haystack by increasingly growing the needle and shrinking the
haystack, until the needle can be easily picked out by a blind grab.

There are a wide variety of quantum systems that have been considered to
implement such a universal quantum computer~\cite{NielsenChuang}, from nuclear
spins~\cite{CoryPD1998}, to quantum dots~\cite{ImamogluPRL1999}, to trapped
ions~\cite{CiracPRL95} to nitrogen-vacancy centers~\cite{NizovtsevOS2005} -- to
name just a few examples. No implementation to date has overcome all the
technical challenges required to implement a working, universal, large-scale
quantum computer.
In this thesis, two systems in particular are used as examples, trapped Rydberg
atoms, reviewed in section~\ref{sec:RydbergImplemenation} of
chapter~\ref{chap:robust}, and superconducting circuits, reviewed in
chapter~\ref{chap:transmon}. The latter are particularly promising candidates
for quantum computing, since they share they share many of their techniques with
present day classical computer chips and can be engineered with great
versatility.

There are some aspects of using quantum systems to store and process information
that go beyond the ideas of universal quantum computers. With the cryptographic
applications of the most prominent quantum algorithms threatening the security
of existing encryption schemes, quantum mechanics has also provided a possible
answer, in the form of \emph{quantum communication} and \emph{quantum
cryptography}~\cite[and references therein]{GisinNP2007, GisinRMP2002}.
\index{quantum communication}
\index{quantum cryptography}
Information is stored in the quantum states of a photon, which can easily be
transmitted through optical fibers. Since unknown quantum states cannot be
copied~\cite{WoottersN1982}, and any measurement collapses the quantum state, it
is impossible to wiretap the optical fiber without the eavesdropping being
detected. Thus, quantum communication provides a fundamentally secure channel of
communication. Once a secure channel is established, it can be used to exchange
cryptographic keys, providing the basis for an unbreakable encryption
system~\cite{ShannonBSTJ1949}, although in practical implementations, loopholes
may still exist~\cite{GerhardtNC2011}.
Such \emph{quantum key-distribution networks}
have been successfully implemented and are commercially
available~\cite{ElliottNJP2002, StuckiNJP2011, Oesterling2012}.
\index{quantum key distribution network}

There has also been considerable interest quantum computers that do not fit into
the standard gate model, i.e., ``special-purpose'' quantum computers. One
approach is that of \emph{adiabatic quantum computing}~\cite{FarhiS2001}.
The idea is to engineer a complex Hamiltonian $\Op{H}(T)$ whose (unknown) ground
state encodes the result of a computation. The system is then initialized to the
known ground state of a simpler Hamiltonian $\Op{H}(0)$. Then, the simple
Hamiltonian is slowly (``adiabatically'') transformed in the complex result
Hamiltonian. According to the adiabatic theorem,
\index{adiabatic theorem}
the system will remain in the ground state of the evolving $\Op{H}(t)$ at all
times. In this way, the ground state of the complex Hamiltonian, and thus the
solution to the computation problem is found. Performed fully coherently, at zero
temperature, adiabatic quantum computing can be shown to be mathematically
equivalent to the standard gate model~\cite{Aharonov2004}.
A ``messier'' version at non-zero temperature, in which the qubits are strongly
coupled to their environment throughout, but still maintain some quantum
coherence is known as \emph{quantum annealing}.
\index{quantum annealing}
A quantum processor based on quantum annealing has been built and
marketed~\cite{JohnsonN2011} by the Canadian company DWave, and is to date the
only commercially available quantum information processing device (outside of
quantum communication networks). However, the DWave processor has been subject
to severe criticism~\cite{DamNP2007, RonnowS2014, ShinArXiv1401.7087},
questioning whether it provides any speedup over classical computers.
Nonetheless, applications of the architecture to machine
learning~\cite{NevenArXiv0912.0779,NevenArXiv0804.4457} are currently being
explored.

%%%%%%%%%%%%%%%%%%%%%%%%%%%%%%%%%%%%%%%%%%%%%%%%%%%%%%%%%%%%%%%%%%%%%%%%%%%%%%%%
\section{Decoherence}

The standard description of quantum mechanics \cite{SakuraiBook}
considers a quantum system to be in complete isolation.
This is almost never a realistic assumption.
Any quantum system will have some remaining interaction with its environment;
this is especially true in a control context, where simply the fact that the
system can be influenced from the outside indicates that it cannot be completely
isolated. For example, the same dipole moment that allows an atom to be
controlled with a laser beam also couples that atom to possibly unwanted
stray photons, or even the vacuum mode of the electromagnetic field, causing
spontaneous emission. Such unwanted interaction with the environment is the
source of \emph{decoherence} \cite{BreuerBook},
i.e., the loss of ``quantumness''.
ore precisely, the fixed phase relation between the eigenstates of the Hilbert
space is lost.  Since coherent control relies on exploiting the interference of
exactly these phases, decoherence is fundamentally detrimental to the objective.
For a technology built specifically on quantum features, decoherence is an
obvious challenge.

Mathematically, the effects of decoherence are well-described. Instead of
a state vector $\Ket{\Psi}$ in Hilbert space, the system is now modeled as
a density matrix $\Op{\rho}$ in Liouville space.
The density matrix formalism, presented in chapter~\ref{chap:quantum}, allows to
represent both \emph{pure states} that are equivalent to the state vectors in
Hilbert space, but also \emph{mixed states} that describe the system when
decoherence has taken place. In a closed system, the direct equivalent of the
Schrödinger equation~\eqref{eq:tdse_intro} is the Liouville-von Neumann
equation
\begin{equation}
  \ii \hbar \partdifquo{t} \Op{\rho}_{\text{closed}}
  = \left[\Op{H}, \Op{\rho}_{\text{closed}} \right]\,.
\end{equation}
\index{Liouville-von Neumann equation}
For an open system, the decoherence is expressed as a dissipator $\Dissipator$,
\index{dissipator}
which occurs as an additional term in the Liouville-von Neumann equation.
\begin{equation}
  \ii \hbar \partdifquo{t} \Op{\rho}_{\text{open}}
  = \left[\Op{H}, \Op{\rho}_{\text{open}} \right]
    + \Dissipator\left[ \Op{\rho}_{\text{open} }\right]\,.
\end{equation}
In principle, the methods of OCT still apply to the open quantum system, except
that the control functional to be minimized now has to be expressed in
Liouville space. For example, the Liouville space equivalent to
Eq.~\eqref{eq:J_T_ss} is
\begin{equation}
  J_{T,\text{ss}}(\Op{\rho})
  = 1 - \Abs{\Tr\left[ \Op{\rho}^\dagger(T) \Op{\rho}^{\tgt}\right]}^2
\end{equation}

In the presence of decoherence, the promise of optimal control is
only reinforced. One reason is that in Liouville space, it is even harder than
in Hilbert space to devise analytical control schemes. Thus, numerical tools are
often the only way to effectively tackle the problem. Examples for the use of
numerical optimal control of systems that are subject to decoherence include
laser cooling of the internal degrees of freedom in molecules, maximizing
coherences~\cite{OhtsukiJCP99}, distilling out the influence of
noise~\cite{KallushPRA06}, guiding the dynamics of a quantum
state~\cite{TremblayPRA08}, and the mitigation of Markovian dephasing noise on
single qubits~\cite{GormanPRA12}.

For the realization of large scale quantum computing, it is crucial to limit or
circumvent the effects of decoherence.
The phase relation between the states in the quantum computer is at the heart of
both the algorithms discussed in section~\ref{sec:intro_qc}. In Shor's
algorithm, the underlying quantum Fourier transform identifies global
properties of the wave function, inherently embodied in phase information. In
Grover's algorithm, the correct solution to the search is tagged by a phase flip,
allowing it to be amplified. Thus, the loss of phase information destroys all
chance for true quantum computation, and the speedup it promises over classical
computers. Optimal control provides a tool for addressing the issue of
decoherence, allowing to implement quantum gates with sufficiently high fidelity
that \emph{quantum error correction} can guarantee fault-tolerant quantum
computing.

Taking decoherence into account explicitly in the optimization allows to actively
search for solutions where its effects are minimized. Generally, not all levels
suffer equally from decoherence, and optimal control can restrict the solution to
those levels that are least affected. In chapter~\ref{chap:robust}, we consider
a case where there is significant spontaneous decay from only one of the levels,
and see how optimal control circumvents the effects of decoherence by not
populating this level. Lastly, an obvious way to counter decoherence, is to find
solutions for the control problem that act on a faster time scale than the
dissipation, so that the desired process is implemented before the decoherence
becomes too noticeable. In this context, it becomes especially important that
numerical optimal control can find the shortest possible controls, reaching what
has been called the quantum speed
limit~\cite{CampoPRL2013,LevitinPRL2009,MargolusPD1998,BhattacharyyaJPA1983,GoerzJPB11}.



%%%%%%%%%%%%%%%%%%%%%%%%%%%%%%%%%%%%%%%%%%%%%%%%%%%%%%%%%%%%%%%%%%%%%%%%%%%%%%%%
\section{Organization of the Thesis}

The work presented in this thesis explores possibilities for using numerical
optimal control to implement robust two-qubit gates in open quantum systems.
Chapter~\ref{chap:quantum} gives an overview of the theoretical foundations,
outlining the concepts of quantum computing, followed by an introduction to the
description of open quantum systems.
Chapter~\ref{chap:numerics} discusses the necessary numerical tools for the
representation and dynamical simulation of both open and closed quantum systems,
as well as algorithms for optimal control.

Chapter~\ref{chap:robust} presents a first example for the application of these
techniques to the implementation of a quantum gate for trapped neutral atoms.
Robustness with respect both to dissipation and fluctuations in technical
parameters is achieved by performing the optimization in Liouville space, and
by optimizing for an ensemble of Hamiltonians that samples over the relevant
range of fluctuations. By including the desire for robustness explicitly in the
optimization functional, optimal control can find solutions that meet the
desired objectives.

One of the most promising architectures for quantum computation is based on
superconducting circuits, reviewed in chapter~\ref{chap:transmon}. The qubits in
this implementation allow for unparalleled flexibility, where parameters
and interactions can be engineered over a wide range. Especially with
the development of the \emph{transmon} qubit,
\index{transmon}%
decoherence times have been pushed
to a regime where large scale quantum computing appears within reach. Entangling
gates on two transmons can be implemented by coupling both qubits to a shared
transmission line resonator, which mediates an effective coupling between them.
We discuss the derivation of an effective model for the qubit-qubit
interaction, which is the basis for some of the results presented in
chapters~\ref{chap:pe} and~\ref{chap:3states}. Furthermore, we review
some of the basic gate mechanisms that have been demonstrated experimentally,
before exploring the use of an off-resonant driving field to induce Stark shifts
in the levels of the logical subspace, realizing a holonomic phasegate. Finally,
we show optimization results for a CPHASE, CNOT, and the holonomic gate in the
full qubit-qubit-cavity model.

Chapter~\ref{chap:pe} illustrates the power of optimal control to exploit
any freedom allowed by the optimization functional. Instead of optimizing for
a specific quantum gate implemented on superconducting qubit, the target is an
arbitrary \emph{perfect entangler}, i.e., a gate that entangles some initially
separable states. Not restricting the optimization unnecessarily holds the
promise of allowing optimal control to identify the gates least affected by
decoherence, or the gates that can be implemented in the shortest possible time,
in the hope the decoherence will only become relevant on longer time scales.

The quadratic scaling of Liouville space compared to Hilbert space provides
significant challenges for the simulation and optimization of open quantum
systems. Chapter~\ref{chap:3states} illustrates the possibility of a significant
reduction in the resources required for optimization. Under the assumption that
the optimization target is unitary (e.g., a quantum gate), it is not necessary
to consider a full basis of Liouville space consisting of $d^2$ matrices, where
$d$ is the dimension of the underlying Hilbert space. Instead, at most 3 states
need to be included in the optimization. This is illustrated for both trapped
neutral atoms and superconducting circuits.
Finally, chapter~\ref{chap:outlook} summarizes and gives an outlook.

