% vim: ft=tex iskeyword=@,48-57,_,-,192-255,\: dictionary=bibkeys.lst,labels.lst:
\chapter{Quantum Gates with Superconducting Transmon Qubits}
\label{chap:tmintro}

Superconducting circuits have emerged has one of the most promising
environments for quantum information processing. Unlike most other
implementations (e.g.\ trapped atoms like the Rydberg gate discussed in
chapter~\ref{chap:robust}), they are macroscopic, in the sense that the relevant
degrees of freedom are the quantities describing electrical circuits, such as
charge, current, or flux. However, unlike in classical electrical circuits,
these variables are now fully quantized and described by a wave function. The
quantum behavior results from the electrons forming Cooper pairs and condensing
to a \emph{collective} quantum state as the system is cooled below some
(material-dependent) critical temperature $T_c$ \cite{TinkhamBook}.
% describe phase transition to the superconducting state

From a technical perspective, superconducting qubits have the significant
advantage of building upon fabrication techniques used in standard integrated
circuits. A superconducting layer of usually aluminum or niobium is layered on
a silicon wafer
\cite{DevoretQIP2004, FrunzioITAS2005}.
The circuit elements are then patterned using the standard tools of optical or
electron-beam lithography and chemical etching: a photo-sensitive ``resin''
material is added to the chip and the illuminated with UV light or an electron
beam through a mask. Then, the developed resin is removed and the
exposed superconducting material is etched away, or alternatively, electrical
leads are deposited in the resin gaps \cite{HirstSCHandbook2003}.
For superconducting circuits relevant to quantum computing, one of the central
components is the \emph{Josephson junction}, described below in
section~\ref{sec:jj}. It consists of two layers of superconducting material
separated by an insulating layer. This insulating layer in generally created by
oxidation of the superconducting material \cite{DevoretQIP2004,
FrunzioITAS2005}. Using these well-established production techniques,
a quantum computer based on a superconducting architecture could readily
produced industrially, provided that the more fundamental challenges in building
a large-scale quantum processor can be met, i.e.\ implementing universal quantum
gates and scaling the circuit to a significant number of qubits while
maintaining coherence.

This chapter reviews the fundamentals of superconducting qubits, with emphasis
on the \emph{transmon} design, as well as the gate mechanisms that have been
used to implement two-qubit gates using transmons. We start by explaining the
Josephson junction as the quint-essential non-linear element in
superconducting circuits. The three traditional qubit designs -- the Cooper pair
box (``charge qubit''), the RF-SQUID (``flux qubit''), and the current-biased
Josephson Junction (``phase qubit'') -- are briefly discussed before moving to
the transmon qubit, which improves upon the traditional qubits with
significantly enhanced decoherence times. Finally, some of the gate mechanism
that have been used to to implement two-qubit gates on transmon qubits are
summarized, before ending with a general discussion of the decoherence
properties of the transmon.


\section{The Josephson Junction}
\label{sec:jj}

\begin{table}
  \centering
  \begin{tabular}{ll}
  \toprule
  \multicolumn{2}{c}{junction parameters} \\
  \midrule
  $I_0$        & critical current                                     \\
  $C_j$        & junction capacitance \\
  $Q_r$        & offset charge \\
  $E_c$        & charging energy, $E_c = \frac{1}{2}\frac{(2e)^2}{C_J}$ \\
  $E_J$        & Josephson energy, $E_J = \phi_0 I_0$ \\
  $L_{J0}$     & effective inductance, $L_{J0} = \frac{\phi_0}{I_0}$ \\
  \midrule
  \multicolumn{2}{c}{dynamic quantities} \\
  \midrule
  $\Phi(t)$     & branch flux of junction, see Eq.~(\ref{eq:branchflux}); $\phi(t) \in \Real$ \\
  $I(t)$        & Josephson current, see Eq.~(\ref{eq:jjcurrent}); $I(t) \in \Real$ \\
  $\delta(t)$   & ``phase'', $\delta(t) = \frac{2\pi \Phi(t)}{\Phi_0} = \frac{\Phi(t)}{\phi_0}$; $\delta \in \Real$ \\
  $\theta(t)$   & condensate phase diff., $\theta(t) = \delta(t)\!\!\mod 2 \pi$; $\theta(t) \in [0, 2 \pi)$ \\
  $L_J(\delta)$ & Josephson inductance, $L_J(\delta) = L_{J0} \cos^{-1} \delta$ \\
  $N(t)$        & number of tunneled cooper pairs; $N(t) \in \Integer$ \\
  \midrule
  \multicolumn{2}{c}{fundamental constants} \\
  \midrule
  $\Phi_0$     & flux quantum, $\Phi_0 = \frac{h}{2e}$ \\
  $\phi_0$     & reduced flux quantum, $\phi_0 = \frac{\Phi_0}{2\pi}$ \\
  \bottomrule
  \end{tabular}
  \caption{Summary of quantities and constants used for the characterization of
  a Josephson Junction. For derived quantities, the dependence on fundamental
  quantities is given. The first group contains static parameters that depend on
  the geometric properties of the junction at production. An exception is $E_J$
  which can also be made tunable by splitting the junction in two and running an
  external flux through the resulting loop. The second group are the dynamic
  quantities from the which the tunneling energy and the
  capacitive energy of the junction derive, and which allow to formulate the
  Hamiltonian.  The ``phase'' $\delta$ is more properly known as the
  ``gauge-invariant phase difference'' across the junction. The last group
  defines the universal flux quantum.}
  \label{tab:jj_params}
\end{table}

\begin{figure}[htbp]
  \centering
  \includegraphics{figures/jj}
  \caption{Josephson Junction}
  \label{fig:jj}
\end{figure}

The Josephson junction consists of two superconducting leads separated by a thin
insulating layer, as shown in Fig.~\ref{fig:jj}. This structure acts as
a capacitor with capacitance $C_j$. From classical electrodynamics, if one of
the leads (capacitor plates) holds a charge of $Q$, the energy stored in the
capacitor is
\begin{equation}
  E_{cap} = \frac{1}{2} \frac{Q^2}{C_J}
  \label{eq:E_capacity}
\end{equation}
In addition to the capacitive effect, Cooper pairs of electrons can tunnel
through the insulating layer (owing to the quantum mechanical nature of the
circuit), resulting in a \emph{Josephson current}.
The behavior at the is governed by two equations: First, the branch flux of any
electrical element can be obtained from the voltage $V$ across that element.
\begin{equation}
  \Phi(t) = \int_{-\infty}^{t} V(t') \dd t'
  \label{eq:branchflux}
\end{equation}
Second, the Josephson junction has a characteristic non-linear relation between
the tunneling current through that junction and the branch flux
\cite{JosephsonAP1965},
\begin{equation}
  I(t) = I_0 \sin \delta(t)
  = I_0 \sin \frac{2 \pi \Phi(t)}{\Phi_0}
  = I_0 \sin \frac{\Phi(t)}{\phi_0}
  \label{eq:jjcurrent}
\end{equation}
Here, $I_0$ is the maximum current the junction can support, and $\delta(t)$ is
the phase difference of wave functions on each side of the junction, which is
directly related to the branch flux $\Phi(t)$ of the junction element. All the
quantities typically used to describe the junction are listed in
Table~\ref{tab:jj_params}.

If the junction holds an initial charge difference (``offset charge'') of $Q_r$
and $N$ Cooper pairs of electrons (charge $2e$) tunnel through the junction, the
capacitive energy of the junction according to Eq.~(\ref{eq:E_capacity}) is
\begin{equation}
  E_{cj} = \frac{1}{2} \frac{(2e)^2}{C_J} \left( N - \frac{Q_r}{2e} \right)^2
\end{equation}
Note that the offset charge $Q_r/2e$ will generally be much larger then $N$, and
cannot easily be controlled during the production process.
The energy due to the tunneling current is given by
\begin{equation}
  E_{Ij} = \int_{-\infty}^{t} I(t') V(t') dt' = -E_J \cos \delta\,,
\end{equation}
% TODO explain
with $E_J$ defined in Table~\ref{tab:jj_params}. Together, both terms yield the
Hamiltonian
\begin{equation}
  \Op{H}_{jj}
  = E_c \left( \Op{N} - \frac{Q_r}{2e} \right)^2
    - E_{J} \cos \Op{\theta}\,.
 \label{eq:H_JJ}
\end{equation}
% TODO canonical variables

\section{Traditional Superconducting Qubits}

The Hamiltonian in Eq.~(\ref{eq:H_JJ}) is dominated by the value of $Q_r$,
obtains a random value during the production process and makes the naked
Josephson Junction ill-suited as a qubit. To obtain a well-defined qubit, there
are three ``traditional'' superconducting qubit designs that add further
elements to the Josephson-Junction circuit: the Cooper pair box (``charge
qubit''), the flux qubit, and the phase qubit.

\begin{equation}
  \Op{n} e^{i \Op{\phi}}
  = i \frac{\partial}{\partial\Op{\phi}} e^{i \Op{\phi}}
  = e^{i \Op{phi}} \left( \Op{N} + 1 \right)
\end{equation}

\begin{equation}
  \Op{H} = 4 E_C \left( \Op{N} - N_r \right)
          - \frac{E_J}{2} \sum_n \left(
            \KetBra{n}{n+1} + \KetBra{n+1}{n}
          \right)
\end{equation}


\section{The Transmon Qubit}

The transmon qubit is an improvement over the Cooper pair box (``charge
qubit''), \cite{JKochPRA07}

\section{Gate Mechanism}

\subsection{Direct Resonant iSWAP Gate}
\cite{DewesPRL2012}

\subsection{Higher-Level Resonance-Induced Dynamical CPHASE gate}
\cite{DiCarloN09}

\subsection{Resonator-Sideband-Induced iSWAP Gate}
\cite{LeekPRB2009}

\subsection{Cross-Resonance CNOT Gate}
\cite{ChowPRL2011}

\subsection{Bell-Rabi Single-Step-Entanglement Gate}
\cite{PolettoPRL2012}

\subsection{Microwave-Activated CPHASE Gate}
\cite{ChowNJP2013}

\subsection{Resonantor-Induced CPHASE Gate}

\section{Dissipation}
