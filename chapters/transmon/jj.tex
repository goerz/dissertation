\documentclass{diss}
\usepackage{mymacros}

\usepackage{tikz,pgflibraryshapes}
\usetikzlibrary{arrows, calc, decorations.pathmorphing,circuits.ee.IEC}
\usetikzlibrary{shapes.misc, fit}

\usepackage[psfixbb,graphics,tightpage,active]{preview}
\PreviewEnvironment{tikzpicture}

\begin{document}
%%%%%%%%%%%%%%%%%%%%%%%%%%%%%%%%%%%%%%%%%%%%%%%%%%%%%%%%%%%%%%%%%%%%%%%%%%%%%%%%

\begin{tikzpicture}[
  circuit ee IEC,
  x=1cm,
  y=1cm,
  set resistor graphic=var resistor IEC graphic,
  set make contact graphic= var make contact IEC graphic,
]
%\draw[step=1.0,gray,very thin] (0,-3) grid +(12,6); % background grid

\tikzset{circuit declare symbol = jj}
\tikzset{%
  set jj graphic={%
    draw,
    shape=rectangle,
    append after command={%
      % http://tex.stackexchange.com/questions/73164/how-to-have-a-cross-out-rectangle-in-tikz
      node [
          fit=(\tikzlastnode),
          draw=black, inner sep=-\pgflinewidth, cross out
      ] {}
    },
    minimum size=5mm
  }
}

\tikzset{cross/.style={cross out,
  draw=black, minimum size=2*(#1-\pgflinewidth), inner sep=0pt, outer sep=0pt},
  cross/.default={1pt}
}

%\tikzstyle{every node}=[font=\scriptsize]


%%%%%%%%%%%%%%%%%%%%%%%%%%%%%%%%%%%%%%%%%%%%%%%%%%%%%%%%%%%%%%%%%%%%%%%%%%%%%%%%
% Circuit
\begin{scope}[xshift=0.0, yshift=0]
  \node [contact] (contact-top) at (0,0.2) {};
  \node [contact] (contact-bottom) at (0,-0.2) {};
  \coordinate (circ-bottom-left)  at (0,-1.5);
  \coordinate (circ-bottom-right) at (1.5,-1.5);
  \coordinate (circ-top-right)    at (1.5, 1.5);
  \coordinate (circ-top-left)     at (0, 1.5);
  \draw (contact-bottom) -- (circ-bottom-left) -- (circ-bottom-right)
        to [jj] (circ-top-right)
        -- (circ-top-left) -- (contact-top);
\end{scope}
%%%%%%%%%%%%%%%%%%%%%%%%%%%%%%%%%%%%%%%%%%%%%%%%%%%%%%%%%%%%%%%%%%%%%%%%%%%%%%%%


%%%%%%%%%%%%%%%%%%%%%%%%%%%%%%%%%%%%%%%%%%%%%%%%%%%%%%%%%%%%%%%%%%%%%%%%%%%%%%%%
% Layers
\begin{scope}[xshift=4.5cm, yshift=2.5cm, rotate=-90]

  \coordinate (left-anchor) at (0,0); % left-bottom point of left aluminum layer
  \def\LayerThickness{0.45}; % Thickness of all layers
  \def\AlLengthLeft  {2.5};  % Length of left aluminum layer
  \def\AlLengthRight {2.0};  % Length of right aluminum layer (at bottom)
  \def\AlOverlap     {1.75};  % Length  of top aluminum layer
  \def\SlopeLength   {0.2};  % For sloped edges, horizontal difference between bottom and top of layer

  \pgfmathsetmacro\LayersWidth{\AlLengthLeft+\LayerThickness+\AlLengthRight};
  \coordinate (right-anchor) at ($ (left-anchor) + (\LayersWidth, 0) $); % right-bottom point of right aluminum layer

  % left aluminum layer
  \draw (left-anchor) -- ++(\AlLengthLeft, 0) -- ++(0, \LayerThickness)
        -- ++(-\AlLengthLeft, 0);
        %-- cycle;

  % right aluminum layer
  \draw (right-anchor) -- ++($ (-\AlLengthRight, 0) $)
        -- ++($ (0, {2*\LayerThickness}) $)
        -- ++($ ({-\AlOverlap+\LayerThickness-\SlopeLength}, 0) $)
        -- ++(\SlopeLength, \LayerThickness) -- ++(\AlOverlap, 0)
        -- ++($ (0, {-2*\LayerThickness}) $)
        -- ++($ ({\AlLengthRight-\LayerThickness}, 0) $);
        %-- cycle;

  % insulating layer
  \draw[fill=gray] (left-anchor)++(\AlLengthLeft, 0)
        -- ++(\LayerThickness, 0)                                     % right
        -- ++($ (0, 2*\LayerThickness) $)                             % up
        -- ++($ ({-\AlOverlap+\LayerThickness-\SlopeLength}, 0) $)    % left
        -- ++($ ({-1*\SlopeLength}, -\LayerThickness) $)              % down (slope)
        -- ++($ ({2*\SlopeLength+\AlOverlap-2*\LayerThickness}, 0) $) % right
        -- ++(0, -\LayerThickness);

  % Labels
  \draw (left-anchor) + ($ (\LayerThickness, 0.5*\LayerThickness) $)
       node {Al};
  \draw (right-anchor) + ($ (-\LayerThickness, 0.5*\LayerThickness) $)
       node {Al};

  \draw (left-anchor) + ($ (\AlLengthLeft+0.5*\LayerThickness, \LayerThickness) $)
       node {AlO};

\end{scope}
%%%%%%%%%%%%%%%%%%%%%%%%%%%%%%%%%%%%%%%%%%%%%%%%%%%%%%%%%%%%%%%%%%%%%%%%%%%%%%%%

%%%%%%%%%%%%%%%%%%%%%%%%%%%%%%%%%%%%%%%%%%%%%%%%%%%%%%%%%%%%%%%%%%%%%%%%%%%%%%%%
% Layers
\begin{scope}[xshift=8.0cm, yshift=0, rotate=0]
  \node [contact] (contact-top)    at (0,1.25) {};
  \node [contact] (contact-bottom) at (0,-1.25) {};

  \draw (contact-bottom) -- +(0, -0.5);
  \draw (contact-top)    -- +(0,  0.5);

  \draw (contact-bottom) -- ++(0.75, 0)  % right
                         -- node[midway, cross=4pt]{}  node[midway, right=2pt]{$I_0$}
                            ++(0, 2.5)    % up
                         -- ++(-1.5, 0) % left
                         to[capacitor={info=$C_J$}] ++(0, -2.5) % down
                         -- (contact-bottom);
\end{scope}
%%%%%%%%%%%%%%%%%%%%%%%%%%%%%%%%%%%%%%%%%%%%%%%%%%%%%%%%%%%%%%%%%%%%%%%%%%%%%%%%


\node at (6.5,0) {$\hat{=}$};
\draw[dashed] (2,0.33)  -- +(50:3.0);
\draw[dashed] (2,-0.33) -- +(-50:3.0);
\draw[dashed] (1.5, 0) circle[radius=0.601];

\end{tikzpicture}
%%%%%%%%%%%%%%%%%%%%%%%%%%%%%%%%%%%%%%%%%%%%%%%%%%%%%%%%%%%%%%%%%%%%%%%%%%%%%%%%

\end{document}
