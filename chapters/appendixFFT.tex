\chapter{Applications of the Fast-Fourier-Transform}
\label{AppendixFFT}

The Fast-Fourier-Transform~\cite{BrighamFFTBook} is a versatile numerical
tool. It is an efficient implementation of the discrete
Fourier transform that maps between a function in a canonical variable and
a function in the conjugate variable, e.g.\
space ($x$) and wavenumber ($k$, equal to momentum with $\hbar=1$),
or time ($t$) and angular frequency ($\omega$). For these two examples, the
discrete Fourier transform $F$ of a function $f$ reads as
\begin{align}
  F(\omega_j)
  &= \sum_{n=0}^{N-1} f(t_n) \ee^{-\ii \frac{2\pi j n}{N}}
  \approx \sum_{n=0}^{N-1} f(t_n) \ee^{-\ii \omega_j t_n}\,,
  \\
  F(k_j)
  &= \sum_{n=0}^{N-1} f(x_n) \ee^{-\ii \frac{2\pi j n}{N}}
  \approx \sum_{n=0}^{N-1} f(x_n) \ee^{-\ii k_j x_n}\,,
\end{align}
where $N$ is the sampling size, and using the correspondence
\begin{equation}
  w_j t_n \approx \frac{2 \pi j n}{N}\,,
  \qquad
  k_j x_n \approx \frac{2 \pi j n}{N}\,.
  \label{eq:discrete_to_cont_FTT}
\end{equation}

This transformation is widely implemented in numerical libraries as
$F(\omega_j) \equiv \FFT f(t_n)$ with the inverse transform defined by
\begin{equation}
  f(t_n) = \frac{1}{N} \FFT^{-1} F(\omega_j)
         = \frac{1}{N} \FFT^{-1} \FFT f(t_n)\,.
\end{equation}
The FFT scales as $N \log N$ with the sampling size~\cite{DuhamelSP1990}.

\section{The Frequency Grid}

When using the output of the FFT routine, e.g.\ as the spectrum of
a time-dependent signal $f(t_n)$, it is important to understand exactly which
angular frequency values $\omega_j$ the resulting amplitudes $F(\omega_j)$
correspond to. This differs between odd and even $N$, a detail that is often
neglected and can lead to subtle errors in numerical calculations.

If the original signal is of duration $T = t_{N-1} - t_0$, and has a sampling
rate $dt = \frac{T}{N-1}$, the result of a call to the FFT routine is
an array of $N$ complex numbers, containing the amplitudes for angular
frequencies between $-\omega_{\max}$ and $+\omega_{\max}$, with
\begin{equation}
  \omega_{\max} = \begin{cases}
     \frac{\pi}{dt} = \frac{(N-1)\pi}{T}
                      & \text{if $N$ even} \\
     \frac{N-1}{N} \frac{\pi}{dt} = \frac{N-1}{N} \frac{(N-1)\pi}{T}
                      & \text{if $N$ odd}
                  \end{cases}
\end{equation}

The layout of the frequency array also depends on whether $N$ is odd or even. In
any case, the frequency array consists of two parts: the first sub-array of
length $l$ contains the amplitudes of the positive frequencies, the remaining sub-array
of length $N-l$ (running from $l+1$ to $N$) contains the amplitudes for the
negative frequencies.
\begin{itemize}[noitemsep,nolistsep]

  \item $N$ even.

  For even $N$, there are $l=N/2$ positive frequencies, and the values
  correspond to $$0, d\omega, \dots, \omega_{\max}-d\omega,$$
  followed by $$-\omega_{\max}, -\omega_{\max}+d\omega, \dots, -d\omega.$$

  \item $N$ odd.

  For odd $N$, there are $l=N/2 + 1$ positive frequencies, and the values
  correspond to $$0, d\omega, \dots, \omega_{\max},$$ followed by
  $$-\omega_{\max}, -\omega_{\max}+d\omega, \dots, -d\omega.$$
\end{itemize}

The spectral resolution is
\begin{equation}
  d\omega = \begin{cases}
      \frac{2 \omega_{\max}}{N} = \frac{2\pi}{T} - \frac{2\pi}{N T}
            & \text{if $N$ even} \\
      \frac{2 \omega_{\max}}{N-1} = \frac{2\pi}{T}
            & \text{if $N$ odd}
        \end{cases}\,.
\end{equation}
With $t_n  = n \frac{T}{N}$ and $\omega_j = j \, d\omega$,
Eq.~\eqref{eq:discrete_to_cont_FTT} is recovered.  If $f(t) \in \Real$, then $F(\omega) = F(-\omega)^{*}$. For complex signals, on the other hand, the positive and the negative part of the spectrum are \emph{not} equivalent.
\section{Derivatives and the Kinetic Operator}

The FFT can be used to calculate the derivative of a signal
$f(x)$ evenly sampled at $N$ points.
\begin{equation}
  \partdifquo{x} \equiv \frac{1}{N} \FFT^{-1} \left( \ii k \right) \FFT
\end{equation}
The kinetic operator in one Cartesian dimension is
\begin{equation}
  \Op{T} = \frac{\Op{p}^2}{2m}
         = -\frac{\hbar}{2m} \frac{\partial^2}{\partial x^2}\,.
\end{equation}
Using the FFT, this becomes
\begin{equation}
\begin{split}
  \Op{T}&= -\frac{\hbar^2}{2m} \frac{1}{N} \FFT^{-1} \ii k
             \FFT \frac{1}{N} \FFT^{-1} \ii k \FFT \\
        &= \frac{\hbar^2}{2m} \frac{1}{N}\FFT^{-1} k^2 \FFT\,.
\end{split}
\end{equation}

\section{Cosine-Transform and Chebychev Coefficients}

While for $f(x) = e^{x}$, the Chebychev coefficients can be derived analytically
to be proportional to the Bessel functions, see
section~\ref{subsec:chebychev} in chapter~\ref{chap:numerics}. For a general
function, however, the coefficients are calculated via a cosine transform,
as~\cite{NdongJCP09}
\begin{equation}
  a_n = \frac{2 - \delta_{n0}}{N} \sum_{k=0}^{N-1} f(x_k) \cos(n \theta_k)\,,
\end{equation}
with $\theta_k \equiv \arccos(x_k)$, which can be implemented via a
FFT~\cite{RaoDCTBook1990}.

The sampling points $\{x_i\}$ are a drawn from the domain of $f(x)$,
corresponding to rescaled points $\{\xi_i\} \in [-1,1]$. There are two possible
choices.:
\begin{itemize}
  \item Gauss-Lobatto-Chebychev grid,
        (''closed interpolation'')
  \begin{equation}
    \xi_k = \cos\left( \frac{k \pi}{N-1}
                \right); \qquad \xi_k \in [1, -1]\,.
    \label{eq:gauss_lobatto}
  \end{equation}
  The $\xi_k$ are the extrema of the Chebychev polynomials, plus
  endpoints
  \item Gauss-Chebychev (''open interpolation'')
  \begin{equation}
    \xi_k = \cos\left( \frac{\left(k + \frac{1}{2}\right) \pi}{N}
                \right); \qquad \xi_k \in (1, -1)\,.
    \label{eq:gauss_cheby}
  \end{equation}
  The $\xi_k$ are the roots of the Chebychev polynomials.
\end{itemize}
For either choice, the Chebychev coefficients for the expansion of $f(x)$ are
calculated as described in Algorithm~\ref{al:ChebyCoeffs}.

\begin{algorithm}
  \caption{{\sc ChebychevCoefficients}
  for expansion of $f(x)$.
  \label{al:ChebyCoeffs}
  }
  \begin{algorithmic}[1]
    \Statex
    \Require{$f(x)$ with $x \in [x_{\min}, x_{\max}]$;
             maximum number $n_{\max}$  of coefficients
    }
    \Ensure{Array of Chebychev coefficients $[a_0\dots a_n]$, $n < n_{\max}$
    allowing to approximate $f(x)$ to pre-defined precision.}
    \Statex
    \Procedure{ChebyCoeffs}{$f(x)$}
     \State $\Delta = x_{\max} - x_{\min}$
     \State $\alpha = \frac{1}{2} \Delta$;
            $\beta = \alpha + x_{\min}$
     \State $F_{0:2 n_{\max}-3} = 0$
            \Comment{allocation to size $2(n_{\max}-1)$}
     \For{$k = 0:n_{\max}-1$}
        \State $\xi_{k} = \cos\left( \frac{k \pi}{n_{\max} - 1} \right)$
               {\bf or}
               $\xi_{k} = \cos\left( \frac{\left(k + \frac{1}{2} \right) \pi}
                                        {n_{\max}}
                          \right)$
        \State $F_k = f(\alpha \xi_{k} + \beta)$
     \EndFor
     \For{$k = 1:n_{\max}-2$} \Comment{mirror, without endpoints}
        \State $F_{n_{\max}-1+k} = F_{n_{\max}-1-k}$
     \EndFor
     \State $F$ = \Call{FFT}{$F$}
     \State $F = F / (n_{max}-1)$
     \State $F_0 = \frac{1}{2} F_0$
     \State $F_{n_{\max}-1} = \frac{1}{2} F_{n_{\max}-1}$
     \For{$i=0: n_{\max}-1$}
       \State $a_i = F_i$
       \IIf{$\Abs{a_i} < $~limit} exit loop with $n=i$
     \EndFor
     \State \Return $[a_0, \dots a_n]$
   \EndProcedure
  \end{algorithmic}
\end{algorithm}

