\chapter{Applications of the Fast-Fourier-Transform}
\label{AppendixFFT}

\section{The Frequency Grid}

Assume we do a numeric Fourier Transform of a pulse in time domain. The
original pulse is represented as an array of $n$ complex numbers:the time
domain energy amplitudes in atomic units. The values are ordered so that the
first value is at $t_{\text{start}}$ and the last value is at
$t_{\text{stop}}$.%
%\footnote{Note that $n$ is the length of the \emph{pulse},
%not of the global time grid:
%$n = nt-1$;
%$t_{\text{start}}=t_0 + \frac{dt}{2}$;
%$t_{\text{stop}} = T - \frac{dt}{2}$}
The points are separated by
\begin{equation}
  dt = \frac{t_{\text{stop}}-t_{\text{start}}}{n-1} = \frac{\Delta T}{n-1};
  \quad
  \Delta T=t_{\text{stop}}-t_{\text{start}}
\end{equation}

The resulting pulse spectrum is again represented as an array of
$n$ complex numbers, containing the amplitudes for frequencies between
$-\omega_{\max}$ and $+\omega_{\max}$, with
\begin{equation}
  \omega_{\max} = \begin{cases}
     \frac{\pi}{dt} = \frac{(n-1)\pi}{\Delta T}
                      & \text{if $n$ even} \\
     \frac{n-1}{n} \frac{\pi}{dt} = \frac{n-1}{n} \frac{(n-1)\pi}{\Delta T}
                      & \text{if $n$ odd}
                  \end{cases}
\end{equation}

The layout of the frequency array depends on whether $n$ is odd or even. In any
case, the frequency array consists of two parts: the first sub-array of length $l$
contains the amplitudes of the positive frequencies, the remaining sub-array of
length $n-l$ (running from $l+1$ to $n$) contains the amplitudes for the
negative frequencies.

\paragraph{$n$ even}
For an even $n$, we have $l=n/2$, and the values in the frequency array correspond
to the frequencies $$0, d\omega, \dots, \omega_{\max}-d\omega,$$ and then
$$-\omega_{\max}, -\omega_{\max}+d\omega, \dots, -d\omega.$$ This means that
\begin{equation}
  d\omega = \frac{2 \omega_{\max}}{n} = \frac{2\pi}{\Delta T} - \frac{2\pi}{n \cdot \Delta T}
\end{equation}

\paragraph{$n$ odd}
For an odd $n$, we have $l=n/2 + 1$, and the values in the frequency array correspond
to the frequencies $$0, d\omega, \dots, \omega_{\max},$$ and then
$$-\omega_{\max}, -\omega_{\max}+d\omega, \dots, -d\omega.$$ This means that
\begin{equation}
  d\omega = \frac{2 \omega_{\max}}{n-1} = \frac{2\pi}{\Delta T}
\end{equation}

If you want to ensure a given minimum spectral resolution of $d\omega$, you
have to set $\Delta T \geq 2 \pi / d \omega$

\section{Derivatives}

\newpage
\section{Cosine-Transform and Chebychev Coefficients}

