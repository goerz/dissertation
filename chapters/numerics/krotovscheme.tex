\documentclass{diss}
\usepackage{mymacros}

\usepackage{tikz}

\usepackage[psfixbb,graphics,tightpage,active]{preview}
\PreviewEnvironment{tikzpicture}

\begin{document}
%%%%%%%%%%%%%%%%%%%%%%%%%%%%%%%%%%%%%%%%%%%%%%%%%%%%%%%%%%%%%%%%%%%%%%%%%%%%%%%%
\begin{tikzpicture}[
  x=0.85cm,
  y=1.0cm,
]
  %\tikzset{help lines/.style=gray, ultra thin}
  %\tikzset{help lines/.style=white, ultra thin}
  %\draw[style=help lines, step=1cm] (0,0) grid (20,15);
\tikzstyle{every node}+=[font= \small ]

%%% propagation boxes
% lower
\draw[color=gray!20, fill=gray!20,rounded corners=10] (2,2.6) rectangle (16,4.40);
%\draw[color=gray!20, fill=gray!20,rounded corners=10] (14,2.8) rectangle (16,4.75);
% upper
\draw[color=gray!20, fill=gray!20,rounded corners=10] (2,5.60) rectangle (16,7.4);
%\draw[color=gray!20, fill=gray!20,rounded corners=10] (14,5.25) rectangle (16,7.2);

%%% forward propagation

\node (psifw1) at (3,4) {%
  \begin{tikzpicture}
    \draw (0,0)--(0.5,-0.25)--(1,0);
    \node at (0.5,0.25) {$\Psi_k(0)$};
  \end{tikzpicture}
};

\node (psifw2) at (6,4) {%
  \begin{tikzpicture}
    \draw (0,0)--(0.5,-0.25)--(1,0);
    \node at (0.5,0.25) {$\Psi_k(t_1)$};
  \end{tikzpicture}
};
\draw[->] (psifw1) .. controls +(1,-1) and +(-1,-1) .. node(epsn1)[fill=gray!20]{$\epsilon^{(i+1)}_1$} (psifw2);

\node (psifw3) at (9,4) {%
  \begin{tikzpicture}
    \draw[color=gray!20] (0,0)--(0.5,-0.25)--(1,0);
    \node[color=gray!20] at (0.5,0.25) {$\Psi_k(t)$};
    \node at (0.5,0.25) {\dots};
  \end{tikzpicture}
};
\draw[->] (psifw2) .. controls +(1,-1) and +(-1,-1) .. node(epsn2)[fill=gray!20]{$\epsilon^{(i+1)}_2$} (psifw3);

\node (psifw4) at (12,4) {%
  \begin{tikzpicture}
    \draw (0,0)--(0.5,-0.25)--(1,0);
    \node at (0.5,0.25) {$\Psi_k(t_{nt-1})$};
  \end{tikzpicture}
};
\draw[->] (psifw3) .. controls +(1,-1) and +(-1,-1) .. node(epsn3)[fill=gray!20]{$\epsilon^{(i+1)}_{nt-2}$} (psifw4);

\node (psifw5) at (15,4) {%
  \begin{tikzpicture}
    %\draw[color=gray!20](0,0)--(0.5,-0.25)--(1,0);
    \node at (0.5,0.25) {$\Psi_k(T)$};
  \end{tikzpicture}
};
\draw[->] (psifw4) .. controls +(1,-1) and +(-1,-1) .. node(epsn4)[fill=gray!20]{$\epsilon^{(i+1)}_{nt-1}$} (psifw5);

%%% backward propagation

\node (chi1) at (3,6) {%
  \begin{tikzpicture}
    \draw (0,0)--(0.5,0.25)--(1,0);
    \node at (0.5,-0.25) {$\chi_k(0)$};
  \end{tikzpicture}
};

\node (chi2) at (6,6) {%
  \begin{tikzpicture}
    \draw (0,0)--(0.5,0.25)--(1,0);
    \node at (0.5,-0.25) {$\chi_k(t_1)$};
  \end{tikzpicture}
};
\draw[<-] (chi1) .. controls +(1,1) and +(-1,1) .. node[fill=gray!20]{$\epsilon^{(i)}_1$} (chi2);

\node (chi3) at (9,6) {%
  \begin{tikzpicture}
    \draw[color=gray!20] (0,0)--(0.5,0.25)--(1,0);
    \node[color=gray!20] at (0.5,-0.25) {$\chi_k(t)$};
    \node at (0.5,-0.25) {\dots};
  \end{tikzpicture}
};
\draw[<-] (chi2) .. controls +(1,1) and +(-1,1) .. node[fill=gray!20]{$\epsilon^{(i)}_2$} (chi3);

\node (chi4) at (12,6) {%
  \begin{tikzpicture}
    \draw (0,0)--(0.5,0.25)--(1,0);
    \node at (0.5,-0.25) {$\chi_k(t_{nt-1})$};
  \end{tikzpicture}
};
\draw[<-] (chi3) .. controls +(1,1) and +(-1,1) .. node[fill=gray!20]{$\epsilon^{(i)}_{nt-2}$} (chi4);

\node (chi5) at (15,6) {%
  \begin{tikzpicture}
    \draw[color=gray!20] (0,0)--(0.5,0.25)--(1,0);
    \node[] at (0.5,-0.25) {$\chi_k(T)$};
  \end{tikzpicture}
};
\draw[<-] (chi4) .. controls +(1,1) and +(-1,1) .. node[fill=gray!20]{$\epsilon^{(i)}_{nt-1}$} (chi5);

%%% mu
\node (mu1) at (3,5) {%
  \begin{tikzpicture}
    \draw (-0.5,0.4)--(0.5,0.4);
    \node at (0,0) {$\frac{\partial \Op{H}}{\partial \epsilon}$};
    \draw (-0.5,-0.4)--(0.5,-0.4);
  \end{tikzpicture}
};
\draw[->] (mu1) -| (epsn1);

\node (mu2) at (6,5) {%
  \begin{tikzpicture}
    \draw (-0.5,0.4)--(0.5,0.4);
    \node at (0,0) {$\frac{\partial \Op{H}}{\partial \epsilon}$};
    \draw (-0.5,-0.4)--(0.5,-0.4);
  \end{tikzpicture}
};
\draw[->] (mu2) -| (epsn2);

\node (mu3) at (9,5) {%
  \begin{tikzpicture}
    \draw[color=white] (-0.5,0.4)--(0.5,0.4);
    \node[color=white] at (0,0) {$\frac{\partial \Op{H}}{\partial \epsilon}$};
    \node at (0,0) {\dots};
    \draw[color=white] (-0.5,-0.4)--(0.5,-0.4);
  \end{tikzpicture}
};
\draw[->] (mu3) -| (epsn3);

\node (mu4) at (12,5) {%
  \begin{tikzpicture}
    \draw (-0.5,0.4)--(0.5,0.4);
    \node at (0,0) {$\frac{\partial \Op{H}}{\partial \epsilon}$};
    \draw (-0.5,-0.4)--(0.5,-0.4);
  \end{tikzpicture}
};
\draw[->] (mu4) -| (epsn4);

%\node at (15,5) {%
%  \begin{tikzpicture}
%    \draw (-0.5,0)--(0.5,0);
%  \end{tikzpicture}
%};
%\node at (16.1,5) {%
%  \begin{tikzpicture}
%    \node at (0,0) {$ = \; \tau$};
%  \end{tikzpicture}
%};

%
\end{tikzpicture}

%%%%%%%%%%%%%%%%%%%%%%%%%%%%%%%%%%%%%%%%%%%%%%%%%%%%%%%%%%%%%%%%%%%%%%%%%%%%%%%%

\end{document}
