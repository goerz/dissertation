% vim: ft=tex iskeyword=@,48-57,_,-,192-255,\: dictionary=bibkeys.lst,labels.lst:
\chapter{Implementation of a RIP Transmon Gate}
\label{chap:tmresults}


\section{Entanglement Creation}

\begin{table}[htbp]
  \centering
  %\begin{tabular}{l@{\hskip 12pt}r@{\hskip 12pt}r@{\hskip 12pt}r@{\hskip 12pt}r@{\hskip 12pt}r@{\hskip 12pt}l}
  \begin{tabular}{lrrrrrl}
  \toprule
  \#  &  {$\omega_c$} & {$\omega_1$} & {$\omega_2$} & {$g$} & {$\alpha$} & Comment\\
  \midrule
  1   &  6000         & 6860         & 7250         & 70    & -300       & cavity on the left \\
  2   &  8100         & 6860         & 7250         & 70    & -300       & cavity on the right \\
  3   &  8600         & 6860         & 7250         & 70    & -300       & right, with slightly larger qubit-cavity detuning\\
  4   &  8200         & 6850         & 7350         & 70    & -250       & right, with increased qubit separation \\
  5   &  6000         & 7860         & 8250         & 70    & -300       & cavity on the left, with large qubit-cavity detuning\\
  6   &  6000         & 7860         & 8250         & 90    & -300       & cavity on the left, with large qubit-cavity detuning and large $g$\\
  \bottomrule
  \end{tabular}
  \caption{Parameter sets}
  \label{tab:RIP_entang_params}
\end{table}

Another perspective under which to analyze different parameter sets is the
creation of entanglement for Gaussian/Blackmann pulses of increasing amplitude.
We examine a set of system parameters listed in Tab.~\ref{tab:RIP_entang_params}.


\begin{figure}[htbp]
  \centering
  \includegraphics{rip_entanglement}
  \caption{Entanglement (concurrence) generated after $T=\SI{200}{ns}$, for
  Blackmann pulse of varying maximum amplitude. $d$ is the detuning of the
  drive from the cavity, $d = \omega_d - \omega_c$ (i.e.\ negative numbers
  indicate driving left of the cavity, positive number right of the cavity).
  In the legend for each line, the number of cavity and qubits
  levels included in the propagation are indicated in parenthesis, for the value
  of $E_0$ at which the entanglement reaches its first local extremum. E.g., the
  curve for $d = \SI{-60}{MHz}$ for parameter set 1 (blue curve in top
  panel) requires 30 cavity levels and 11 qubit levels in order to be converged
  for a pulse amplitude of $E_0 \approx \SI{250}{MHz}$, where a concurrence of 1.0
  is reached.
  The panels from top to bottom correspond to the parameter sets listed in
  Tab.~\ref{tab:params}}
  \label{fig:rip_entanglement}
\end{figure}

The entanglement reached after propagating for $T=\SI{200}{ns}$  for these
parameter sets, and with a drive at varying frequencies left or right of the
cavity, as given by $d$, is shown in Fig.~\ref{fig:rip_entanglement}. Note how the
entanglement generated by the drive can interfere either constructively or
destructively with the field-free entanglement. The field-free entanglement is
due to the static coupling of the qubit and the cavity.

The curve for parameter set 1, $d=-60$ shows a significant qubit level
excitation (11 qubit levels needed for numerical convergence). This is
counter-intuitive, since the cavity is left of the qubits and the drive is left
of the cavity, and thus should be far off-resonant for the qubit transition.
However, the population dynamics show an excitation $\Ket{1} \rightarrow
\Ket{3}$ of the left qubit.
