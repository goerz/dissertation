\chapter{Time-dependent Quantum Mechanics}
\label{AppendixTD}

The dynamics of a quantum state is described by the time-dependent Schrödinger
equation
\index{Schrödinger equation!time-dependent}
\begin{equation}
  \ii \hbar \frac{\partial}{\partial t} \Ket{\Psi}
  = \Op{H}(t) \Ket{\Psi}
  \label{eq:tdse}
\end{equation}

It has the formal solution $\Ket{\Psi(t)} = \Op{U}[t,0] \Ket{\Psi(0)}$ with the
time evolution operator
\index{time evolution operator}
given as
\begin{equation}
%\begin{split}
  \Op{U}[t,0]
  %&
  = \Timeorder \exp\left[
      - \frac{\ii}{\hbar} \int_{0}^{t} \Op{H}(t') \dd t'
    \right]\,,
  %\\ &
  %= 1 + \sum_{n=1}^{\infty} \frac{(-\ii)^n}{\hbar^n n!}
    %\int_{0}^{t} \dd t^{(1)} \dots \int_{0}^{t} \dd t^{(n)}
    %\Timeorder\left[ \Op{H}(t^{(1)}\dots \Op{H}(t^{(n)}\right]\,,
%\end{split}
\end{equation}
with the time ordering operator $\Timeorder$.
\index{time ordering operator}
For a time-independent Hamiltonian $\Op{H}$ the time evolution operator becomes
$\Op{U}[t,0] = \ee^{-\frac{\ii}{\hbar} \Op{H} t}$. In a numerical context, we
approximate any time-dependent Hamiltonian as piecewise constant on a time grid
with time step $dt$. Then, the total time evolution operator is the product of
\Op{U} at each time step,
\begin{equation}
  \Op{U}[t,0]
  = \prod_{i} \ee^{-\frac{\ii}{\hbar} \Op{H}(t_i) dt}\,,
\end{equation}
where $\Op{H}(t_i)$ is the (constant) Hamiltonian at the $i$'th time step.

We split the total Hamiltonian as $\Op{H}(t) = \Op{H}_0 + \Op{H}_c(t)$ into
a constant part, the \emph{drift} Hamiltonian $\Op{H}_0$, and a time-dependent
part, the \emph{interaction} or \emph{control} Hamiltonian $\Op{H}_c(t)$. For
example, we consider an atom illuminated by a laser beam, see
Appendix~\ref{AppendixLMI}. The drift Hamiltonian
in this case is diagonal and contains the atom's electronic energy levels.
The interaction Hamiltonian in the dipole approximation
\index{dipole approximation}
where the spatial dependence of the laser field can be neglected on the length
scale of the atom contains the dipole operator $\Op{\mu}$
\index{dipole operator}
coupling to the laser field
\begin{equation}
  \epsilon(t) = E_0 S(t) \cos(\omega t)
\end{equation}
with peak amplitude $E_0$, shape $S(t)$ and laser frequency $\omega$.
When we refer to the interaction Hamiltonian as the \emph{control} Hamiltonian,
we indicate that the amplitude and shape of time-dependent laser field can be
controlled to steer the system in some desired way.

% example: two-level system (Rabi-Cycling)
